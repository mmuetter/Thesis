
\newcommand{\createTukeyTable}[4]{
  \begin{table}[H]
    \centering
    \caption{#4: Multiple comparison between the effects of #2 on the frequencies of #3 using Tukey's post-hoc analysis.}
    \label{tab:#1}
    \input{chapter_2_sup/tables/#1}
  \end{table}
}

%% Commands
\newcommand{\makeCSVtable}[4]{%
  \csvreader[
    head to column names,
    separator=semicolon,
    tabular     = #1,
    table head  = \hline\bfseries #2 \\\hline,
    table foot  = \hline
  ]{#3}{}{#4}%
}
\newcommand{\transitionmatrixbody}[1]{%
  \resizebox{\linewidth}{!}{%
    \begin{tabular}{r|cccccc}
      \diagbox[width=8em,height=4em]%
      {$\phi(\hat T)$}{$\phi(T)$} &
      $U$ & $S$ & $A_r$ & $B_r$ & $(A_r\&B_r)$ & $AB_r$ \\\hline
      \csvreader[
        separator=semicolon,
        head to column names,
        column names={
          index=\cidx,
          U    =\cU,
          S    =\cS,
          Ar   =\cAr,
          Br   =\cBr,
          AuB  =\cAuB,
          ABr  =\cABr
        }
      ]{chapter_2_sup/tables/#1.csv}{}{% no “filter” needed
        \cidx & \cU & \cS & \cAr & \cBr & \cAuB & \cABr\\
      }%
    \end{tabular}%
  }%
}

\newcommand{\createANOVATable}[4]{
  \begin{table}[H]
    \centering
    \caption{#4: Effect of #2 on the frequency of #3 (ANOVA).}
    \label{tab:#1}
    \input{chapter_2_sup/tables/#1}
  \end{table}
}

\newcommand{\panelcaption}[2]{
  \centering
  % Display the title if provided
  \ifx&#1&%
  % No title provided, do nothing
  \else
  \textbf{#1}\\[1ex] % Title with a small space after
  \fi
  \captionsetup{labelformat=empty, skip=-10pt} % This makes the caption label empty
  \caption{}
  \label{#2}
}

% Define default values for xshift and yshift
\pgfkeys{
  /subpanel/.is family, /subpanel,
  default/.style = {xshift=0.15, yshift=-0.15}, % Default xshift and yshift
  xshift/.store in = \xshift, % Store xshift value
  yshift/.store in = \yshift, % Store yshift value
}

% Define the subpanel command
\newcommand{\subpanel}[5][]{ % Correct number of arguments
  \pgfkeys{/subpanel, default, #1} % Parse the key-value arguments
  % #2: Width of subfigure
  % #3: Content (e.g., \includegraphics)
  % #4: Panel title
  % #5: Label
  \setlength{\parskip}{-20pt}

  \begin{subfigure}[t]{#2}
    \panelcaption{#4}{#5}
    \begin{tikzpicture}
      % Place the image
      \node[anchor=south west, inner sep=0] (image) at (0,0) {#3};
      % Place the subcaption label using xshift and yshift
      \node[anchor=north west, inner sep=3pt] at
      ([xshift=\xshift, yshift=\yshift]image.north west) {\textbf{(\thesubfigure)}};
    \end{tikzpicture}
  \end{subfigure}
}

\newsavebox{\tikzbox} % Define a savebox for the TikZ image

\newcommand{\tikzpanel}[5][0pt]{ % Default value for shift is now 0pt
  \sbox{\tikzbox}{\input{#3}} % Save the TikZ image in the box
  \begin{subfigure}[t]{#2}
    \panelcaption{#4}{#5}
    \begin{tikzpicture}
      \node[anchor=south west, inner sep=0] (image) at (0,0) {
        \begin{adjustbox}{width=\linewidth,keepaspectratio}
          \usebox{\tikzbox}
        \end{adjustbox}
      };
      % Place the subcaption label at the top left corner of the image with horizontal shift
      \node[anchor=north west, inner sep=3pt] at ([xshift=#1]image.north west) {\textbf{(\thesubfigure)}};
    \end{tikzpicture}
  \end{subfigure}
}

% SI-style equation numbering: S1, S2, ...
\renewcommand{\theequation}{S\arabic{equation}}
\renewcommand{\eqref}[1]{Eq.~S\ref{#1}}
