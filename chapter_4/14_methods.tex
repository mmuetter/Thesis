%%% WETLAB STUFF
\FloatBarrier
\paragraph{Strains and Media.}
We used the bioluminescent strain \textit{Escherichia coli} MG1655 \(\Delta\)galK::(kan\(^R\)-luxCDABE) constructed previously \cite{Muetter2025}.
Cultures were grown in LB medium (Sigma L3022). 

\paragraph{Drug preparation.}
In the experiments, we explored the interactions among six drugs (\autoref{tab:drugs}), resulting in 15 drug-drug pairs.
For each pair, one compound was designated drug~A and the other drug~B. 
Each compound was prepared as a stock (\num{20}\,x) at the highest planned working concentration. 
We then performed a twofold dilution series in a 12-column deepwell plate, resulting in 11 diluted concentrations and one drug-free column. 
For drug~A, we transferred \SI{125}{\micro\liter} per well into a 96-well plate with a horizontal concentration gradient; this is referred to as subreservoir~A. 
For drug~B, we transferred \SI{50}{\micro\liter} per well into two 96-well plates with vertical concentration gradients (each 6x12 layout); these are the antibiotic-reservoir plates (I and II). 
All plates were stored at \SI{-80}{\celsius} to minimise degradation over time while accepting a one-time degradation due to freezing and thawing.


%% DRUG TABLE
\begin{table}
    \caption{Drugs used in the experiments, the solvent, the estimated zMIC (obtained by fitting the single-drug pharmacodynamic curves), and the supplier with catalogue number.}
    \centering
    \small
    \begin{tabular}{lccp{1.8cm}}
\toprule
\small
drug & solvent &  zMIC [\si{\micro\gram\per\milli\liter}]&  supplier \\
\midrule
amoxicillin (AMO) & water & 8.42 & ThermoFischer, J61290.14 \\
chloramphenicol (CHL) & DMSO & 54.11 & Sigma, C0378 \\
colistin (COL) & water & 0.58 &Sigma, C4461 \\
penicillin (PEN) & water & 153.86 & Roth, HP48.2 \\
polymyxin B (POL) & water & 0.51 & Roth, 0235.1 \\
tetracycline (TET) & DMSO & 15.15 &Sigma, T3383 \\
\bottomrule
\end{tabular}


\begin{comment}
\begin{tabular}{lcccp{1.8cm}}
\toprule
\small
drug & solvent & $c_{1}$ [\si{\micro\gram\per\milli\liter}] & zMIC[\si{\micro\gram\per\milli\liter}]&  supplier \\
\midrule
amoxicillin (AMO) & water & 128 & zMIC & ThermoFischer, J61290.14 \\
chloramphenicol (CHL) & DMSO & 512 && Sigma, C0378 \\
colistin (COL) & water & 128 & &Sigma, C4461 \\
penicillin (PEN) & water & 2048 && Roth, HP48.2 \\
polymyxin B (POL) & water & 128 && Roth, 0235.1 \\
tetracycline (TET) & DMSO & 64 & &Sigma, T3383 \\
\bottomrule
\end{tabular}
\end{comment}
    \label{tab:drugs}
\end{table}


\paragraph{Dose response assays.}
For each assay, we prepared four overnight cultures grown for \SI{14}{\hour}. 
Cultures were distributed to fill an entire 384-well source plate (Greiner 781073), so that the replicates formed a \(2\times2\) block format (e.g. C1--rep1, C2--rep2, D1--rep3, D2--rep4).
We then prefilled two white 384‑well assay plates (Greiner 781073) with \SI{54}{\micro\liter} LB per well and transferred inocula from the source plate to both assay plates (I, II)  using the Evo 200 liquid handling platform (Tecan) with a pintool (dilution approximately $1:150$). 
Plates were incubated for \SI{2}{\hour} to reach exponential phase. 
Simultaneously, subreservoir~A and both antibiotic‑reservoir plates were thawed. 
\SI{50}{\micro\liter} from each well of the subreservoir~A were transferred to the corresponding wells of both antibiotic‑reservoir plates, generating a \num{10}\,x mixture of drugs A and B. 
For both assay plates, the assay started as follows: a baseline luminescence reading was taken, \SI{6}{\micro\liter} of the \num{10}\,x drug mixture was added (defining \(t=0\) at dosing for the respective wells), and a second luminescence reading was taken.
Subsequently, we alternated between reading assay plates I and II for a total duration of \SI{5}{\hour}.
Luminescence was recorded with an Infinite~F200 plate reader (Tecan) using a \SI{250}{\milli\second} integration time.


\paragraph{Data preprocessing.}
To estimate light noise, we conducted a calibration experiment in which six source wells (E5, E12, E20, L5, L12, L20) contained stationary-phase cultures while all other wells remained empty. 
From a single luminescence read of the full plate, we constructed a distance-dependent light-noise kernel (\autoref{fig:light_distribution}) and corrected each well by subtracting the summed contributions from neighbouring wells (restricted to distances $d\le 3$, \autoref{eq:noise_corr}). 
We defined the lower limit of detection as \(I_{\mathrm{LoD}}=\SI{10}{RLU}\) and the upper limit of detection as \SI{1e6}{RLU}.
We defined the common analysis horizon \(T\) as the earliest time point at which any untreated control well exceeded the upper detection limit, yielding \(T\approx\SI{2}{\hour}\). 
For each well, we computed the fraction of observations up to \(T\) below \(I_{\mathrm{LoD}}\). 
Wells with more than $20\%$ of observations below \(I_{\mathrm{LoD}}\) were excluded from all subsequent analyses. 
For retained wells, values below \(I_{\mathrm{LoD}}\) were censored by replacing them with \(I_{\mathrm{LoD}}\). 
For numerical integration, we linearly interpolated the log-normalised signal to obtain values on the shared horizon \(t=T\) (see also \siref{ssec:lightnoise}). 

\paragraph{Time-weighted net growth rate.}
We define \(\psi\) as the linearly time-weighted net growth rate (SI: \autoref{eq:psi_weighted}) over the common time horizon \(T\).
We infer \(\psi\) by calculating the scaled area under the log-normalised light intensity \(Y(t)=\ln\!\big(I(t)/I(0)\big)\) (SI: \autoref{eq:norm}, \autoref{eq:Ydef}),
\begin{equation}
    \psi
    =
    \frac{2}{T^{2}} \int_{0}^{T}Y(t)\,\mathrm{d}t.
    \label{eq:effective_psi}
\end{equation}



\paragraph{Classification of inhibitory and sub-inhibitory conditions.}
We classify each checkerboard condition as inhibitory, sub-inhibitory, or not significantly different from zero by testing whether replicate time-weighted growth-rate estimates \(\psi(c_A,c_B)\) are consistently positive or negative.
With four replicates, the binomial probability of observing all positive (or all negative) signs with equal probability is \(0.5^4=0.0625\).
This classification is used to annotate the \(\psi\) heatmaps (\autoref{fig:psi_chl_tet}, \autoref{fig:psi}) and to define inhibitory and sub-inhibitory regime pools.


\paragraph{Single-drug pharmacodynamic curve fitting.}
For each drug, we aggregated all mono-treatment data across experiments and fitted a pharmacodynamic (PD) curve (\(f(c)\)) to the corresponding time-weighted growth-rate estimates \(\psi\).
\begin{equation}
    f(c) = \psi_{0} - \frac{(\psi_{0}-\psi_{\min})(c/z_{\mathrm{MIC}})^{\kappa}} {(c/z_{\mathrm{MIC}})^{\kappa}-(\psi_{\min}/\psi_{0})}, \quad [0,\infty)\rightarrow[\psi_{\min},\psi_{0}]
    \label{eq:single_drug_pd}
\end{equation}
Here, \(\psi_{0}\) denotes the maximum and \(\psi_{\min}\) the minimum time-weighted growth rate, corresponding to \(c=0\) and \(c\to\infty\), respectively.
\(\kappa\) denotes the Hill coefficient and \(z_{\mathrm{MIC}}\) the concentration at which \(f(c)=0\). 
The model parameters \(\psi_{0},\psi_{\min},\kappa,z_{\mathrm{MIC}}\) were jointly estimated by least-squares fitting of \autoref{eq:single_drug_pd}. 

\paragraph{Distribution of treatment effects.}
We define the treatment effect $\tau$ as the linearly time-weighted average difference between the untreated and treated temporal net growth rates over the common time horizon \(T\) (SI: \autoref{eq:intYdiff}). 
Equivalently, \(\tau\) is the scaled area between the control and treatment trajectories on the log-normalised scale (illustrated in \autoref{fig:peptide-interactions}b). 
Because \(\tau\) is a comparative measure between one untreated and one treated well, we bootstrap treated--control well pairs with replacement (\(B=200\); details in \siref{ssec:bootstrap_interactions}). 
For each condition \((c_A,c_B)\) and draw \(b\), we compute \(\tau_b(c_A,c_B)=\psi_b(\emptyset)-\psi_b(c_A,c_B)\), yielding the distribution of treatment effects $\bm{\tau}(c_A, c_B)$. 
Throughout this manuscript, distributions are denoted by bold symbols.
We classify such bootstrap distributions as significant if the central \(95\%\) bootstrap interval excludes zero, and as not significant otherwise. 



\paragraph{Bliss-based interaction score \(\mu\).}
We define the Bliss based interaction score as the normalised deviation from the Bliss-prediction for combined treatment effects (see \siref{ssec:bliss} and \autoref{eq:tau_bliss_main}):
\begin{equation}
    \mu_b(c_A,c_B)  =
    \frac{\tau_b(c_A)+\tau_b(c_B)-\tau_b(c_A,c_B)}
    {\mathrm{median}\!\bigl(\bm{\tau}(c_A)\bigr)+\mathrm{median}\!\bigl(\bm{\tau}(c_B)\bigr)}.
    \label{eq:mu}
\end{equation}
To avoid misinterpreting noise as interaction, we only evaluate \(\mu(c_A,c_B)\) for eligible conditions.
A condition is eligible if \(\bm{\tau}(c_A,c_B)\) and at least one of \(\bm{\tau}(c_A)\) or \(\bm{\tau}(c_B)\) are significant.


\paragraph{Loewe-based interaction score \(\nu\).}
We define a Loewe interaction score (details in \siref{ssec:loewe}) that quantifies deviations from dose equivalence as
\begin{equation}
    \nu_b(c_A,c_B)
    =
    \frac{c_A}{f_A^{-1}\!\big(\psi_b(c_A,c_B)\big)}
    +
    \frac{c_B}{f_B^{-1}\!\big(\psi_b(c_A,c_B)\big)}
    -1.
    \label{eq:fici}
\end{equation}
Here \(f_i\) denotes a pharmacodynamic function mapping concentration to net growth, \(f_i(c)=\psi\). 
The inverse \(f_i^{-1}(\psi)\) is the corresponding monotreatment-equivalent concentration of drug \(i\). 
Our definition of \(\nu\) is analogous to the combination index \((\mathrm{CI}-1)\) \cite{Chou1984}, but uses a different pharmacodynamic model \(f_i\) (either \autoref{eq:psi_inverse_closed_form} or inferred numerically from the monotreatment edges of the surface splines). 
To avoid misinterpreting noise as interaction, we only evaluate \(\nu(c_A,c_B)\) if \(\bm{\tau}(c_A,c_B)\) is significant and at least one of \(\bm{\tau}(c_A,0)\) or \(\bm{\tau}(0,c_B)\) is significant. 





\paragraph{Polar reparametrization.}
We normalise concentrations of drug \(i\) using the corresponding \(\mathrm{zMIC}_i\) estimates obtained from the single-drug PD fits (\autoref{tab:drugs}): \(z_i=c_i/\mathrm{zMIC}_i\).
We then define polar coordinates based on the normalised concentrations:
\begin{equation}
    z=\sqrt{z_{A}^{2}+z_{B}^{2}},
    \qquad
    \phi=\arctan2\!\bigl(z_{B},z_{A}\bigr),
    \label{eq:polar_coordinates}
\end{equation}
with \(\phi\) corresponding to the \emph{mixing angle} and \(z\) to the \emph{combined dose}. 
These coordinates were used to plot polar pharmacodynamic curves, which show the treatment effect over the combined dose \(z\) for a fixed mixing angle \(\phi\). 
We obtain these curves as one-dimensional cross-sections through the surface splines at fixed \(\phi\). 
