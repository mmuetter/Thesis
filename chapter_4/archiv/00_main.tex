\documentclass[times, twoside, watermark,labelsep=period, figurename=Fig.]{settings/SimpleTemplate}

% Lead author for footer
\leadauthor{Muetter}

% White space & figure placement.
\setcounter{topnumber}{5}
\setcounter{bottomnumber}{5}
\setcounter{totalnumber}{10}


\renewcommand{\topfraction}{0.95}
\renewcommand{\bottomfraction}{0.95}
\renewcommand{\textfraction}{0.05}
\renewcommand{\floatpagefraction}{0.8}

% Math extras
\usepackage{amsmath,amssymb,amsfonts}
\usepackage{mathrsfs}
\usepackage{bm}

% Floats and algorithms
\usepackage{graphicx}
\usepackage{multirow}
\usepackage{booktabs}
\usepackage{stfloats}
\usepackage{subcaption}
\usepackage{overpic}
\usepackage{placeins}


% Code and algorithm pseudocode
\usepackage{algorithm}
\usepackage{algorithmicx}
\usepackage{algpseudocode}
\usepackage{listings}

% Chemistry and scientific units
\usepackage{mhchem}
\usepackage{siunitx}
\DeclareSIUnit\ml{ml}
\sisetup{per-mode = fraction}

% Appendix, cross-referencing, citations
\usepackage[title]{appendix}
\usepackage{xr}
\usepackage{xcite}

% Misc formatting
\usepackage{colortbl}
\usepackage{enumitem}
\usepackage[switch]{lineno}

\newcommand{\siref}[1]{%
  \begingroup
  \def\sectionautorefname{SI}%
  \def\subsectionautorefname{SI}%
  \def\subsubsectionautorefname{SI}%
  \autoref{#1}%
  \endgroup
}


\renewcommand{\arraystretch}{1.25}
\setlength{\aboverulesep}{-1pt} 
\setlength{\belowrulesep}{-1pt}
\definecolor{tablecolor}{rgb}{0.4, 0.7, 0.8}

% Input custom macros, if they exist
\usepackage{xr}

%%% HELPER CODE
\makeatletter
\newcommand*{\addFileDependency}[1]{% argument=file name and extension
  \typeout{(#1)}% latexmk will find this
  \@addtofilelist{#1}% include in \listfiles
  \IfFileExists{#1}{}{\typeout{No file #1.}}% warn if file doesn't exist
}
\makeatother

\newcommand*{\myexternaldocument}[1]{%
    \externaldocument{#1}%
    \addFileDependency{#1.aux}%
}
%%% END HELPER CODE

%%% Use the relative path to your supplementary `.aux` file
\myexternaldocument{settings/si_refs} % Path to the .aux file


% Combination effects depend on dose and mixing ratio.
% Antibiotic Combination Effects at Subinhibitory Doses do not Reliably Predict Effects at Inhibitory Concentrations
% 

\title{Antimicrobial Combination Effects at Subinhibitory Doses do not Reliably Predict Effects at Inhibitory Concentrations}
\shorttitle{The Pharmacodynamics of Drug Combinations}

\author[1,\Letter]{Malte Muetter}
\author[1]{Daniel Angst}
\author[1]{Roland Regoes}
\author[1]{Sebastian Bonhoeffer}

\affil[1]{Department of Environmental Systems Science, ETH Zürich, Universitätstrasse 16, 8092 Zürich, Switzerland}



\begin{document}

\linenumbers
\maketitle

\begin{abstract}
    Assessing how combinations of antibiotics inhibit bacterial growth above and beyond the individual constituent antibiotics is riddled with several difficulties: (i) measuring death rates at clinically relevant drug concentrations is challenging, (ii) there is no unifying definition of what constitutes synergistic or antagonistic interactions, and (iii) both synergism and antagonism may be concentration dependent.
    To assess how well sub-inhibitory measurements predict inhibitory behaviour, we quantified drug interactions for 15 pairwise drug combinations on a \(12\times 12\) concentration checkerboard recording 8{,}640 time-resolved luminescence trajectories.
    To handle time-varying treatment effects and allow fair comparisons between drugs with distinct killing dynamics, we introduced a growth-integrated rate-like metric \(\psi\) to summarise each trajectory and assigned interaction types (synergistic/neutral/antagonistic) based on Bliss independence and Loewe additivity.
    We found that the interaction type frequently changes as the concentration increases from sub-inhibitory to inhibitory concentrations.
    Moreover, interaction type depended on the mixing ratio, implying that single-point sub-inhibitory measurements alone are not sufficient to predict interactions at clinically relevant concentrations.
\end{abstract}


\begin{keywords}
    antibiotic combinations,  dose-dependence, drug interactions, pharmacodynamics, time-variant treatment effects
\end{keywords}


\begin{corrauthor}
science\at maltemuetter.ch
\end{corrauthor}

% Main text
%%%%%%%%%%%%%%%%% INTRODUCTION %%%%%%%%%%%%%%%%%%%%%%%
%%%%%%%%%%%%%%%%%%%%%%%%%%%%%%%%%%%%%%%%%%%%%%%%%%%%%%%
\section{Introduction}\label{sec:introduction}
%Understanding how bacterial populations respond to selection requires knowledge of how the growth and death rates of their constituent members change as a function of the selective pressure.
Accurate characterization of changes in population size under treatment is essential for understanding the evolution of antibiotic resistance.
Commonly, the effect of treatment on bacterial populations is quantified by pharmacodynamic (PD) curves.
PD curves quantify the relationship between drug concentration and the rate of population change (net growth) \cite{Regoes2004}.
These range from no antibiotic, through sub-MIC concentrations that only reduce population growth, to super-MIC concentrations that kill bacteria and lead to a population decline.

The growth parameter most often used in PD curves is the exponential rate of change in living bacteria, \(\psi_B\), reflecting both division and death.
However, other population properties, such as changes in the number of culturable bacteria or total biomass, may also be relevant, depending on the specific biological question.

In practice, PD-curves are fitted to the rate of change of a measured proxy signal such as optical density (OD), colony-forming units (CFU) or bioluminescent light intensity.

OD is a cost-effective approach for real-time, high-throughput monitoring of culture turbidity without sacrificing the population.
OD is positively correlated (within a certain range) to cell density.
However, since OD cannot distinguish between living and dead cells, this estimate of cell density is only reliable for increasing or stable population sizes, making it unsuitable for quantifying negative rates (kill rates).

CFU assays estimate bacterial density by counting the colonies that grow on permissive agar media from plated samples. They remain the gold standard for measuring population size under both sub-MIC and super-MIC conditions and are widely used to quantify pharmacodynamic curves (e.g.\ \cite{Regoes2004, Foerster2016}).

Luminescence assays measure the light emitted by bioluminescent bacterial cultures and can be used as a proxy to estimate changes in population size.
Two main approaches for biological assays are: eukaryotic \textit{luc} systems and the prokaryotic \textit{lux} systems. The \textit{luc} system, derived from eukaryotes such as fireflies, uses an ATP-dependent luciferase that oxidises luciferin to emit light \cite{Vellend1977}. It was adapted for bacterial reporters by chromosomal integration in \textit{Mycobacterium tuberculosis} \cite{Jacobs1993} and later tested for quantifying antibiotic killing in \textit{Streptococcus gordonii} \cite{Loeliger2003}.
However, \textit{luc}-based assays are limited by sensitivity to intracellular ATP, the need for addition of a costly substrate, and luciferin degradation, making continuous measurement in the same culture impractical.

By contrast, the \textit{lux} operon of prokaryotes such as \textit{Photorhabdus luminescens} encodes all components required to sustain the bioluminescence reaction \cite{Engebrecht1985,Meighen1991}.
No external substrate is needed, so light production can be recorded continuously in the same culture, making the \textit{lux} system better suited for high-throughput applications than the luc system.
Accordingly, it has been widely used to record growth curves and quantify sub-MIC treatment effects \cite{Kishony2003,Yeh2006,Chait2007,Larsson2014,Kavcic2020,Angermayr2022}.

While high-throughput OD and luminescence measurements at sub-MIC concentrations provide valuable insights into drug effects on growth rates, the super-MIC range is clinically more relevant.
A comprehensive investigation of super-MIC population dynamics (for example, pharmacodynamics of drug combinations or resistance mutations) using CFU remains impractical, as it is labor-intensive and inherently low-throughput.

Whether lux luminescence can be extended to super-MIC ranges remains unclear, as direct comparisons between CFU- and luminescence-based measurements are scarce and so far limited to only a few drugs \cite{Salisbury1999, Beard2002, Alloush2003}.
Here we evaluate the limitations of lux luminescence assays and assess whether they can reliably quantify growth rates at super-MIC concentrations.
For this, we compare changes in light intensity with changes in CFU counts across 20 antimicrobials spanning 11 classes, including penicillins, cephalosporins, carbapenems, polymyxins, quinolones, rifamycins, tetracyclines, amphenicols, folate antagonists, fosfomycin, and antimicrobial peptides.
Luminescence- and CFU-based rates aligned for some antimicrobials (e.g., colistin, amoxicillin) but diverged for others (e.g., ciprofloxacin).
Here, we explore the potential and limitations of both methods, identify the conditions under which they align with the rate of change of population size, and discuss their implications for studying antimicrobial effectiveness across sub-MIC and super-MIC ranges.



%%%%%%%%%%%%%%%%% Results %%%%%%%%%%%%%%%%%%%%%%%
%%%%%%%%%%%%%%%%%%%%%%%%%%%%%%%%%%%%%%%%%%%%%%%%%%%%%%%
\section{Results}\label{sec:results}
% Results introduction
We quantified drug interactions for 15 pairwise combinations of six antibiotics spanning three mechanistic classes: polymyxins (colistin, COL; polymyxin~B, POL), \(\beta\)-lactams (amoxicillin, AMO; penicillin, PEN), and ribosome-targeting inhibitors (chloramphenicol, CHL; tetracycline, TET).
For each drug pair, we assessed 144 conditions in a \(12\times 12\) concentration checkerboard.
We used light intensity trajectories  $I(t)$ measured for 8{,}640 bioluminescent cultures to infer changes in population size over time.
Monotreatment trajectories are shown in \autoref{fig:single-timecourses}.

\begin{figure}
  \centering
  \begin{overpic}[width=89mm]{chapter_4/figures/peptide_interaction.pdf}
    \put(5,97){\small\textbf{(a)}}
    \put(5,66){\small\textbf{(b)}}
    \put(5,36){\small\textbf{(c)}}
    \put(45,50){\scriptsize$\mathrm{AUC}_\mathrm{ctrl} = \displaystyle \int_{0}^{T} Y_{\mathrm{ctrl}}(t)\,\mathrm{d}t$}
    \put(23,43){\scriptsize$\mathrm{AUC}_\mathrm{treat} = \displaystyle \int_{0}^{T} Y_{\mathrm{treat}}(t)\,\mathrm{d}t$}
  \end{overpic}
  \caption{
    Panel (a) shows the median normalised light-intensity trajectories \(y(t)=I(t)/I(0)\) for the combination of polymyxin~B (POL, \SI{0.5}{\micro\gram\per\milli\liter}) and tetracycline (TET, \SI{2}{\micro\gram\per\milli\liter}), together with the corresponding monotherapies and the untreated control, with shaded bands indicating the interquartile range across biological replicates.
    Panel (b) shows the log-transformed signal \(Y(t)=\ln(y(t))\) for the untreated control and the combination treatment from panel (a), together with their associated areas under the curve.
    Panel (c) shows a simple peptide--antibiotic interaction model illustrating qualitatively distinct dynamics under: no drugs, antibiotic (slow constant decline), peptide (rapid initial decline), and peptide plus antibiotic (rapid initial decline followed by a constant slow decline).
  }
  \label{fig:peptide-interactions}
\end{figure}

%%%%%%%%
%%% PEPTIDE - Antibitic Interactions
%% Effective net growth
%%%%%%%%%
\paragraph{Time-variant growth rates and treatment effects.}
Most drug-interaction metrics are based on estimates of \textit{treatment effects}, which summarise growth trajectories as a scalar.
These estimates are typically either rate-based, obtained by fitting an exponential rate of change to a growth trajectory \cite{Regoes2004, Yeh2006, Chevereau2015a, Yu2016, Angermayr2022}, or area under the curve (AUC)-based, obtained by integrating a (transformed) growth trajectory over time \cite{MacGowan2000, Lenhard2015, Rao2017, Caballero2018, Chen2020, Sanchez-Hevia2025}.
Rate-based approaches are easy to interpret and can be used directly in epidemiological models.
However, under time-variant treatment effects, inferred slopes are sensitive to the choice of the fitting window.
This sensitivity is especially problematic when assessing interactions among drugs with very different killing dynamics (e.g., in the example shown in \autoref{fig:peptide-interactions}a), where treatment effects start at different time points and persist for different durations.

Given that many observed trajectories exhibited strongly time-varying treatment effects, we summarised each trajectory by the time-weighted net growth rate \(\psi\) (Methods), defined as a time-weighted average of the instantaneous net growth rate \(\hat{\psi}(t)\) (\autoref{eq:psi_weighted}).
This way, we emphasise early treatment effects, as changes in net growth that occur earlier affect the trajectory longer than equally sized changes occurring later.
Equivalently, \(\psi\) is proportional to the area under the log-normalised luminescence trajectory \(Y(t)=\ln(I(t)/I(0))\) (\autoref{eq:Ydef}).
Treatment effects are then defined as the difference between the time-weighted net growth rates of an untreated control and the treated condition, and correspond to the scaled area between the curves (\autoref{fig:peptide-interactions}b).

Median time-weighted net growth rates (\(n=4\)) for all conditions are shown in \autoref{fig:psi_chl_tet} and \autoref{fig:psi}, with markers indicating the classification as inhibitory (stars), sub-inhibitory (diamonds), or intermediate (no marker). For further details see Methods.
We then fitted single-drug pharmacodynamic curves to the monotreatment estimates (\autoref{fig:single-pdcurves}).

For each condition $(c_A, c_B)$ we estimate the bootstrap distribution \(\bm{\tau}(c_A,c_B)\) by resampling pairs of control and treatment wells with replacement (Methods, \siref{ssec:bootstrap\_interactions}).
Throughout this manuscript, we denote distributions with bold symbols.
For each bootstrap draw \(b\), we compute \(\tau_b(c_A,c_B)=\psi_b(\emptyset)-\psi_b(c_A,c_B)\).

\begin{figure}
  \centering
  \includegraphics[width = 89 mm]{chapter_4/figures/psi_chloramphenicol_tetracycline.pdf}
  \caption{
    Time-weighted growth rate \(\psi\) for combinations of CHL and TET.
    Each cell corresponds to one concentration pair, with colour indicating the median of $\psi$ across \(n=4\) biological replicates.
    Markers denote whether the distribution of $\psi$ is significantly different from zero (see Methods): stars indicate significantly positive values (net growth), and diamonds indicate significantly negative values (net killing).
  }
  \label{fig:psi_chl_tet}
\end{figure}

%%%%%%%%
%% Interaction Models
%%%%%%%%%
\paragraph{Bliss independence implies additive treatment effects.}
Because the original Bliss formulation is probabilistic, we translate it into our growth-based treatment-effect framework (see \siref{ssec:bliss}).
Bliss independence implies multiplicative survival  \(S_{AB}(T)=S_A(T)\,S_B(T)\) (\autoref{eq:bliss_prob}).
After substituting \(S_i(T)=\exp\!\big(-\int_0^T \hat{\tau}_i(u)\,du\big)\) with \(\hat{\tau}_i(t)\), denoting the \emph{instantaneous treatment effect}, we can derive that Bliss implies additive \emph{time-weighted treatment effects}:
\begin{equation}
  \tau^{\mathrm{Bliss}}(c_A,c_B)=\tau(c_A)+\tau(c_B).
  \label{eq:tau_bliss_main}
\end{equation}

% Not A Toy model.
\paragraph{Peptide -- non-peptide interaction model.}
We observed that a combination of non-peptide drugs with relatively constant inhibition dynamics and peptide-like drugs, which induce a sharp early decline followed by (almost) unimpaired growth, is dominated by the peptide drug in the early phase and by the non-peptide drug in the later phase.
A well-behaved example of such a combination is shown in \autoref{fig:peptide-interactions}a for polymyxin~B (POL) and tetracycline (TET).
Because most interaction metrics do not explicitly account for time-varying effects, it is unclear what combined treatment effects to expect.
We, therefore, constructed a simple model for an idealised scenario.
Drug~A immediately reduces the population by a factor \(\alpha\) but does not affect the subsequent dynamics, while drug~B acts invariantly over time by reducing the constant net growth rate from \(\hat{\psi}_{\mathrm{ctrl}}\) to \(\hat{\psi}_{B}\).
In combination, the population is first reduced by \(\alpha\) and then grows with \(\hat{\psi}_{B}\).
Substituting our rate definition (\autoref{eq:effective_psi}) into this model yields additive treatment effects, \(\tau_{AB}=\tau_A+\tau_B\) (see \siref{ssec:toy_model}), equivalent to the Bliss prediction derived above.

\paragraph{Interaction scores $(\mu, \nu)$.}
Based on the Bliss prediction  of combined treatment effects derived above, we define the Bliss interaction score \(\mu\) as the normalised deviation from the prediction (\autoref{eq:mu}).
The distribution of \(\mu\) can be inferred based on the distributions of treatment effects $\bm{\tau}$ described above.
For Loewe additivity, we define the interaction score \(\nu\) as the deviation from dose equivalence (\autoref{eq:fici}).
We obtain \(\nu_b\) from the inverted single-drug pharmacodynamic curves \(f_i^{-1}(\psi_b)\) (\autoref{eq:psi_inverse_closed_form}).
We then assign an interaction type (synergistic, neutral, antagonistic) depending on whether the \(95\%\) interval of \(\bm{\mu}\) or \(\bm{\nu}\) lies below, contains, or lies above zero.
The resulting interaction types based on  \(\bm{\mu}\) and \(\bm{\nu}\) for each combination are shown in Figures \ref{fig:mu_interaction_heatmap} and \ref{fig:fici_interaction_heatmap}.

%%%%%%%%
%% Bootstrap treatment effects
%%%%%%%%%
\paragraph{Disagreement between sub-inhibitory and inhibitory  interaction types within reference models.}
Using the per-condition distributions of interaction scores  described above, we next assessed whether interaction classifications remain consistent across concentration regimes.
We normalised concentrations \(c_i\) by the corresponding \(\mathrm{zMIC}_i\) (defined by \(f_i(c_i) =0\); \autoref{tab:pd_curve_parameter}), setting \(z_i=c_i/\mathrm{zMIC}_i\).
We then separately summarised inhibitory and sub-inhibitory conditions using a second, higher-level bootstrap.
Specifically, we sampled conditions \((c_A,c_B)_r\) for \(r=(1,\dots,200)\) with replacement, weighted by their mixing ratio (\autoref{eq:weights}).
For each sampled condition, we drew one estimate from the condition's distributions \(\bm{\mu}(c_A,c_B)\) and \(\bm{\nu}(c_A,c_B)\).
To compare regimes and reference models, we define three alignment classes: agreement (same classification), soft disagreement (neutral in one but synergistic or antagonistic in the other), and strong disagreement (opposite classifications).
The resulting comparisons between sub-inhibitory and inhibitory regimes for \(\bm{\nu}_{\mathrm{sub}}\), \(\bm{\nu}_{\mathrm{inh}}\) and \(\bm{\mu}_{\mathrm{sub}}\), \(\bm{\mu}_{\mathrm{inh}}\) are shown in \autoref{fig:interation-distributions}a,b and the resulting interaction types in \autoref{tab:intearction_summary}.

Under Bliss independence, we observed six combinations with agreement and 9 cases of disagreement, of which one resulted in strong disagreement (AMO+PEN) and eight resulted in soft disagreement.
Under Loewe additivity, we also observed six combinations with agreement and 9 cases of disagreement.
Here, we observed two cases of strong disagreement (COL+TET and POL+TET) and seven with soft disagreement.

\begin{figure*}
  \centering
  \begin{overpic}[width=189mm]{chapter_4/figures/distributions_triptych.pdf}
  \put(5,44){\large\textbf{a)}}
\put(38,44){\large\textbf{b)}}
\put(70,44){\large\textbf{c)}}
\end{overpic}

\caption{
Two-dimensional summaries of interaction estimates across regimes and models.
Panels \textbf{(a)} and \textbf{(b)} compare sub--inhibitory vs.\ inhibitory interaction scores, using \textbf{(a)} the Bliss interaction score $\mu$ and \textbf{(b)} the Loewe interaction score $\nu$.
Each point represents one drug combination, plotted as the median estimate in the sub--inhibitory regime (x-axis) against the median estimate in the inhibitory regime (y-axis), with 95\% bootstrap intervals shown as horizontal and vertical error bars.
\textbf{(c)} Loewe vs.\ Bliss comparison across regimes: each point corresponds to one combination and regime, plotted as the median Loewe interaction score $\nu$ (x-axis) against the corresponding Bliss interaction score $\mu$ (y-axis), with 95\% bootstrap intervals.
Markers encode the drug combination (shared legend for panels \textbf{(a)} and \textbf{(b)}), while marker shape encodes the regime in panel \textbf{(c)}.
Colors indicate classification agreement between the two compared axes in each panel:
\emph{agreement} if both classifications match (N--N, S--S, A--A),
\emph{soft disagreement} if one classification is neutral and the other is non-neutral (N--S, N--A),
and \emph{strong disagreement} if the classifications are opposite (S--A).
}
\label{fig:interation-distributions}
\end{figure*}

% BLISS VS LOEWE
\paragraph{Disagreement between reference models, within concentration regimes.}
\autoref{fig:interation-distributions}c replots the same interaction summaries described above, but now compares the Bliss interaction score \(\mu\) to the Loewe interaction score \(\nu\) across both sub-inhibitory and inhibitory regimes in a single panel.
\autoref{fig:si-loewe-vs-bliss-2d}a,b show the same comparison but  separated by regime.
Across all 30 (2\(\times\)15) comparisons, the Bliss and Loewe-based classifications agree in 14 cases, show soft disagreement in 15 cases, and show strong disagreement in one case (CHL+TET) (\autoref{fig:interation-distributions}c).

%%%%%%%%
%% SURFACE PART
%%%%%%%%%
\paragraph{Interaction types can change with dose even at fixed mixing ratio.}
Above, we compared interaction summaries between sub-inhibitory and inhibitory regimes by aggregating condition-wise estimates across a range of doses and mixing ratios.
We next ask whether interaction types also change (i) as the dose increases at a fixed mixing ratio, and (ii) as the mixing ratio varies at a fixed effect level.
To facilitate both analyses, we reparameterize concentration pairs \((c_A,c_B)\) in polar coordinates \((z,\phi)\), where \(z\) is the combined dose and \(\phi\) is the mixing angle (\autoref{eq:polar_coordinates}).

For each drug combination, we fitted 25 continuous, monotonically decreasing surface splines on bootstrap datasets of \(\psi(c_A,c_B)\) (\autoref{fig:surface_psi}).
Based on these splines, we estimate \emph{polar pharmacodynamic curves}, which show \(\psi\) as a function of the combined dose \(z\) at a fixed mixing angle \(\phi=45^\circ\) (equal mixing in units of \(\mathrm{zMIC}\)).
At a given dose, predicted values above the observed \(\psi\) indicate synergy, whereas predicted values below the observed \(\psi\) indicate antagonism.
For both reference models, there are examples where the interaction changes direction as the combined dose increases.
For the combination AMO+COL (\autoref{fig:surface_plots}a), the Loewe-based interaction shifts from antagonism at lower doses to synergy at higher doses.
For AMO+PEN (\autoref{fig:surface_plots}b) Bliss-based interaction flips, with synergy at lower doses and strong antagonism at higher doses.
\emph{Polar pharmacodynamic curves} at \(45^\circ\) for all combinations are shown in \autoref{fig:surface_pd}.

\paragraph{Interaction types can depend on the mixing ratio.}
To assess whether interaction types depend on the mixing ratio at a fixed effect level, we extracted isoboles from the median surface spline, i.e., the path \((z,\phi)\) along which the time-weighted net growth rate is constant (\(\psi=\SI{0}{\per\hour}\); \autoref{fig:surface_isoboles}).
Along each isobole, we evaluate Bliss- and Loewe-based predictions and plot these predictions as a function of \(\phi\).
For most combinations, the inferred interaction type is stable across \(\phi\), as exemplified by CHL+TET (\autoref{fig:surface_plots}c).
However, some combinations show mixing-ratio dependence, as illustrated for COL+PEN (\autoref{fig:surface_plots}d).
Plots for all combinations are shown in \autoref{fig:surface_angular_interaction}.

%% FIGURE 4
\begin{figure}
\centering
\begin{overpic}[width = 89mm]{chapter_4/figures/surface_plot.pdf}
\put(2,95){\textbf{(a)}}
\put(2,50){\textbf{(c)}}
\put(51,95){\textbf{(b)}}
\put(51,50){\textbf{(d)}}
\end{overpic}
\caption{
Panels (a,b) show \emph{polar pharmacodynamic curves} at \(\phi=45^\circ\) (corresponding to a \(1{:}1\) ratio in units of \(\mathrm{zMIC}\)) for (a) AMO+COL and (b) AMO+PEN.
The x-axis shows the combined dose \(z\), where \(z=1/\sqrt{2}\) (blue dotted line) corresponds to both single-drug doses equaling \(0.5\,\mathrm{zMIC}\), and \(z=\sqrt{2}\) (blue dashed line) corresponds to both equaling \(1\,\mathrm{zMIC}\).
Panels (c,d) show the Bliss- and Loewe-based predictions for \(\psi\) over the mixing angle \(\phi\) along the observed isobole at \(\psi=\SI{0}{\per\hour}\) for (c) COL+PEN and (d) CHL+TET.
}

\label{fig:surface_plots}
\end{figure}

\paragraph{Inoculum effects.}
We noticed a much larger-than-expected variation in pre-treatment light intensity $I_0$ in our data, which corresponds to the size of the inoculum.
To assess the impact of this variation on the results, we regressed \(\psi\) for each single-drug and concentration on the pre-treatment signal \(I_0\) (\siref{ssec:inoculum}).
We found negligible inoculum effects for AMO, CHL, PEN, and TET, but substantial effects for COL and POL at intermediate concentrations (see \autoref{fig:inoculum}).
Because the size of the inoculum did not show a significant directional trend along the concentration index (\(p=0.092\)), the variance of the inoculum mainly adds noise at intermediate concentrations of COL and POL, contributing to an increased scatter in \autoref{fig:single-pdcurves}c,e.



%%%%%%%%%%%%%%%%% Discussion %%%%%%%%%%%%%%%%%%%%%%%
%%%%%%%%%%%%%%%%%%%%%%%%%%%%%%%%%%%%%%%%%%%%%%%%%%%%%%%
\section{Discussion}\label{sec:discussion}
We quantified interaction patterns across a wide concentration range for several drug combinations with different modes of action. 
Quantifying the size of the bacterial population is notoriously difficult because bacterial death is not well defined, has multiple aspects, and no single method captures all of them \cite{Wu2024}. 
In this work, we used bioluminescence as a proxy for population size. 
This choice enabled us to record \num{8640} finely time-resolved (every 10 minutes for five hours) growth trajectories at high throughput. 
The approach assumes that the mean per-cell luminosity remains approximately constant over time, which may not always hold. 
Consequently, we restricted our analysis to drugs for which luminescence was previously shown to track population dynamics reasonably well \cite{Muetter2025}, hence, we do not expect qualitatively different conclusions when using an alternative readout. 

For many of these trajectories, particularly those treated with a polymyxin, the light intensity dropped below the detection limit within the first few minutes.
In principle, slope-based estimates can still be inferred for such curves. 
However, comparing these estimates to trajectories with weaker but constant treatment effects is conceptually hard to justify. 
To enable fair comparisons across drugs with different onset times and killing dynamics, we used a weighted growth measure integrated over a longer timeframe, which is equal for all trajectories (\(T \approx \SI{2}{\hour}\)).
However, this choice comes at the cost of having to discard all trajectories that fall below the detection limit within that timeframe. 

To investigate how these different treatment dynamics combine, we developed a simplified antibiotic--peptide interaction model.
This model predicts that combinations of a short-acting, peptide-like drug with a time-invariant drug should follow Bliss independence, while drugs with similar mechanisms should follow Loewe additivity.
We observed these behaviours qualitatively across combinations to varying degrees in \autoref{fig:surface_isoboles} and \autoref{fig:surface_pd}.

To address our core question—how predictive sub-MIC interaction patterns are for inhibitory interactions—we aggregated interaction estimates for sub-inhibitory and inhibitory conditions and compared them.
For both reference models, we observed that more than half of the combinations showed soft disagreement between inhibitory and sub-inhibitory regimes.
This does not necessarily imply that sub-inhibitory measures are uninformative, because soft disagreement involves a neutral classification in one regime, and neutrality can arise from variance. 
This variance is partly a consequence of aggregating across a diverse set of conditions covering a wide range of mixing ratios, which we observed can influence the interaction type (\autoref{fig:surface_angular_interaction}).
Importantly, for both reference models, we observed more cases of synergistic or antagonistic agreement than strong disagreement, indicating that sub-inhibitory interaction measures retain some qualitative predictive value.

Our results also confirm the practical limitations of Loewe-based interaction measures at high concentrations that have been reported previously (\cite{Meyer2019}). 
Since Loewe relies on the inverse of the pharmacodynamic function of the single drugs, it is only defined when the combination effect lies within the dynamic ranges of both monotreatment effects. 
For drug pairs with strongly different maximal killing capacities, this condition fails in large parts of the checkerboard (see undefined regions in \autoref{fig:fici_interaction_heatmap}). 
This is a severe limitation that prohibits the quantification of drug interactions for a large number of therapeutically relevant conditions.
Consistent with previous work (e.g. \cite{Vlot2019}), we found that the Bliss and Loewe frameworks can produce contrasting classifications.

Our findings show that conclusions about synergy or antagonism depend on the concentration range, mixing ratio, and the chosen interaction model. 
Accordingly, single-point measurements at a single sub-inhibitory concentration are insufficient to reliably characterise drug interactions at clinically relevant inhibitory concentrations.

\begin{comment}
For a subset of combinations, we found unambiguous interaction classifications across conditions and models (see \autoref{tab:intearction_summary}). 
Specifically, we consistently observed antagonism (AAAA) for COL--PEN, AMO--CHL, AMO--POL, and CHL--PEN, with PEN--TET and AMO--COL showing similarly strong trends (three times antagonism and one neutral classification). 
The observation that pairing \(\beta\)-lactams with ribosome-targeting drugs is antagonistic is well supported in the literature (e.g. by \cite{Jawetz1957, Ocampo2014}), whereas reports for penicillins combined with polymyxins are comparatively scarce and heterogeneous. 
In contrast, we consistently observed synergy (SSSS) for CHL--POL, while CHL--COL also tended towards synergy but with more mixed classifications (ASNS). 
For polymyxins, synergy with chloramphenicol has also been reported in the colistin-combination literature.
\end{comment}


%%%%%%%%%%%%%%%%%%%%%%%%%% METHODS      %%%%%%%%%%%%%%%%%%%%%%
%%%%%%%%%%%%%%%%%%%%%%%%%%%%%%%%%%%%%%%%%%%%%%%%%%%%%%%
\section{Methods}\label{sec:methods}
\paragraph{Strains.}
We generated a bioluminescent strain by integrating a modified \textit{P. luminescens luxCDABE} operon, driven by the constitutive $\lambda$-Pr promoter, together with a kanamycin resistance cassette (as a marker) into the chromosome of \textit{Escherichia coli} MG1655.
This integration replaced the galK gene and was achieved using $\lambda$-Red recombination (\cite{Datsenko2000}), following a protocol by Hughes \cite[19--26]{Hughes2015}.
The integrated elements were derived from the pCS-$\lambda$ plasmid (\cite{Kishony2003, Bjarnason2003}).
Primers are listed in \sitab{6}.
For all time-course experiments, we prepared three replicate exponential cultures by diluting overnight cultures (grown for approximately 18 hours) 1:100 and growing them to exponential phase for 1--1.5 hours.

\paragraph{Media.}
We used LB (Sigma L3022) as a liquid medium and, as a solid medium for CFU plating, LB with \SI{1.5}{\percent} agar.
Cultures were treated by diluting the tenfold working concentration of one of 20 antimicrobials 1:10.
All MICs, determined by broth microdilution (\cite{EUCAST2025}), and the concentrations used are listed in \sitab{1}.
Working concentrations were centered around the MIC, with some variation due to rounding convenience and variability in repeated MIC tests.
For colistin and polymyxin B, we used lower concentrations, as higher concentrations in our setup consistently yielded too few colonies for meaningful analysis.

Phosphate-buffered saline (PBS, Sigma 79383) was used as the diluent for CFU assays.
If cultures were treated with pexiganan, \SI{100}{\milli M} \ce{MgCl2} was added.
For microscopy, we added \SI{1}{\micro\gram\per\milli\liter} propidium iodide (PI) to the liquid medium and used PBS/PI agar plates (containing PBS with \SI{1.5}{\percent} agar and \SI{1}{\micro\gram\per\milli\liter} PI) as solid medium.

\paragraph{Automated CFU plating.}
We automated the high-throughput colony-count method described by \cite{Jett1997} using an Evo 200 liquid-handling platform (Tecan) integrated with a Liconic STX100 incubator.
The platform handles liquids and automatically moves, images, and incubates plates.
We produced six colony streaks by spotting six \SI{10}{\micro\liter} drops of diluted bacterial culture onto a one‑well agar plate.
Plates were automatically tilted for \SI{7}{\second} on a custom‑built tilter integrated into the handling platform, to spread the drops and distribute the bacteria.
After incubation, plate images were captured using the Pickolo camera (SciRobotics).

The platform is controlled by custom-generated worklists executed in the native software ``Evoware''.
These worklists were generated using the Python package \texttt{pypetting} (version 1.0.1).
We analyzed the captured images of the agar plates using a custom colony-recognition script that automatically identifies colonies and allowed the manual addition of unidentified colonies and the removal of mismatched ones.
All Python classes for generating the worklists and analyzing colonies are available at Zenodo (DOI: \href{https://doi.org/10.5281/zenodo.15261184}{10.5281/zenodo.15261184}).

\paragraph{Luminescence measurements.}
To record the luminescent light intensity, we used an Infinite F200 spectrophotometer plate reader (Tecan), which is also integrated into the liquid-handling platform, with an exposure time of one second.
We set \(20\;\mathrm{rlu}\) as the lower detection limit and excluded all data points below.

\paragraph{Luminescence-CFU assay setup.}
To measure the CFU and light intensity at seven time points, we treated the exponential cultures and then distributed them onto seven (one for each time point) white 384-well plates (Greiner, 781073), with each culture well containing \SI{54}{\micro\liter} medium and \SI{6}{\micro\liter} 10x stock solution.
We adjusted the duration of the experiments between two and five hours, depending on the anticipated kill rate.
For each time point, an assay plate was transferred from the incubator to the plate reader for luminescence measurement.
Subsequently, a dilution series was conducted directly in the white plate and plated using the automated plating method, after which the plate was discarded.

\paragraph{Rapid luminescence-CFU assay setup.}
This experiment is a variation of the \textit{Luminescence-CFU assay setup}, adjusted to measure rapid kill curves for the antimicrobial peptide pexiganan.
In this setup, we captured four time points within \SI{5}{\minute}.
Cultures were treated in a 96-deep-well plate (Greiner, 780285) by adding \SI{100}{\micro\liter} of the 10x stock solution to \SI{900}{\micro\liter} exponential phase culture.
\SI{60}{\micro\liter} of the treated culture was then transferred to a 384-well white plate (Greiner, 781073) and placed in the plate reader for continuous luminescence recording.
For the four CFU time points, samples were taken directly from the deep-well plate, automatically diluted in PBS supplemented with 100mM \ce{MgCl2} in a 96-well plate (Greiner, 655101) to halt the antimicrobial activity and then plated.

\paragraph{Morphology experiments.}
To assess treatment-induced morphological changes, we imaged treated (for 2 hours) and untreated bacteria by spotting \SI{2}{\micro\liter} droplets onto PBS/PI-agar plates.
The spots were cut out and flipped onto Ibidi $\mu$-dishes (Ibidi, 80136) for imaging.
We used an Eclipse Ti2 microscope (Nikon) with a 100x objective connected to a DS-Qi2 Nikon Scientific CMOS (sCMOS) camera to image the bacterial cells.
The microscope setup included an additional 1.5x zoom, which was used only for some images due to unintentional variation.
We estimated the width and length of the bacterial cells using a custom Python script, as described in \siappendix{4}.

\paragraph{Fitting rates of change $\psi$.}
To compare the rates of change of two signals, we first excluded all data below the detection limits (empty plates or light intensity below \SI{20}{rlu}).
We then truncated both signals at the latest time point where both remained above the detection limit, ensuring the same time frame was used for comparison.
Next, we bootstrapped 200 datasets with replacement per signal, while ensuring that each dataset contained more than one time point
For each dataset, we applied a simple regression to fit an exponential function to all time points of each time-kill curve resulting in distributions with 200 rate estimates each.

\paragraph{Significance.}
We classify two distributions of rates as not significantly different (n.s.) if the mean of each distribution falls within the 95\% confidence interval of the other.
Otherwise, we classify them as significantly different (*).



%%%%%%%%%%%%%%%%%%%%%%%% Data Availability %%%%%%%%%%%%%%%%%%%
%%%%%%%%%%%%%%%%%%%%%%%%%%%%%%%%%%%%%%%%%%%%%%%%%%%%%%%

\section{Data, Materials, and Software Availability}
Experimental datasets and code are available at Zenodo (DOI: \href{https://doi.org/10.5281/zenodo.18374151}{10.5281/zenodo.18374151}). 


\section*{Acknowledgements}
We thank ETH Zurich for funding this work. 
During manuscript preparation, we used OpenAI’s ChatGPT for editorial assistance (coding, grammar, and proofreading). 


% References
\section*{Bibliography}
\bibliographystyle{settings/SimpleBib}
\bibliography{settings/combination_paper}

\end{document}
