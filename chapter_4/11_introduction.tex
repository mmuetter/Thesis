
% AMR IS BAD.
Antimicrobial resistance (AMR) poses a high global health burden, having caused an estimated 1.14 million deaths in 2021 \cite{Naghavi2024}, while the time until clinical resistance evolves against new drug classes is decreasing \cite{Witzany2020}.
To develop a more sustainable approach to antibiotic use, a deeper understanding of the influence of treatment on the evolution of resistance is necessary.

% COMBINATION IS COOL
One strategy that has been shown to slow the evolution of resistance in notoriously fast-evolving pathogens, such as HIV, malaria, and tuberculosis, is combination therapy \cite{Gulick1997, Goldberg2012}.
In theoretical models, combination therapy seems to outperform other strategies (such as mixing or cycling) \cite{Bonhoeffer1997, Tepekule2017, Uecker2021}, and it also seems to perform best in in vitro simulations \cite{Angst2021, Muetter2024}. 
However, clinical evidence for the efficacy of combination therapy is inconclusive, and a comprehensive meta-analysis did not identify a significant overall benefit \cite{Siedentop2024}.

% DRUG INTERACTIONS
Several factors contribute to the inconclusiveness of clinical studies, ranging from ethical constraints limiting the selection of focal pathogens (i.e. not HIV or \emph{M. tuberculosis}) to the lack of statistical power to track resistance evolution \cite{Siedentop2024}.
Another potential reason is drug interactions. 
Drug combinations can exhibit synergistic or antagonistic  interactions, depending on whether the combined effect is stronger or weaker than the expectation based on the effects of the single drugs. 
However, defining this expectation is non-trivial, as null models are numerous and the best choice to describe drug neutrality remains debated (\cite{Foucquier2015, Duarte2022}).  
Here, we use two classical and widely applied reference frameworks to define neutrality: the response-based Bliss independence \cite{Bliss1939} and dose-based Loewe additivity \cite{Loewe1926}. 


% BLISS 
Bliss independence is based on the assumption that the probability of being affected by one drug is statistically independent of being affected by the other \cite{Bliss1939}:
\begin{equation}
    p_{AB}^\mathrm{Bliss}=p_A+p_B-p_A\,p_B .    
\end{equation}
Under this assumption, Bliss independence is expected to be more appropriate for drug pairs with distinct and non-overlapping mechanisms of action \cite{Baeder2016}.


% LOEWE
In contrast, Loewe additivity -- introduced as iso-additivity by Frei in 1913 \cite{Frei1913} and later formalised by Loewe \cite{Loewe1926} -- defines neutrality through dose equivalence: a combination is considered neutral if pairs of concentrations that produce the same effect $\psi$ (i.e. a measure of net growth) lie on a linear isobole connecting the corresponding single-drug doses \cite{Loewe1926}: 
\begin{equation}
\frac{c_A}{f_A^{-1}\!\big(\psi(c_A, c_B)\big)}
+
\frac{c_B}{f_B^{-1}\!\big(\psi(c_A, c_B)\big)}
=
1. 
\label{eq:loewe_core}
\end{equation}
Here $f_i^{-1}$  represents the inverse of a pharmacodynamic function  $f_i(c)$ that maps drug concentrations $c_i$ to net growth rates $\psi$ \cite{Regoes2004}. 


%% PD STUFF
There are already many studies on  drug interactions below the minimum inhibitory concentration (MIC)  (e.g. \cite{Yeh2006, OShaughnessy2006, Chevereau2015, Russ2018, Katzir2019, Kavcic2020, Rezzoagli2020, Sanchez-Hevia2025, huntelman2023}); or at the  MIC level (e.g. \cite{Yang2017, Ruden2019, Liu2022}).
Studies covering the arguably clinically more relevant inhibitory regime are scarce and often cover only a few points in the inhibitory range (e.g. \cite{Ocampo2014, Rao2017, Yu2016, Caballero2018, Chen2020}).
The primary reason for this is methodological: under sub-inhibitory conditions, growth rates can be easily assessed at high throughput using optical density measurements, whereas inhibitory conditions typically require labour-intensive, low-throughput colony-forming unit assays \cite{Bognar2024}. 
This lack of data on effects at high concentrations represents a problem, not only because inhibitory conditions are more therapeutically relevant but also because the right tail of the pharmacodynamic curve has a strong influence on resistance evolution \cite{Yu2018}.

%WHAT we do
Although it is well established that drug interactions can depend on dose and mixing ratio \cite{Chou1984,OShaughnessy2006, Meletiadis2007, Rezzoagli2020, Kavcic2021, Meletiadis2007}, it remains unclear how reliably sub-inhibitory interaction measurements predict interaction behaviour in the inhibitory regime.
Here, we provide an extensive map of interactions for 15 pairs of antibiotics on $12\times 12$ concentration checkerboards that include inhibitory concentrations, using time-resolved bioluminescence trajectories.
Using Bliss independence and Loewe additivity, we quantify how often sub-inhibitory interaction classifications generalise to inhibitory conditions and how often they fail, including cases of qualitative reversals across regimes.

