% Results introduction
We quantified drug interactions for 15 pairwise combinations of six antibiotics spanning three mechanistic classes: polymyxins (colistin, COL; polymyxin~B, POL), \(\beta\)-lactams (amoxicillin, AMO; penicillin, PEN), and ribosome-targeting inhibitors (chloramphenicol, CHL; tetracycline, TET).
For each drug pair, we assessed 144 conditions in a \(12\times 12\) concentration checkerboard.
We used light intensity trajectories  $I(t)$ measured for 8{,}640 bioluminescent cultures to infer changes in population size over time.
Monotreatment trajectories are shown in \autoref{fig:single-timecourses}.

\begin{figure}
  \centering
  \begin{overpic}[width=89mm]{chapter_4/figures/peptide_interaction.pdf}
    \put(5,97){\small\textbf{(a)}}
    \put(5,66){\small\textbf{(b)}}
    \put(5,36){\small\textbf{(c)}}
    \put(45,50){\scriptsize$\mathrm{AUC}_\mathrm{ctrl} = \displaystyle \int_{0}^{T} Y_{\mathrm{ctrl}}(t)\,\mathrm{d}t$}
    \put(23,43){\scriptsize$\mathrm{AUC}_\mathrm{treat} = \displaystyle \int_{0}^{T} Y_{\mathrm{treat}}(t)\,\mathrm{d}t$}
  \end{overpic}
  \caption{
    Panel (a) shows the median normalised light-intensity trajectories \(y(t)=I(t)/I(0)\) for the combination of polymyxin~B (POL, \SI{0.5}{\micro\gram\per\milli\liter}) and tetracycline (TET, \SI{2}{\micro\gram\per\milli\liter}), together with the corresponding monotherapies and the untreated control, with shaded bands indicating the interquartile range across biological replicates.
    Panel (b) shows the log-transformed signal \(Y(t)=\ln(y(t))\) for the untreated control and the combination treatment from panel (a), together with their associated areas under the curve.
    Panel (c) shows a simple peptide--antibiotic interaction model illustrating qualitatively distinct dynamics under: no drugs, antibiotic (slow constant decline), peptide (rapid initial decline), and peptide plus antibiotic (rapid initial decline followed by a constant slow decline).
  }
  \label{fig:peptide-interactions}
\end{figure}

%%%%%%%%
%%% PEPTIDE - Antibitic Interactions
%% Effective net growth
%%%%%%%%%
\paragraph{Time-variant growth rates and treatment effects.}
Most drug-interaction metrics are based on estimates of \textit{treatment effects}, which summarise growth trajectories as a scalar.
These estimates are typically either rate-based, obtained by fitting an exponential rate of change to a growth trajectory \cite{Regoes2004, Yeh2006, Chevereau2015a, Yu2016, Angermayr2022}, or area under the curve (AUC)-based, obtained by integrating a (transformed) growth trajectory over time \cite{MacGowan2000, Lenhard2015, Rao2017, Caballero2018, Chen2020, Sanchez-Hevia2025}.
Rate-based approaches are easy to interpret and can be used directly in epidemiological models.
However, under time-variant treatment effects, inferred slopes are sensitive to the choice of the fitting window.
This sensitivity is especially problematic when assessing interactions among drugs with very different killing dynamics (e.g., in the example shown in \autoref{fig:peptide-interactions}a), where treatment effects start at different time points and persist for different durations.

Given that many observed trajectories exhibited strongly time-varying treatment effects, we summarised each trajectory by the time-weighted net growth rate \(\psi\) (Methods), defined as a time-weighted average of the instantaneous net growth rate \(\hat{\psi}(t)\) (\autoref{eq:psi_weighted}).
This way, we emphasise early treatment effects, as changes in net growth that occur earlier affect the trajectory longer than equally sized changes occurring later.
Equivalently, \(\psi\) is proportional to the area under the log-normalised luminescence trajectory \(Y(t)=\ln(I(t)/I(0))\) (\autoref{eq:Ydef}).
Treatment effects are then defined as the difference between the time-weighted net growth rates of an untreated control and the treated condition, and correspond to the scaled area between the curves (\autoref{fig:peptide-interactions}b).

Median time-weighted net growth rates (\(n=4\)) for all conditions are shown in \autoref{fig:psi_chl_tet} and \autoref{fig:psi}, with markers indicating the classification as inhibitory (stars), sub-inhibitory (diamonds), or intermediate (no marker). For further details see Methods.
We then fitted single-drug pharmacodynamic curves to the monotreatment estimates (\autoref{fig:single-pdcurves}).

For each condition $(c_A, c_B)$ we estimate the bootstrap distribution \(\bm{\tau}(c_A,c_B)\) by resampling pairs of control and treatment wells with replacement (Methods, \siref{ssec:bootstrap\_interactions}).
Throughout this manuscript, we denote distributions with bold symbols.
For each bootstrap draw \(b\), we compute \(\tau_b(c_A,c_B)=\psi_b(\emptyset)-\psi_b(c_A,c_B)\).

\begin{figure}
  \centering
  \includegraphics[width = 89 mm]{chapter_4/figures/psi_chloramphenicol_tetracycline.pdf}
  \caption{
    Time-weighted growth rate \(\psi\) for combinations of CHL and TET.
    Each cell corresponds to one concentration pair, with colour indicating the median of $\psi$ across \(n=4\) biological replicates.
    Markers denote whether the distribution of $\psi$ is significantly different from zero (see Methods): stars indicate significantly positive values (net growth), and diamonds indicate significantly negative values (net killing).
  }
  \label{fig:psi_chl_tet}
\end{figure}

%%%%%%%%
%% Interaction Models
%%%%%%%%%
\paragraph{Bliss independence implies additive treatment effects.}
Because the original Bliss formulation is probabilistic, we translate it into our growth-based treatment-effect framework (see \siref{ssec:bliss}).
Bliss independence implies multiplicative survival  \(S_{AB}(T)=S_A(T)\,S_B(T)\) (\autoref{eq:bliss_prob}).
After substituting \(S_i(T)=\exp\!\big(-\int_0^T \hat{\tau}_i(u)\,du\big)\) with \(\hat{\tau}_i(t)\), denoting the \emph{instantaneous treatment effect}, we can derive that Bliss implies additive \emph{time-weighted treatment effects}:
\begin{equation}
  \tau^{\mathrm{Bliss}}(c_A,c_B)=\tau(c_A)+\tau(c_B).
  \label{eq:tau_bliss_main}
\end{equation}

% Not A Toy model.
\paragraph{Peptide -- non-peptide interaction model.}
We observed that a combination of non-peptide drugs with relatively constant inhibition dynamics and peptide-like drugs, which induce a sharp early decline followed by (almost) unimpaired growth, is dominated by the peptide drug in the early phase and by the non-peptide drug in the later phase.
A well-behaved example of such a combination is shown in \autoref{fig:peptide-interactions}a for polymyxin~B (POL) and tetracycline (TET).
Because most interaction metrics do not explicitly account for time-varying effects, it is unclear what combined treatment effects to expect.
We, therefore, constructed a simple model for an idealised scenario.
Drug~A immediately reduces the population by a factor \(\alpha\) but does not affect the subsequent dynamics, while drug~B acts invariantly over time by reducing the constant net growth rate from \(\hat{\psi}_{\mathrm{ctrl}}\) to \(\hat{\psi}_{B}\).
In combination, the population is first reduced by \(\alpha\) and then grows with \(\hat{\psi}_{B}\).
Substituting our rate definition (\autoref{eq:effective_psi}) into this model yields additive treatment effects, \(\tau_{AB}=\tau_A+\tau_B\) (see \siref{ssec:toy_model}), equivalent to the Bliss prediction derived above.

\paragraph{Interaction scores $(\mu, \nu)$.}
Based on the Bliss prediction  of combined treatment effects derived above, we define the Bliss interaction score \(\mu\) as the normalised deviation from the prediction (\autoref{eq:mu}).
The distribution of \(\mu\) can be inferred based on the distributions of treatment effects $\bm{\tau}$ described above.
For Loewe additivity, we define the interaction score \(\nu\) as the deviation from dose equivalence (\autoref{eq:fici}).
We obtain \(\nu_b\) from the inverted single-drug pharmacodynamic curves \(f_i^{-1}(\psi_b)\) (\autoref{eq:psi_inverse_closed_form}).
We then assign an interaction type (synergistic, neutral, antagonistic) depending on whether the \(95\%\) interval of \(\bm{\mu}\) or \(\bm{\nu}\) lies below, contains, or lies above zero.
The resulting interaction types based on  \(\bm{\mu}\) and \(\bm{\nu}\) for each combination are shown in Figures \ref{fig:mu_interaction_heatmap} and \ref{fig:fici_interaction_heatmap}.

%%%%%%%%
%% Bootstrap treatment effects
%%%%%%%%%
\paragraph{Disagreement between sub-inhibitory and inhibitory  interaction types within reference models.}
Using the per-condition distributions of interaction scores  described above, we next assessed whether interaction classifications remain consistent across concentration regimes.
We normalised concentrations \(c_i\) by the corresponding \(\mathrm{zMIC}_i\) (defined by \(f_i(c_i) =0\); \autoref{tab:pd_curve_parameter}), setting \(z_i=c_i/\mathrm{zMIC}_i\).
We then separately summarised inhibitory and sub-inhibitory conditions using a second, higher-level bootstrap.
Specifically, we sampled conditions \((c_A,c_B)_r\) for \(r=(1,\dots,200)\) with replacement, weighted by their mixing ratio (\autoref{eq:weights}).
For each sampled condition, we drew one estimate from the condition's distributions \(\bm{\mu}(c_A,c_B)\) and \(\bm{\nu}(c_A,c_B)\).
To compare regimes and reference models, we define three alignment classes: agreement (same classification), soft disagreement (neutral in one but synergistic or antagonistic in the other), and strong disagreement (opposite classifications).
The resulting comparisons between sub-inhibitory and inhibitory regimes for \(\bm{\nu}_{\mathrm{sub}}\), \(\bm{\nu}_{\mathrm{inh}}\) and \(\bm{\mu}_{\mathrm{sub}}\), \(\bm{\mu}_{\mathrm{inh}}\) are shown in \autoref{fig:interation-distributions}a,b and the resulting interaction types in \autoref{tab:intearction_summary}.

Under Bliss independence, we observed six combinations with agreement and 9 cases of disagreement, of which one resulted in strong disagreement (AMO+PEN) and eight resulted in soft disagreement.
Under Loewe additivity, we also observed six combinations with agreement and 9 cases of disagreement.
Here, we observed two cases of strong disagreement (COL+TET and POL+TET) and seven with soft disagreement.

\begin{figure*}
  \centering
  \begin{overpic}[width=189mm]{chapter_4/figures/distributions_triptych.pdf}
  \put(5,44){\large\textbf{a)}}
\put(38,44){\large\textbf{b)}}
\put(70,44){\large\textbf{c)}}
\end{overpic}

\caption{
Two-dimensional summaries of interaction estimates across regimes and models.
Panels \textbf{(a)} and \textbf{(b)} compare sub--inhibitory vs.\ inhibitory interaction scores, using \textbf{(a)} the Bliss interaction score $\mu$ and \textbf{(b)} the Loewe interaction score $\nu$.
Each point represents one drug combination, plotted as the median estimate in the sub--inhibitory regime (x-axis) against the median estimate in the inhibitory regime (y-axis), with 95\% bootstrap intervals shown as horizontal and vertical error bars.
\textbf{(c)} Loewe vs.\ Bliss comparison across regimes: each point corresponds to one combination and regime, plotted as the median Loewe interaction score $\nu$ (x-axis) against the corresponding Bliss interaction score $\mu$ (y-axis), with 95\% bootstrap intervals.
Markers encode the drug combination (shared legend for panels \textbf{(a)} and \textbf{(b)}), while marker shape encodes the regime in panel \textbf{(c)}.
Colors indicate classification agreement between the two compared axes in each panel:
\emph{agreement} if both classifications match (N--N, S--S, A--A),
\emph{soft disagreement} if one classification is neutral and the other is non-neutral (N--S, N--A),
and \emph{strong disagreement} if the classifications are opposite (S--A).
}
\label{fig:interation-distributions}
\end{figure*}

% BLISS VS LOEWE
\paragraph{Disagreement between reference models, within concentration regimes.}
\autoref{fig:interation-distributions}c replots the same interaction summaries described above, but now compares the Bliss interaction score \(\mu\) to the Loewe interaction score \(\nu\) across both sub-inhibitory and inhibitory regimes in a single panel.
\autoref{fig:si-loewe-vs-bliss-2d}a,b show the same comparison but  separated by regime.
Across all 30 (2\(\times\)15) comparisons, the Bliss and Loewe-based classifications agree in 14 cases, show soft disagreement in 15 cases, and show strong disagreement in one case (CHL+TET) (\autoref{fig:interation-distributions}c).

%%%%%%%%
%% SURFACE PART
%%%%%%%%%
\paragraph{Interaction types can change with dose even at fixed mixing ratio.}
Above, we compared interaction summaries between sub-inhibitory and inhibitory regimes by aggregating condition-wise estimates across a range of doses and mixing ratios.
We next ask whether interaction types also change (i) as the dose increases at a fixed mixing ratio, and (ii) as the mixing ratio varies at a fixed effect level.
To facilitate both analyses, we reparameterize concentration pairs \((c_A,c_B)\) in polar coordinates \((z,\phi)\), where \(z\) is the combined dose and \(\phi\) is the mixing angle (\autoref{eq:polar_coordinates}).

For each drug combination, we fitted 25 continuous, monotonically decreasing surface splines on bootstrap datasets of \(\psi(c_A,c_B)\) (\autoref{fig:surface_psi}).
Based on these splines, we estimate \emph{polar pharmacodynamic curves}, which show \(\psi\) as a function of the combined dose \(z\) at a fixed mixing angle \(\phi=45^\circ\) (equal mixing in units of \(\mathrm{zMIC}\)).
At a given dose, predicted values above the observed \(\psi\) indicate synergy, whereas predicted values below the observed \(\psi\) indicate antagonism.
For both reference models, there are examples where the interaction changes direction as the combined dose increases.
For the combination AMO+COL (\autoref{fig:surface_plots}a), the Loewe-based interaction shifts from antagonism at lower doses to synergy at higher doses.
For AMO+PEN (\autoref{fig:surface_plots}b) Bliss-based interaction flips, with synergy at lower doses and strong antagonism at higher doses.
\emph{Polar pharmacodynamic curves} at \(45^\circ\) for all combinations are shown in \autoref{fig:surface_pd}.

\paragraph{Interaction types can depend on the mixing ratio.}
To assess whether interaction types depend on the mixing ratio at a fixed effect level, we extracted isoboles from the median surface spline, i.e., the path \((z,\phi)\) along which the time-weighted net growth rate is constant (\(\psi=\SI{0}{\per\hour}\); \autoref{fig:surface_isoboles}).
Along each isobole, we evaluate Bliss- and Loewe-based predictions and plot these predictions as a function of \(\phi\).
For most combinations, the inferred interaction type is stable across \(\phi\), as exemplified by CHL+TET (\autoref{fig:surface_plots}c).
However, some combinations show mixing-ratio dependence, as illustrated for COL+PEN (\autoref{fig:surface_plots}d).
Plots for all combinations are shown in \autoref{fig:surface_angular_interaction}.

%% FIGURE 4
\begin{figure}
\centering
\begin{overpic}[width = 89mm]{chapter_4/figures/surface_plot.pdf}
\put(2,95){\textbf{(a)}}
\put(2,50){\textbf{(c)}}
\put(51,95){\textbf{(b)}}
\put(51,50){\textbf{(d)}}
\end{overpic}
\caption{
Panels (a,b) show \emph{polar pharmacodynamic curves} at \(\phi=45^\circ\) (corresponding to a \(1{:}1\) ratio in units of \(\mathrm{zMIC}\)) for (a) AMO+COL and (b) AMO+PEN.
The x-axis shows the combined dose \(z\), where \(z=1/\sqrt{2}\) (blue dotted line) corresponds to both single-drug doses equaling \(0.5\,\mathrm{zMIC}\), and \(z=\sqrt{2}\) (blue dashed line) corresponds to both equaling \(1\,\mathrm{zMIC}\).
Panels (c,d) show the Bliss- and Loewe-based predictions for \(\psi\) over the mixing angle \(\phi\) along the observed isobole at \(\psi=\SI{0}{\per\hour}\) for (c) COL+PEN and (d) CHL+TET.
}

\label{fig:surface_plots}
\end{figure}

\paragraph{Inoculum effects.}
We noticed a much larger-than-expected variation in pre-treatment light intensity $I_0$ in our data, which corresponds to the size of the inoculum.
To assess the impact of this variation on the results, we regressed \(\psi\) for each single-drug and concentration on the pre-treatment signal \(I_0\) (\siref{ssec:inoculum}).
We found negligible inoculum effects for AMO, CHL, PEN, and TET, but substantial effects for COL and POL at intermediate concentrations (see \autoref{fig:inoculum}).
Because the size of the inoculum did not show a significant directional trend along the concentration index (\(p=0.092\)), the variance of the inoculum mainly adds noise at intermediate concentrations of COL and POL, contributing to an increased scatter in \autoref{fig:single-pdcurves}c,e.
