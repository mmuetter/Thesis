% !TEX root = ../main.tex

\documentclass[../main.tex]{subfiles}
\newcommand{\siref}[1]{\textcolor{blue}{S\thechapter.#1}}
\newcommand{\sifig}[1]{\textcolor{blue}{Figure~S\thechapter.#1}}
\newcommand{\sitab}[1]{\textcolor{blue}{Table~S\thechapter.#1}}
\newcommand{\sieq}[1]{\textcolor{blue}{Equation~S\thechapter.#1}}
\newcommand{\sisec}[1]{\textcolor{blue}{#1}}
\newcommand{\panelref}[2]{\autoref{#1}\textcolor{blue}{#2}}
\newcommand{\siappendix}[1]{\textcolor{blue}{SI Appendix~\thechapter.#1}}

\title{Antimicrobial Combination Effects at Sub-inhibitory Doses do not Reliably Predict Effects at Inhibitory Concentrations}
\author{Malte Mütter, Daniel Angst, Roland Regoes, Sebastian Bonhoeffer}
\manuscriptabstract{
  Assessing whether drug combinations synergise or antagonise is difficult for several reasons: (i) measuring bacterial death rates at clinically relevant drug concentrations is methodologically challenging, (ii) there is no unifying definition of what constitutes synergistic or antagonistic interactions, and (iii) both synergism and antagonism may be concentration and mixing ratio dependent.
  To assess how well sub-inhibitory measurements predict inhibitory behaviour, we quantified drug interactions for \num{15} pairwise drug combinations on a concentration checkerboard covering a wide range of inhibitory and sub-inhibitory concentrations.
  To this end, we recorded \num{8640} time-resolved luminescence trajectories that have been shown to track bacterial population declines.
  To handle time-varying treatment effects and allow fair comparisons between drugs with distinct killing dynamics, we introduced a growth-integrated rate-like metric \(\psi\) to summarise each trajectory, and assigned interaction types (synergistic/independent/antagonistic) based on Bliss independence and Loewe additivity.
  We found that the interaction type frequently changes as the concentration increases from sub-inhibitory to inhibitory concentrations.
  Moreover, interaction type depended on the mixing ratio, implying that single-point sub-inhibitory measurements are not sufficient to predict interactions at clinically relevant concentrations.
}

\begin{document}
\manuscriptfront

\section{Introduction}

% AMR IS BAD.
Antimicrobial resistance (AMR) poses a major global health burden, with an estimated 1.14 million deaths directly attributable to resistant infections in 2021 alone \cite{Naghavi2024}.
Moreover, the lag between the introduction of a new antibiotic class and the emergence of clinical resistance appears to be shrinking \cite{Witzany2020}.
To develop a more sustainable approach to antibiotic use, a deeper understanding of the influence of treatment on the evolution of resistance is necessary.

% COMBINATION IS COOL
One strategy that has been shown to slow the evolution of resistance in notoriously fast-evolving pathogens, such as HIV, malaria, and tuberculosis, is combination therapy \cite{Gulick1997, Goldberg2012}.
In theoretical models, combination therapy seems to outperform other strategies (such as mixing or cycling) \cite{Bonhoeffer1997, Tepekule2017, Uecker2021}, and it also seems to perform best in in vitro simulations of epidemics \cite{Angst2021, Muetter2024}. 
However, clinical evidence for the efficacy of combination therapy is inconclusive, and a comprehensive meta-analysis did not identify a significant overall benefit \cite{Siedentop2024}.

% DRUG INTERACTIONS
Several factors contribute to the inconclusiveness of clinical studies, ranging from ethical constraints that limit focal pathogens to comparatively benign ones (i.e.\ excluding HIV and \emph{M.\ tuberculosis}) to the lack of statistical power to track resistance evolution \cite{Siedentop2024}.
Another potential reason is drug interactions. 
Drug combinations can exhibit synergistic or antagonistic interactions, depending on whether the combined effect is stronger or weaker than the expectation based on the effects of the single drugs. 
However, defining this expectation is non-trivial, as null models are numerous and the best choice to describe drug independence remains debated \cite{Foucquier2015, Duarte2022}.  
Here, we use two classical and widely applied reference frameworks to define independence: the response-based Bliss independence \cite{Bliss1939} and dose-based Loewe additivity \cite{Loewe1926}. 


% BLISS 
Bliss independence is based on the assumption that the probability of being affected by one drug is statistically independent of being affected by the other \cite{Bliss1939}:
\begin{equation}
    p_{AB}^\mathrm{Bliss}=p_A+p_B-p_A\,p_B .    
\end{equation}
Under this assumption, Bliss independence is expected to be more appropriate for drug pairs with distinct and non-overlapping mechanisms of action \cite{Baeder2016}.


% LOEWE
In contrast, Loewe additivity -- first discussed as iso-additivity by Frei in 1913 \cite{Frei1913} and later formalised by Loewe and Muischnek \cite{Loewe1926} -- uses dose equivalence as a null model.
A combination is considered additive if pairs of concentrations that produce the same effect $\psi$ (i.e.\ a measure of net growth) lie on a linear isobole connecting the equivalent single-drug doses \cite{Loewe1926}:
\begin{equation}
\frac{c_A}{f_A^{-1}\!\big(\psi(c_A, c_B)\big)}
+
\frac{c_B}{f_B^{-1}\!\big(\psi(c_A, c_B)\big)}
=
1. 
\label{eq:loewe_core}
\end{equation}
Here $f_i^{-1}$  represents the inverse of a pharmacodynamic function  $f_i(c)$ that maps drug concentrations $c_i$ to an effect measure $\psi$. 


%% PD STUFF
There are many studies on drug interactions below the minimum inhibitory concentration (MIC) (e.g.\ \cite{Yeh2006, OShaughnessy2006, Chevereau2015, Russ2018, Katzir2019, Kavcic2020, Rezzoagli2020, Sanchez-Hevia2025, huntelman2023}) and at the MIC level (e.g.\ \cite{Yang2017, Ruden2019, Liu2022}).
However, studies covering the arguably clinically more relevant inhibitory regime are scarce and often cover only a few points in the inhibitory range (e.g. \cite{Ocampo2014, Rao2017, Yu2016, Caballero2018, Chen2020}).
The primary reason for this is methodological: under sub-inhibitory conditions, growth rates can be easily assessed at high throughput using optical density measurements, whereas inhibitory conditions typically require labour-intensive, low-throughput colony-forming unit assays \cite{Bognar2024}. 
This lack of data on effects at high concentrations is a problem, not only because inhibitory conditions are more relevant therapeutically, but also because the right, inhibitory, tail of the pharmacodynamic curve has a strong influence on resistance evolution \cite{Yu2018}.

%WHAT we do
It has already been shown that drug interactions can depend on dose and mixing ratio \cite{Chou1984,OShaughnessy2006,Meletiadis2007,Rezzoagli2020,Kavcic2021}.
However, the existing literature provides limited quantitative insight into how frequently interaction classifications remain stable versus how often they change with concentration and mixing ratio.
Here, we provide an extensive map of interactions for \num{15} antibiotic pairs, each measured on a \(12\times 12\) concentration checkerboard that includes inhibitory concentrations using time-resolved bioluminescence trajectories.
We quantify how often sub-inhibitory interaction classifications based on Bliss independence and Loewe additivity generalise to inhibitory conditions and how often they fail, including cases of qualitative reversals across regimes.




\section{Results}
% Results introduction
We quantified drug interactions for 15 pairwise combinations of six antibiotics spanning three mechanistic classes: polymyxins (colistin, COL; polymyxin~B, POL), \(\beta\)-lactams (amoxicillin, AMO; penicillin, PEN), and ribosome-targeting inhibitors (chloramphenicol, CHL; tetracycline, TET).
For each drug pair, we assessed 144 conditions in a \(12\times 12\) concentration checkerboard.
We used light intensity trajectories  $I(t)$ measured for 8{,}640 bioluminescent cultures to infer changes in population size over time.
Monotreatment trajectories are shown in \autoref{fig:single-timecourses}.

\begin{figure}
  \centering
  \begin{overpic}[width=89mm]{chapter_4/figures/peptide_interaction.pdf}
    \put(5,97){\small\textbf{(a)}}
    \put(5,66){\small\textbf{(b)}}
    \put(5,36){\small\textbf{(c)}}
    \put(45,50){\scriptsize$\mathrm{AUC}_\mathrm{ctrl} = \displaystyle \int_{0}^{T} Y_{\mathrm{ctrl}}(t)\,\mathrm{d}t$}
    \put(23,43){\scriptsize$\mathrm{AUC}_\mathrm{treat} = \displaystyle \int_{0}^{T} Y_{\mathrm{treat}}(t)\,\mathrm{d}t$}
  \end{overpic}
  \caption{
    Panel (a) shows the median normalised light-intensity trajectories \(y(t)=I(t)/I(0)\) for the combination of polymyxin~B (POL, \SI{0.5}{\micro\gram\per\milli\liter}) and tetracycline (TET, \SI{2}{\micro\gram\per\milli\liter}), together with the corresponding monotherapies and the untreated control, with shaded bands indicating the interquartile range across biological replicates.
    Panel (b) shows the log-transformed signal \(Y(t)=\ln(y(t))\) for the untreated control and the combination treatment from panel (a), together with their associated areas under the curve.
    Panel (c) shows a simple peptide--antibiotic interaction model illustrating qualitatively distinct dynamics under: no drugs, antibiotic (slow constant decline), peptide (rapid initial decline), and peptide plus antibiotic (rapid initial decline followed by a constant slow decline).
  }
  \label{fig:peptide-interactions}
\end{figure}

%%%%%%%%
%%% PEPTIDE - Antibitic Interactions
%% Effective net growth
%%%%%%%%%
\paragraph{Time-variant growth rates and treatment effects.}
Most drug-interaction metrics are based on estimates of \textit{treatment effects}, which summarise growth trajectories as a scalar.
These estimates are typically either rate-based, obtained by fitting an exponential rate of change to a growth trajectory \cite{Regoes2004, Yeh2006, Chevereau2015a, Yu2016, Angermayr2022}, or area under the curve (AUC)-based, obtained by integrating a (transformed) growth trajectory over time \cite{MacGowan2000, Lenhard2015, Rao2017, Caballero2018, Chen2020, Sanchez-Hevia2025}.
Rate-based approaches are easy to interpret and can be used directly in epidemiological models.
However, under time-variant treatment effects, inferred slopes are sensitive to the choice of the fitting window.
This sensitivity is especially problematic when assessing interactions among drugs with very different killing dynamics (e.g., in the example shown in \autoref{fig:peptide-interactions}a), where treatment effects start at different time points and persist for different durations.

Given that many observed trajectories exhibited strongly time-varying treatment effects, we summarised each trajectory by the time-weighted net growth rate \(\psi\) (Methods), defined as a time-weighted average of the instantaneous net growth rate \(\hat{\psi}(t)\) (\autoref{eq:psi_weighted}).
This way, we emphasise early treatment effects, as changes in net growth that occur earlier affect the trajectory longer than equally sized changes occurring later.
Equivalently, \(\psi\) is proportional to the area under the log-normalised luminescence trajectory \(Y(t)=\ln(I(t)/I(0))\) (\autoref{eq:Ydef}).
Treatment effects are then defined as the difference between the time-weighted net growth rates of an untreated control and the treated condition, and correspond to the scaled area between the curves (\autoref{fig:peptide-interactions}b).

Median time-weighted net growth rates (\(n=4\)) for all conditions are shown in \autoref{fig:psi_chl_tet} and \autoref{fig:psi}, with markers indicating the classification as inhibitory (stars), sub-inhibitory (diamonds), or intermediate (no marker). For further details see Methods.
We then fitted single-drug pharmacodynamic curves to the monotreatment estimates (\autoref{fig:single-pdcurves}).

For each condition $(c_A, c_B)$ we estimate the bootstrap distribution \(\bm{\tau}(c_A,c_B)\) by resampling pairs of control and treatment wells with replacement (Methods, \siref{ssec:bootstrap\_interactions}).
Throughout this manuscript, we denote distributions with bold symbols.
For each bootstrap draw \(b\), we compute \(\tau_b(c_A,c_B)=\psi_b(\emptyset)-\psi_b(c_A,c_B)\).

\begin{figure}
  \centering
  \includegraphics[width = 89 mm]{chapter_4/figures/psi_chloramphenicol_tetracycline.pdf}
  \caption{
    Time-weighted growth rate \(\psi\) for combinations of CHL and TET.
    Each cell corresponds to one concentration pair, with colour indicating the median of $\psi$ across \(n=4\) biological replicates.
    Markers denote whether the distribution of $\psi$ is significantly different from zero (see Methods): stars indicate significantly positive values (net growth), and diamonds indicate significantly negative values (net killing).
  }
  \label{fig:psi_chl_tet}
\end{figure}

%%%%%%%%
%% Interaction Models
%%%%%%%%%
\paragraph{Bliss independence implies additive treatment effects.}
Because the original Bliss formulation is probabilistic, we translate it into our growth-based treatment-effect framework (see \siref{ssec:bliss}).
Bliss independence implies multiplicative survival  \(S_{AB}(T)=S_A(T)\,S_B(T)\) (\autoref{eq:bliss_prob}).
After substituting \(S_i(T)=\exp\!\big(-\int_0^T \hat{\tau}_i(u)\,du\big)\) with \(\hat{\tau}_i(t)\), denoting the \emph{instantaneous treatment effect}, we can derive that Bliss implies additive \emph{time-weighted treatment effects}:
\begin{equation}
  \tau^{\mathrm{Bliss}}(c_A,c_B)=\tau(c_A)+\tau(c_B).
  \label{eq:tau_bliss_main}
\end{equation}

% Not A Toy model.
\paragraph{Peptide -- non-peptide interaction model.}
We observed that a combination of non-peptide drugs with relatively constant inhibition dynamics and peptide-like drugs, which induce a sharp early decline followed by (almost) unimpaired growth, is dominated by the peptide drug in the early phase and by the non-peptide drug in the later phase.
A well-behaved example of such a combination is shown in \autoref{fig:peptide-interactions}a for polymyxin~B (POL) and tetracycline (TET).
Because most interaction metrics do not explicitly account for time-varying effects, it is unclear what combined treatment effects to expect.
We, therefore, constructed a simple model for an idealised scenario.
Drug~A immediately reduces the population by a factor \(\alpha\) but does not affect the subsequent dynamics, while drug~B acts invariantly over time by reducing the constant net growth rate from \(\hat{\psi}_{\mathrm{ctrl}}\) to \(\hat{\psi}_{B}\).
In combination, the population is first reduced by \(\alpha\) and then grows with \(\hat{\psi}_{B}\).
Substituting our rate definition (\autoref{eq:effective_psi}) into this model yields additive treatment effects, \(\tau_{AB}=\tau_A+\tau_B\) (see \siref{ssec:toy_model}), equivalent to the Bliss prediction derived above.

\paragraph{Interaction scores $(\mu, \nu)$.}
Based on the Bliss prediction  of combined treatment effects derived above, we define the Bliss interaction score \(\mu\) as the normalised deviation from the prediction (\autoref{eq:mu}).
The distribution of \(\mu\) can be inferred based on the distributions of treatment effects $\bm{\tau}$ described above.
For Loewe additivity, we define the interaction score \(\nu\) as the deviation from dose equivalence (\autoref{eq:fici}).
We obtain \(\nu_b\) from the inverted single-drug pharmacodynamic curves \(f_i^{-1}(\psi_b)\) (\autoref{eq:psi_inverse_closed_form}).
We then assign an interaction type (synergistic, neutral, antagonistic) depending on whether the \(95\%\) interval of \(\bm{\mu}\) or \(\bm{\nu}\) lies below, contains, or lies above zero.
The resulting interaction types based on  \(\bm{\mu}\) and \(\bm{\nu}\) for each combination are shown in Figures \ref{fig:mu_interaction_heatmap} and \ref{fig:fici_interaction_heatmap}.

%%%%%%%%
%% Bootstrap treatment effects
%%%%%%%%%
\paragraph{Disagreement between sub-inhibitory and inhibitory  interaction types within reference models.}
Using the per-condition distributions of interaction scores  described above, we next assessed whether interaction classifications remain consistent across concentration regimes.
We normalised concentrations \(c_i\) by the corresponding \(\mathrm{zMIC}_i\) (defined by \(f_i(c_i) =0\); \autoref{tab:pd_curve_parameter}), setting \(z_i=c_i/\mathrm{zMIC}_i\).
We then separately summarised inhibitory and sub-inhibitory conditions using a second, higher-level bootstrap.
Specifically, we sampled conditions \((c_A,c_B)_r\) for \(r=(1,\dots,200)\) with replacement, weighted by their mixing ratio (\autoref{eq:weights}).
For each sampled condition, we drew one estimate from the condition's distributions \(\bm{\mu}(c_A,c_B)\) and \(\bm{\nu}(c_A,c_B)\).
To compare regimes and reference models, we define three alignment classes: agreement (same classification), soft disagreement (neutral in one but synergistic or antagonistic in the other), and strong disagreement (opposite classifications).
The resulting comparisons between sub-inhibitory and inhibitory regimes for \(\bm{\nu}_{\mathrm{sub}}\), \(\bm{\nu}_{\mathrm{inh}}\) and \(\bm{\mu}_{\mathrm{sub}}\), \(\bm{\mu}_{\mathrm{inh}}\) are shown in \autoref{fig:interation-distributions}a,b and the resulting interaction types in \autoref{tab:intearction_summary}.

Under Bliss independence, we observed six combinations with agreement and 9 cases of disagreement, of which one resulted in strong disagreement (AMO+PEN) and eight resulted in soft disagreement.
Under Loewe additivity, we also observed six combinations with agreement and 9 cases of disagreement.
Here, we observed two cases of strong disagreement (COL+TET and POL+TET) and seven with soft disagreement.

\begin{figure*}
  \centering
  \begin{overpic}[width=189mm]{chapter_4/figures/distributions_triptych.pdf}
  \put(5,44){\large\textbf{a)}}
\put(38,44){\large\textbf{b)}}
\put(70,44){\large\textbf{c)}}
\end{overpic}

\caption{
Two-dimensional summaries of interaction estimates across regimes and models.
Panels \textbf{(a)} and \textbf{(b)} compare sub--inhibitory vs.\ inhibitory interaction scores, using \textbf{(a)} the Bliss interaction score $\mu$ and \textbf{(b)} the Loewe interaction score $\nu$.
Each point represents one drug combination, plotted as the median estimate in the sub--inhibitory regime (x-axis) against the median estimate in the inhibitory regime (y-axis), with 95\% bootstrap intervals shown as horizontal and vertical error bars.
\textbf{(c)} Loewe vs.\ Bliss comparison across regimes: each point corresponds to one combination and regime, plotted as the median Loewe interaction score $\nu$ (x-axis) against the corresponding Bliss interaction score $\mu$ (y-axis), with 95\% bootstrap intervals.
Markers encode the drug combination (shared legend for panels \textbf{(a)} and \textbf{(b)}), while marker shape encodes the regime in panel \textbf{(c)}.
Colors indicate classification agreement between the two compared axes in each panel:
\emph{agreement} if both classifications match (N--N, S--S, A--A),
\emph{soft disagreement} if one classification is neutral and the other is non-neutral (N--S, N--A),
and \emph{strong disagreement} if the classifications are opposite (S--A).
}
\label{fig:interation-distributions}
\end{figure*}

% BLISS VS LOEWE
\paragraph{Disagreement between reference models, within concentration regimes.}
\autoref{fig:interation-distributions}c replots the same interaction summaries described above, but now compares the Bliss interaction score \(\mu\) to the Loewe interaction score \(\nu\) across both sub-inhibitory and inhibitory regimes in a single panel.
\autoref{fig:si-loewe-vs-bliss-2d}a,b show the same comparison but  separated by regime.
Across all 30 (2\(\times\)15) comparisons, the Bliss and Loewe-based classifications agree in 14 cases, show soft disagreement in 15 cases, and show strong disagreement in one case (CHL+TET) (\autoref{fig:interation-distributions}c).

%%%%%%%%
%% SURFACE PART
%%%%%%%%%
\paragraph{Interaction types can change with dose even at fixed mixing ratio.}
Above, we compared interaction summaries between sub-inhibitory and inhibitory regimes by aggregating condition-wise estimates across a range of doses and mixing ratios.
We next ask whether interaction types also change (i) as the dose increases at a fixed mixing ratio, and (ii) as the mixing ratio varies at a fixed effect level.
To facilitate both analyses, we reparameterize concentration pairs \((c_A,c_B)\) in polar coordinates \((z,\phi)\), where \(z\) is the combined dose and \(\phi\) is the mixing angle (\autoref{eq:polar_coordinates}).

For each drug combination, we fitted 25 continuous, monotonically decreasing surface splines on bootstrap datasets of \(\psi(c_A,c_B)\) (\autoref{fig:surface_psi}).
Based on these splines, we estimate \emph{polar pharmacodynamic curves}, which show \(\psi\) as a function of the combined dose \(z\) at a fixed mixing angle \(\phi=45^\circ\) (equal mixing in units of \(\mathrm{zMIC}\)).
At a given dose, predicted values above the observed \(\psi\) indicate synergy, whereas predicted values below the observed \(\psi\) indicate antagonism.
For both reference models, there are examples where the interaction changes direction as the combined dose increases.
For the combination AMO+COL (\autoref{fig:surface_plots}a), the Loewe-based interaction shifts from antagonism at lower doses to synergy at higher doses.
For AMO+PEN (\autoref{fig:surface_plots}b) Bliss-based interaction flips, with synergy at lower doses and strong antagonism at higher doses.
\emph{Polar pharmacodynamic curves} at \(45^\circ\) for all combinations are shown in \autoref{fig:surface_pd}.

\paragraph{Interaction types can depend on the mixing ratio.}
To assess whether interaction types depend on the mixing ratio at a fixed effect level, we extracted isoboles from the median surface spline, i.e., the path \((z,\phi)\) along which the time-weighted net growth rate is constant (\(\psi=\SI{0}{\per\hour}\); \autoref{fig:surface_isoboles}).
Along each isobole, we evaluate Bliss- and Loewe-based predictions and plot these predictions as a function of \(\phi\).
For most combinations, the inferred interaction type is stable across \(\phi\), as exemplified by CHL+TET (\autoref{fig:surface_plots}c).
However, some combinations show mixing-ratio dependence, as illustrated for COL+PEN (\autoref{fig:surface_plots}d).
Plots for all combinations are shown in \autoref{fig:surface_angular_interaction}.

%% FIGURE 4
\begin{figure}
\centering
\begin{overpic}[width = 89mm]{chapter_4/figures/surface_plot.pdf}
\put(2,95){\textbf{(a)}}
\put(2,50){\textbf{(c)}}
\put(51,95){\textbf{(b)}}
\put(51,50){\textbf{(d)}}
\end{overpic}
\caption{
Panels (a,b) show \emph{polar pharmacodynamic curves} at \(\phi=45^\circ\) (corresponding to a \(1{:}1\) ratio in units of \(\mathrm{zMIC}\)) for (a) AMO+COL and (b) AMO+PEN.
The x-axis shows the combined dose \(z\), where \(z=1/\sqrt{2}\) (blue dotted line) corresponds to both single-drug doses equaling \(0.5\,\mathrm{zMIC}\), and \(z=\sqrt{2}\) (blue dashed line) corresponds to both equaling \(1\,\mathrm{zMIC}\).
Panels (c,d) show the Bliss- and Loewe-based predictions for \(\psi\) over the mixing angle \(\phi\) along the observed isobole at \(\psi=\SI{0}{\per\hour}\) for (c) COL+PEN and (d) CHL+TET.
}

\label{fig:surface_plots}
\end{figure}

\paragraph{Inoculum effects.}
We noticed a much larger-than-expected variation in pre-treatment light intensity $I_0$ in our data, which corresponds to the size of the inoculum.
To assess the impact of this variation on the results, we regressed \(\psi\) for each single-drug and concentration on the pre-treatment signal \(I_0\) (\siref{ssec:inoculum}).
We found negligible inoculum effects for AMO, CHL, PEN, and TET, but substantial effects for COL and POL at intermediate concentrations (see \autoref{fig:inoculum}).
Because the size of the inoculum did not show a significant directional trend along the concentration index (\(p=0.092\)), the variance of the inoculum mainly adds noise at intermediate concentrations of COL and POL, contributing to an increased scatter in \autoref{fig:single-pdcurves}c,e.


\section{Discussion}
We quantified interaction patterns across a wide concentration range for several drug combinations with different modes of action. 
Quantifying the size of the bacterial population is notoriously difficult because bacterial death is not well defined, has multiple aspects, and no single method captures all of them \cite{Wu2024}. 
In this work, we used bioluminescence as a proxy for population size. 
This choice enabled us to record \num{8640} finely time-resolved (every 10 minutes for five hours) growth trajectories at high throughput. 
The approach assumes that the mean per-cell luminosity remains approximately constant over time, which may not always hold. 
Consequently, we restricted our analysis to drugs for which luminescence was previously shown to track population dynamics reasonably well \cite{Muetter2025}, hence, we do not expect qualitatively different conclusions when using an alternative readout. 

For many of these trajectories, particularly those treated with a polymyxin, the light intensity dropped below the detection limit within the first few minutes.
In principle, slope-based estimates can still be inferred for such curves. 
However, comparing these estimates to trajectories with weaker but constant treatment effects is conceptually hard to justify. 
To enable fair comparisons across drugs with different onset times and killing dynamics, we used a weighted growth measure integrated over a longer timeframe, which is equal for all trajectories (\(T \approx \SI{2}{\hour}\)).
However, this choice comes at the cost of having to discard all trajectories that fall below the detection limit within that timeframe. 

To investigate how these different treatment dynamics combine, we developed a simplified antibiotic--peptide interaction model.
This model predicts that combinations of a short-acting, peptide-like drug with a time-invariant drug should follow Bliss independence, while drugs with similar mechanisms should follow Loewe additivity.
We observed these behaviours qualitatively across combinations to varying degrees in \autoref{fig:surface_isoboles} and \autoref{fig:surface_pd}.

To address our core question—how predictive sub-MIC interaction patterns are for inhibitory interactions—we aggregated interaction estimates for sub-inhibitory and inhibitory conditions and compared them.
For both reference models, we observed that more than half of the combinations showed soft disagreement between inhibitory and sub-inhibitory regimes.
This does not necessarily imply that sub-inhibitory measures are uninformative, because soft disagreement involves a neutral classification in one regime, and neutrality can arise from variance. 
This variance is partly a consequence of aggregating across a diverse set of conditions covering a wide range of mixing ratios, which we observed can influence the interaction type (\autoref{fig:surface_angular_interaction}).
Importantly, for both reference models, we observed more cases of synergistic or antagonistic agreement than strong disagreement, indicating that sub-inhibitory interaction measures retain some qualitative predictive value.

Our results also confirm the practical limitations of Loewe-based interaction measures at high concentrations that have been reported previously (\cite{Meyer2019}). 
Since Loewe relies on the inverse of the pharmacodynamic function of the single drugs, it is only defined when the combination effect lies within the dynamic ranges of both monotreatment effects. 
For drug pairs with strongly different maximal killing capacities, this condition fails in large parts of the checkerboard (see undefined regions in \autoref{fig:fici_interaction_heatmap}). 
This is a severe limitation that prohibits the quantification of drug interactions for a large number of therapeutically relevant conditions.
Consistent with previous work (e.g. \cite{Vlot2019}), we found that the Bliss and Loewe frameworks can produce contrasting classifications.

Our findings show that conclusions about synergy or antagonism depend on the concentration range, mixing ratio, and the chosen interaction model. 
Accordingly, single-point measurements at a single sub-inhibitory concentration are insufficient to reliably characterise drug interactions at clinically relevant inhibitory concentrations.

\begin{comment}
For a subset of combinations, we found unambiguous interaction classifications across conditions and models (see \autoref{tab:intearction_summary}). 
Specifically, we consistently observed antagonism (AAAA) for COL--PEN, AMO--CHL, AMO--POL, and CHL--PEN, with PEN--TET and AMO--COL showing similarly strong trends (three times antagonism and one neutral classification). 
The observation that pairing \(\beta\)-lactams with ribosome-targeting drugs is antagonistic is well supported in the literature (e.g. by \cite{Jawetz1957, Ocampo2014}), whereas reports for penicillins combined with polymyxins are comparatively scarce and heterogeneous. 
In contrast, we consistently observed synergy (SSSS) for CHL--POL, while CHL--COL also tended towards synergy but with more mixed classifications (ASNS). 
For polymyxins, synergy with chloramphenicol has also been reported in the colistin-combination literature.
\end{comment}

\section{Methods}
%%% WETLAB STUFF
\FloatBarrier
\paragraph{Strains and Media.}
We used the bioluminescent strain \textit{Escherichia coli} MG1655 \(\Delta\)galK::(kan\(^R\)-luxCDABE) constructed previously \cite{Muetter2025}.
Cultures were grown in LB medium (Sigma L3022). 

\paragraph{Drug preparation.}
In the experiments, we explored the interactions among six drugs (\autoref{tab:drugs}), resulting in 15 drug-drug pairs.
For each pair, one compound was designated drug~A and the other drug~B. 
Each compound was prepared as a stock (\num{20}\,x) at the highest planned working concentration. 
We then performed a twofold dilution series in a 12-column deepwell plate, resulting in 11 diluted concentrations and one drug-free column. 
For drug~A, we transferred \SI{125}{\micro\liter} per well into a 96-well plate with a horizontal concentration gradient; this is referred to as subreservoir~A. 
For drug~B, we transferred \SI{50}{\micro\liter} per well into two 96-well plates with vertical concentration gradients (each 6x12 layout); these are the antibiotic-reservoir plates (I and II). 
All plates were stored at \SI{-80}{\celsius} to minimise degradation over time while accepting a one-time degradation due to freezing and thawing.


%% DRUG TABLE
\begin{table}
    \caption{Drugs used in the experiments, the solvent, the estimated zMIC (obtained by fitting the single-drug pharmacodynamic curves), and the supplier with catalogue number.}
    \centering
    \small
    \begin{tabular}{lccp{1.8cm}}
\toprule
\small
drug & solvent &  zMIC [\si{\micro\gram\per\milli\liter}]&  supplier \\
\midrule
amoxicillin (AMO) & water & 8.42 & ThermoFischer, J61290.14 \\
chloramphenicol (CHL) & DMSO & 54.11 & Sigma, C0378 \\
colistin (COL) & water & 0.58 &Sigma, C4461 \\
penicillin (PEN) & water & 153.86 & Roth, HP48.2 \\
polymyxin B (POL) & water & 0.51 & Roth, 0235.1 \\
tetracycline (TET) & DMSO & 15.15 &Sigma, T3383 \\
\bottomrule
\end{tabular}


\begin{comment}
\begin{tabular}{lcccp{1.8cm}}
\toprule
\small
drug & solvent & $c_{1}$ [\si{\micro\gram\per\milli\liter}] & zMIC[\si{\micro\gram\per\milli\liter}]&  supplier \\
\midrule
amoxicillin (AMO) & water & 128 & zMIC & ThermoFischer, J61290.14 \\
chloramphenicol (CHL) & DMSO & 512 && Sigma, C0378 \\
colistin (COL) & water & 128 & &Sigma, C4461 \\
penicillin (PEN) & water & 2048 && Roth, HP48.2 \\
polymyxin B (POL) & water & 128 && Roth, 0235.1 \\
tetracycline (TET) & DMSO & 64 & &Sigma, T3383 \\
\bottomrule
\end{tabular}
\end{comment}
    \label{tab:drugs}
\end{table}


\paragraph{Dose response assays.}
For each assay, we prepared four overnight cultures grown for \SI{14}{\hour}. 
Cultures were distributed to fill an entire 384-well source plate (Greiner 781073), so that the replicates formed a \(2\times2\) block format (e.g. C1--rep1, C2--rep2, D1--rep3, D2--rep4).
We then prefilled two white 384‑well assay plates (Greiner 781073) with \SI{54}{\micro\liter} LB per well and transferred inocula from the source plate to both assay plates (I, II)  using the Evo 200 liquid handling platform (Tecan) with a pintool (dilution approximately $1:150$). 
Plates were incubated for \SI{2}{\hour} to reach exponential phase. 
Simultaneously, subreservoir~A and both antibiotic‑reservoir plates were thawed. 
\SI{50}{\micro\liter} from each well of the subreservoir~A were transferred to the corresponding wells of both antibiotic‑reservoir plates, generating a \num{10}\,x mixture of drugs A and B. 
For both assay plates, the assay started as follows: a baseline luminescence reading was taken, \SI{6}{\micro\liter} of the \num{10}\,x drug mixture was added (defining \(t=0\) at dosing for the respective wells), and a second luminescence reading was taken.
Subsequently, we alternated between reading assay plates I and II for a total duration of \SI{5}{\hour}.
Luminescence was recorded with an Infinite~F200 plate reader (Tecan) using a \SI{250}{\milli\second} integration time.


\paragraph{Data preprocessing.}
To estimate light noise, we conducted a calibration experiment in which six source wells (E5, E12, E20, L5, L12, L20) contained stationary-phase cultures while all other wells remained empty. 
From a single luminescence read of the full plate, we constructed a distance-dependent light-noise kernel (\autoref{fig:light_distribution}) and corrected each well by subtracting the summed contributions from neighbouring wells (restricted to distances $d\le 3$, \autoref{eq:noise_corr}). 
We defined the lower limit of detection as \(I_{\mathrm{LoD}}=\SI{10}{RLU}\) and the upper limit of detection as \SI{1e6}{RLU}.
We defined the common analysis horizon \(T\) as the earliest time point at which any untreated control well exceeded the upper detection limit, yielding \(T\approx\SI{2}{\hour}\). 
For each well, we computed the fraction of observations up to \(T\) below \(I_{\mathrm{LoD}}\). 
Wells with more than $20\%$ of observations below \(I_{\mathrm{LoD}}\) were excluded from all subsequent analyses. 
For retained wells, values below \(I_{\mathrm{LoD}}\) were censored by replacing them with \(I_{\mathrm{LoD}}\). 
For numerical integration, we linearly interpolated the log-normalised signal to obtain values on the shared horizon \(t=T\) (see also \siref{ssec:lightnoise}). 

\paragraph{Time-weighted net growth rate.}
We define \(\psi\) as the linearly time-weighted net growth rate (SI: \autoref{eq:psi_weighted}) over the common time horizon \(T\).
We infer \(\psi\) by calculating the scaled area under the log-normalised light intensity \(Y(t)=\ln\!\big(I(t)/I(0)\big)\) (SI: \autoref{eq:norm}, \autoref{eq:Ydef}),
\begin{equation}
    \psi
    =
    \frac{2}{T^{2}} \int_{0}^{T}Y(t)\,\mathrm{d}t.
    \label{eq:effective_psi}
\end{equation}



\paragraph{Classification of inhibitory and sub-inhibitory conditions.}
We classify each checkerboard condition as inhibitory, sub-inhibitory, or not significantly different from zero by testing whether replicate time-weighted growth-rate estimates \(\psi(c_A,c_B)\) are consistently positive or negative.
With four replicates, the binomial probability of observing all positive (or all negative) signs with equal probability is \(0.5^4=0.0625\).
This classification is used to annotate the \(\psi\) heatmaps (\autoref{fig:psi_chl_tet}, \autoref{fig:psi}) and to define inhibitory and sub-inhibitory regime pools.


\paragraph{Single-drug pharmacodynamic curve fitting.}
For each drug, we aggregated all mono-treatment data across experiments and fitted a pharmacodynamic (PD) curve (\(f(c)\)) to the corresponding time-weighted growth-rate estimates \(\psi\).
\begin{equation}
    f(c) = \psi_{0} - \frac{(\psi_{0}-\psi_{\min})(c/z_{\mathrm{MIC}})^{\kappa}} {(c/z_{\mathrm{MIC}})^{\kappa}-(\psi_{\min}/\psi_{0})}, \quad [0,\infty)\rightarrow[\psi_{\min},\psi_{0}]
    \label{eq:single_drug_pd}
\end{equation}
Here, \(\psi_{0}\) denotes the maximum and \(\psi_{\min}\) the minimum time-weighted growth rate, corresponding to \(c=0\) and \(c\to\infty\), respectively.
\(\kappa\) denotes the Hill coefficient and \(z_{\mathrm{MIC}}\) the concentration at which \(f(c)=0\). 
The model parameters \(\psi_{0},\psi_{\min},\kappa,z_{\mathrm{MIC}}\) were jointly estimated by least-squares fitting of \autoref{eq:single_drug_pd}. 

\paragraph{Distribution of treatment effects.}
We define the treatment effect $\tau$ as the linearly time-weighted average difference between the untreated and treated temporal net growth rates over the common time horizon \(T\) (SI: \autoref{eq:intYdiff}). 
Equivalently, \(\tau\) is the scaled area between the control and treatment trajectories on the log-normalised scale (illustrated in \autoref{fig:peptide-interactions}b). 
Because \(\tau\) is a comparative measure between one untreated and one treated well, we bootstrap treated--control well pairs with replacement (\(B=200\); details in \siref{ssec:bootstrap_interactions}). 
For each condition \((c_A,c_B)\) and draw \(b\), we compute \(\tau_b(c_A,c_B)=\psi_b(\emptyset)-\psi_b(c_A,c_B)\), yielding the distribution of treatment effects $\bm{\tau}(c_A, c_B)$. 
Throughout this manuscript, distributions are denoted by bold symbols.
We classify such bootstrap distributions as significant if the central \(95\%\) bootstrap interval excludes zero, and as not significant otherwise. 



\paragraph{Bliss-based interaction score \(\mu\).}
We define the Bliss based interaction score as the normalised deviation from the Bliss-prediction for combined treatment effects (see \siref{ssec:bliss} and \autoref{eq:tau_bliss_main}):
\begin{equation}
    \mu_b(c_A,c_B)  =
    \frac{\tau_b(c_A)+\tau_b(c_B)-\tau_b(c_A,c_B)}
    {\mathrm{median}\!\bigl(\bm{\tau}(c_A)\bigr)+\mathrm{median}\!\bigl(\bm{\tau}(c_B)\bigr)}.
    \label{eq:mu}
\end{equation}
To avoid misinterpreting noise as interaction, we only evaluate \(\mu(c_A,c_B)\) for eligible conditions.
A condition is eligible if \(\bm{\tau}(c_A,c_B)\) and at least one of \(\bm{\tau}(c_A)\) or \(\bm{\tau}(c_B)\) are significant.


\paragraph{Loewe-based interaction score \(\nu\).}
We define a Loewe interaction score (details in \siref{ssec:loewe}) that quantifies deviations from dose equivalence as
\begin{equation}
    \nu_b(c_A,c_B)
    =
    \frac{c_A}{f_A^{-1}\!\big(\psi_b(c_A,c_B)\big)}
    +
    \frac{c_B}{f_B^{-1}\!\big(\psi_b(c_A,c_B)\big)}
    -1.
    \label{eq:fici}
\end{equation}
Here \(f_i\) denotes a pharmacodynamic function mapping concentration to net growth, \(f_i(c)=\psi\). 
The inverse \(f_i^{-1}(\psi)\) is the corresponding monotreatment-equivalent concentration of drug \(i\). 
Our definition of \(\nu\) is analogous to the combination index \((\mathrm{CI}-1)\) \cite{Chou1984}, but uses a different pharmacodynamic model \(f_i\) (either \autoref{eq:psi_inverse_closed_form} or inferred numerically from the monotreatment edges of the surface splines). 
To avoid misinterpreting noise as interaction, we only evaluate \(\nu(c_A,c_B)\) if \(\bm{\tau}(c_A,c_B)\) is significant and at least one of \(\bm{\tau}(c_A,0)\) or \(\bm{\tau}(0,c_B)\) is significant. 





\paragraph{Polar reparametrization.}
We normalise concentrations of drug \(i\) using the corresponding \(\mathrm{zMIC}_i\) estimates obtained from the single-drug PD fits (\autoref{tab:drugs}): \(z_i=c_i/\mathrm{zMIC}_i\).
We then define polar coordinates based on the normalised concentrations:
\begin{equation}
    z=\sqrt{z_{A}^{2}+z_{B}^{2}},
    \qquad
    \phi=\arctan2\!\bigl(z_{B},z_{A}\bigr),
    \label{eq:polar_coordinates}
\end{equation}
with \(\phi\) corresponding to the \emph{mixing angle} and \(z\) to the \emph{combined dose}. 
These coordinates were used to plot polar pharmacodynamic curves, which show the treatment effect over the combined dose \(z\) for a fixed mixing angle \(\phi\). 
We obtain these curves as one-dimensional cross-sections through the surface splines at fixed \(\phi\). 


\section*{Data, Materials, and Software Availability}
Experimental datasets and code are available at Zenodo (DOI: \href{https://doi.org/10.5281/zenodo.18374151}{10.5281/zenodo.18374151}).

\section*{Acknowledgements}
We thank ETH Zurich for funding this work.
During manuscript preparation, we used OpenAI’s ChatGPT for editorial assistance (coding, grammar, and proofreading).
\end{document}
