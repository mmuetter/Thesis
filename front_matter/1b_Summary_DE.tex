Seit der Entdeckung des Penicillins sind Antibiotika ein Grundpfeiler der modernen Medizin.
Diese Errungenschaft ist heute bedroht, da Bakterien Resistenzmechanismen gegen die meisten Antibiotikaklassen entwickelt haben.
Um den Anstieg der Resistenz zu verlangsamen, sind Änderungen in der Art, wie Antibiotika eingesetzt werden, notwendig.
In dieser Arbeit untersuchen wir die Pharmakodynamik von Antibiotikakombinationen und wie multidrug Behandlungsstrategien die Dynamik plasmidvermittelter Resistenz beeinflussen.
In zeitkritischen klinischen Notfällen wie einer Sepsis kann die Therapie nicht auf phänotypische Empfindlichkeitstests warten und stützt sich daher auf vordefinierte empirische Behandlungsstrategien.
Um zu untersuchen, wie solche Strategien die Persistenz plasmidvermittelter Resistenz und das Entstehen doppelter Resistenz beeinflussen, führten wir großskalige automatisierte \textit{in vitro}-Experimente durch, die krankenhausähnliche Transmissionsdynamiken nachbilden.
Über die meisten Szenarien hinweg war es am effektivsten, Patienten mit zwei Antibiotika gleichzeitig zu behandeln (Kombinationstherapie).

Da die Wirksamkeit der Kombinationstherapie durch Arzneimittelinteraktionen (d.\,h.\ Synergie, Antagonismus oder Unabhängigkeit) geprägt wird, die als Abweichungen von erwarteten kombinierten Effekten definiert sind.
Um Behandlungseffekte unter klinisch relevanten hemmenden Bedingungen in hohem Durchsatz erfassen zu können, prüften wir, ob biolumineszenzbasierte Lichtintensität ein geeigneter Proxy für die Zellzahldynamik ist.
Für 20 antimikrobielle Wirkstoffe verglichen wir Biolumineszenz-Zeitreihen mit Kolonie-bildenden Einheiten (CFU) und ergänzten diese Experimente durch Mikroskopie.
Wir fanden, dass Biolumineszenz die Biomassedynamik besser abbildet als die Zellzahldynamik, dass jedoch beide Maße ähnliche Dynamiken zeigen, wenn die Zellgröße näherungsweise konstant bleibt.
Andererseits beobachteten wir, dass CFU-basierte Schätzungen durch wirkstoffinduzierte Veränderungen der Kultivierbarkeit und durch Antibiotika-Übertrag (carry-over) verzerrt sein können.

Mithilfe dieser biolumineszenzbasierten Methode quantifizierten wir Antibiotikainteraktionen für 15 Wirkstoffpaare auf Checkerboards, die nicht-hemmende bis hemmende Konzentrationen abdecken.
Wir fanden, dass Interaktionstypen bei nicht-hemmenden Konzentrationen häufig von denen bei hemmenden Konzentrationen abweichen.
Zudem können Interaktionstypen vom Mischungsverhältnis abhängen und vom gewählten Referenzmodell beeinflusst werden.
Zusammen zeigen diese Ergebnisse das Potenzial der Kombinationstherapie, liefern methodische Einsichten zu ihrer Optimierung und mahnen zur Vorsicht bei der unkritischen Extrapolation von Befunden über den gemessenen Konzentrationsraum hinweg.
