%\\addcontentsline{toc}{chapter}{Zusammenfassung}
Seit der Entdeckung des Penicillins sind Antibiotika ein Grundpfeiler der modernen Medizin.
Diese Errungenschaft ist heute bedroht, da Bakterien Resistenzmechanismen gegen die meisten Antibiotikaklassen entwickelt haben.
Um den Anstieg der Resistenz zu verlangsamen, sind Änderungen in der Art, wie Antibiotika eingesetzt werden, notwendig.
In dieser Arbeit untersuchen wir die Pharmakodynamik von Antibiotikakombinationen und wie Behandlungsstrategien die Dynamik plasmidvermittelter Resistenz beeinflussen.

In zeitkritischen klinischen Notfällen wie einer Sepsis kann die Therapie nicht auf phänotypische Empfindlichkeitstests warten und stützt sich daher auf vordefinierte empirische Behandlungsstrategien.
Mithilfe großskaliger automatisierter \textit{in vitro}-Experimente, die krankenhausähnliche Transmissionsdynamiken nachbilden, quantifizierten wir, wie solche Strategien die Persistenz plasmidvermittelter Resistenz und das Entstehen doppelter Resistenz beeinflussen.
Über die meisten Szenarien hinweg war es am effektivsten, Patienten gleichzeitig mit zwei Antibiotika zu behandeln (Kombinationstherapie).

Da die Wirksamkeit der Kombinationstherapie durch Arzneimittelinteraktionen (d.\,h.\ Synergie, Antagonismus oder Unabhängigkeit) geprägt wird, entwickelten wir anschließend einen skalierbaren Ansatz, um die Behandlungseffekte auf Bakterienpopulationen bei klinisch relevanten Konzentrationen zu quantifizieren.
Dazu prüften wir, ob biolumineszenzbasierte Messungen geeignet sind, Populationsdynamiken bei klinisch relevanten, hemmenden Antibiotikakonzentrationen zu quantifizieren.
Für 20 antimikrobielle Wirkstoffe verglichen wir Zeitreihen der Lichtintensität mit Kolonie-bildenden Einheiten (CFU) und ergänzten diese Experimente durch Mikroskopie.
Wir fanden, dass Biolumineszenz die Biomassendynamik besser abbildet als die Zellzahldynamik, was insbesondere bei Filamentierung relevant ist.
Andererseits beobachteten wir, dass CFU-basierte Schätzungen durch wirkstoffinduzierte Veränderungen der Kultivierbarkeit und durch Antibiotika-Übertrag (carry-over) verzerrt sein können.

Aufbauend auf dieser biolumineszenzbasierten Methode quantifizierten wir Antibiotikainteraktionen für 15 Wirkstoffpaare, jeweils auf einem Checkerboard, das nicht-hemmende bis hemmende Konzentrationen abdeckt.
Wir fanden, dass Interaktionstypen bei nicht-hemmenden Konzentrationen häufig von denen bei hemmenden Konzentrationen abweichen.
Zudem können Interaktionstypen vom Mischungsverhältnis abhängen und vom gewählten Referenzmodell beeinflusst werden.
Zusammen zeigen diese Ergebnisse das Potenzial der Kombinationstherapie, liefern methodische Einsichten zu ihrer Optimierung und mahnen zur Vorsicht bei der unkritischen Extrapolation von Befunden über den gemessenen Konzentrationsraum hinweg.
