%!TEX root = ../main.tex

\chapter*{Zusammenfassung\markboth{ZUSAMMENFASSUNG}{}}\label{cha:summary}
\addcontentsline{toc}{chapter}{Zusammenfassung}

Seit der Entdeckung des Penicillins bilden Antibiotika einen Grundpfeiler der modernen Medizin und haben unzählige Leben gerettet.
Dieser Erfolg ist jedoch bedroht, da Bakterien gegen die meisten verfügbaren Wirkstoffe Resistenzmechanismen entwickelt haben.
Resistenzen entstehen durch den Erwerb von Resistenzgenen – entweder über Mutationen oder durch horizontalen Gentransfer via mobile genetische Elemente wie Plasmide – und werden abhängig von verabreichten Wirkstoffen, deren Konzentrationen und den Pharmakodynamiken selektiert, die Behandlung und Wirkung miteinander verknüpfen.

Um die Ausbreitung von Resistenzen einzudämmen, sollten Antibiotika idealerweise auf Basis phänotypischer Informationen ausgewählt werden.
In Notfallsituationen ist dies jedoch oft nicht möglich, sodass auf vordefinierte Behandlungsprotokolle zurückgegriffen wird.
Kapitel~2 untersucht, wie diese Standardprotokolle (Behandlungsstrategien) die Persistenz plasmidvermittelter Resistenzen und das Auftreten neuer Doppelresistenzen beeinflussen.
Hierfür führten wir groß angelegte, automatisierte In‑vitro‑Experimente durch, die krankenhausähnliche Übertragungsdynamiken simulieren; zum Einsatz kamen zwei Antibiotika in Kombination mit zwei aus klinischen Isolaten stammenden Resistenzplasmiden.

Die Gabe beider Wirkstoffe an jeden Patienten (Kombinationstherapie) erwies sich in den meisten Szenarien als wirksamste Strategie.
Obwohl theoretische Modelle diesen Erfolg häufig vorhersagen, ist die klinische Datenlage weniger eindeutig.
Ursachen hierfür sind unter anderem synergistische Wirkstoffpaarungen, bei denen der kombinierte Effekt größer ist als die Summe der Einzeleffekte, sowie antagonistische Paarungen, bei denen er kleiner ausfällt.

Bisher wurden Arzneimittelinteraktionen überwiegend bei Konzentrationen unterhalb der minimalen Hemmkonzentration (MIC) untersucht, da Wachstumsraten dort bequem mittels optischer Dichte (OD) erfassbar sind.
OD misst jedoch keinen Bakterienrückgang.
Der klinisch relevante Super‑MIC‑Bereich, in dem Bakterien aktiv abgetötet werden, erfordert koloniebildende Einheiten (CFU), ein arbeitsintensives und wenig durchsatzstarkes Verfahren.

Kapitel~3 prüft daher, ob eine bereits für Sub‑MIC‑Bedingungen validierte, lumineszenzbasierte Hochdurchsatzmethode auch über den gesamten Konzentrationsbereich einsetzbar ist.
Durch gentechnisch veränderte, biolumineszente Bakterien konnten wir zeigen, dass diese Methode die Nettowachstumsrate für Wirkstoffe ohne ausgeprägte Filamentierung präzise erfasst.

Aufbauend darauf untersucht Kapitel~4 Wirkstoff­interaktionen für sechs im Vorversuch bewährte Antibiotika.
Wir analysierten 15 Wirkstoffpaare mit jeweils 144 Konzentrations­kombinationen und entwickelten einen mathematischen Rahmen zur Beschreibung der Pharmakodynamik von Kombinationen.
Durch Vergleich der beobachteten Effekte mit Nullmodellen ohne Interaktion klassifizierten wir Paare als neutral, synergistisch oder antagonistisch und zeigten, dass sich diese Klassifikationen zwischen Sub‑ und Super‑MIC‑Bereichen deutlich ändern können.

Insgesamt verbinden diese Arbeiten experimentelle und theoretische Ansätze, um die Wirkung von Antibiotika auf bakterielle Populationen und die Resistenzentwicklung zu beleuchten.
Sie liefern robuste Methoden zur Quantifizierung von Populationsdynamiken, stärken die Evidenz für Kombinationstherapien und verfeinern unser Verständnis von Arzneimittel­interaktionen über klinisch relevante Konzentrationsbereiche hinweg.
