Since the discovery of penicillin, antibiotics have been a cornerstone of modern medicine.
This achievement is now under threat, as bacteria have evolved resistance to antibiotics across all major drug classes.
Slowing the rise of resistance will require changes in how existing antibiotics are deployed.
In this thesis, we investigate the pharmacodynamics of drug combinations and how multidrug treatment strategies shape the dynamics of plasmid-mediated resistance.

In time-critical clinical emergencies such as sepsis, therapy cannot wait for phenotypic susceptibility testing and therefore relies on predefined empirical strategies.
To assess how these strategies affect the persistence of plasmid-mediated resistance and the emergence of double resistance, we conducted large-scale automated \textit{in vitro} experiments that mimic hospital-like transmission dynamics.
Across most scenarios, treating patients with two antibiotics simultaneously (combination therapy) was the most effective strategy.

Because the effectiveness of combination therapy is shaped by drug interactions (i.e.\ synergy, antagonism, or independence), defined via deviations from expected combined effects.
To assess treatment effects at clinically relevant inhibitory conditions in high throughput, we evaluated whether bioluminescence-based light intensity is a suitable proxy for cell number dynamics.
For 20 antimicrobials, we compared bioluminescence trajectories to colony-forming unit (CFU) counts and supplemented these experiments with microscopy.
We found that bioluminescence aligns better with biomass dynamics than with cell number dynamics, but both measures show similar dynamics when cell size remains approximately constant.
Conversely, we observed that CFU-based estimates can be biased by drug-induced changes in culturability and by antibiotic carry-over.

Using the bioluminescence method, we quantified antibiotic interactions for 15 drug pairs across checkerboards spanning sub-inhibitory to inhibitory concentrations.
We found that interaction types at sub-inhibitory concentrations frequently differ from those at inhibitory concentrations.
In addition, interaction types can vary with mixing ratio and depend on the chosen reference model.
Together, these results highlight the potential of combination therapy, provide methodological insights to optimise it, and caution against uncritical extrapolation of findings across the measured concentration space
