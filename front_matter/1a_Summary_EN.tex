%!TEX root = ../main.tex

Since the discovery of penicillin, antibiotics have been a cornerstone of modern medicine.
This achievement is now under threat, as bacteria have evolved resistance to all major classes of antibiotics.
Slowing the rise of resistance will require changes in how existing antibiotics are deployed.
In this thesis, we investigate the pharmacodynamics of drug combinations and how treatment strategies shape the dynamics of plasmid-mediated resistance.

In time-critical clinical emergencies such as sepsis, therapy cannot wait for phenotypic susceptibility testing and therefore relies on predefined empirical treatment strategies.
Using large-scale automated \textit{in vitro} experiments that mimic hospital-like transmission dynamics, we quantified how such strategies affect the persistence of plasmid-mediated resistance and the emergence of double resistance.
Across most scenarios, treating patients with two antibiotics simultaneously (combination therapy) was the most effective strategy.

Because the effectiveness of combination therapy is shaped by drug interactions (i.e.\ synergy, antagonism, or independence), we next established a scalable way to quantify treatment effects on bacterial populations at clinically relevant concentrations.
To this end, we assessed whether bioluminescence-based measurements are suitable to quantify population dynamics at clinically relevant, inhibitory antibiotic concentrations.
For 20 antimicrobials, we compared bioluminescence trajectories to colony-forming unit (CFU) counts and supplemented these experiments with microscopy.
We found that bioluminescence is a better proxy for biomass dynamics than for cell number dynamics, which matters when cells filament.
Conversely, we observed that CFU-based estimates can be biased by drug-induced changes in culturability and by antibiotic carry-over.

Building on this bioluminescence-based approach, we quantified antibiotic interactions for 15 drug pairs, each on a checkerboard spanning non-inhibitory to inhibitory concentrations.
We found that interaction types at non-inhibitory concentrations frequently differ from those at inhibitory concentrations.
In addition, interaction types can vary with mixing ratio and depend on the chosen reference model.
Together, these results highlight the potential of combination therapy, provide methodological insights to optimise it, and caution against uncritical extrapolation of findings across the measured concentration space.
