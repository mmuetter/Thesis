%!TEX root = ../main.tex

\chapter*{Summary\markboth{SUMMARY}{}}\label{cha:summary}
\addcontentsline{toc}{chapter}{Summary}

Since the discovery of penicillin, antibiotics have been a cornerstone of modern medicine, saving countless lives.
This achievement is now under threat, as bacteria have evolved resistance mechanisms against most available drugs.
Resistance arises through the acquisition of resistance genes—either via mutations or through horizontal transfer on mobile genetic elements such as plasmids—and is selected for according to the administered drugs, their concentrations, and the pharmacodynamics that link treatment to its effect on the bacterial population.

To curb the spread of resistance, antibiotics should ideally be chosen on the basis of phenotypic information.
In emergency situations, however, this is not always feasible.
Therapy then relies on predefined treatment protocols rather than tailored data.
Chapter~2 examines how these default protocols (treatment strategies) influence the persistence of plasmid‑mediated resistance and the emergence of new double resistance.
We performed large‑scale automated in‑vitro experiments that mimic hospital‑like transmission dynamics, using two antibiotics in combination with two resistance plasmids derived from clinical isolates.

We found that administering both drugs to every patient (combination therapy) was the most effective strategy in most scenarios.
Although theoretical models often predict the success of combination therapy, the clinical literature is less conclusive.
This gap has several causes: some drug pairs act synergistically, whereby the combined effect exceeds the sum of single‑drug effects, whereas others act antagonistically, reducing overall efficacy.

Drug interactions have so far been studied mainly at concentrations below the minimal inhibitory concentration (MIC), where growth can be tracked conveniently by optical density (OD).
OD cannot measure bacterial decline, and probing the clinically relevant super‑MIC range—where bacteria are actively killed—requires colony‑forming‑unit (CFU) assays, which are labor‑intensive and unsuited to high throughput.

Chapter~3 therefore evaluates whether a luminescence‑based high‑throughput method, previously validated for sub‑MIC conditions, can also track population dynamics across the entire concentration range.
Using bacteria engineered to emit bioluminescent light, we show that this method accurately estimates net growth rates for drugs that do not induce significant filamentation.

Building on these insights, Chapter~4 explores drug interactions for six antibiotics that performed well in the preceding validation.
We analysed 15 drug pairs across 144 concentration combinations each, introducing a mathematical framework to describe the pharmacodynamics of combinations.
By comparing observed effects to null models of non‑interaction, we classified pairs as neutral, synergistic or antagonistic, and demonstrated that interaction types can shift markedly between sub‑ and super‑MIC conditions.

Together, these studies combine experimental and theoretical approaches to elucidate antibiotic effects on bacterial populations and resistance evolution.
They deliver robust tools for quantifying population dynamics and bolster the case for combination therapy, while refining our understanding of drug–drug interactions across clinically relevant concentration ranges.
