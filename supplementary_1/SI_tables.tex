\begin{table}[H]
    \centering
     \captionof{table}{Number of superinfections ($N_\mathcal{S}$) between high- and low-concentrated $A_r$ and $B_r$ wells and the number of double resistances that emerged ($N_\mathcal{E}$)  under treatment $\vartheta$ across all three experiments.} 
     \label{tab:encounter_advanced}
    \makeCSVtable{r|ccccc}%
            {\bfseries Treatment $\mathbf{\vartheta}$ & $\mathbf{ A^x_r}$  &  $\mathbf{B^x_r}$  & \(\mathbf{N_\mathcal{E}^\vartheta}\) & \(\mathbf{N_\mathcal{S}^\vartheta}\) & \( \mathbf{\frac{N_\mathcal{E}^\vartheta}{N_\mathcal{S}^\vartheta}}\)}%
            {supplementary_1/tables/emergence_advanced.csv}%
            {\treatment & \sa & \sb & \n & \total & \f}
\end{table}


\begin{table}[H]
    \centering
     \captionof{table}{Strains used in this study and their relevant phenotypes. The phenotype in brackets is conferred by the respective plasmid. Cm\textsuperscript{R}: Chloramphenicol resistance, Amp\textsuperscript{R}: Ampicillin resistance, Caz\textsuperscript{R}: Ceftazidime resistance, Tet\textsuperscript{R}: Tetracycline resistance.}
     \label{tab:strainPhenotypes}
    \makeCSVtable{l|ll}%
                {Name & \textbf{Relevant phenotype} &\textbf{Reference}}%
                {supplementary_1/tables/strains_phenotypes.csv}%
                {\name  & \phenotype & \reference}
\end{table}


\begin{table}[H]
    \centering
    \captionof{table}{Statistical comparison of maximum growth rates between the sensitive and plasmid-carrying strains. 
    We used a Mann-Whitney U test for pairwise comparisons, and the p-values were adjusted using the Bonferroni correction. } 
    \begin{tabular}{lrrrr}
\toprule
Comparison & U-statistic & P-value & Corrected P-value & Significant after Bonferroni \\
\midrule
sensitive vs A-resistant & 9.000000 & 0.100000 & 0.300000 & False \\
sensitive vs B-resistant & 3.000000 & 0.700000 & 1.000000 & False \\
sensitive vs double-resistant & 5.000000 & 1.000000 & 1.000000 & False \\
\bottomrule
\end{tabular}

    \label{tab:man_ferroni}
\end{table}


\begin{table}[H]
    \centering
    \captionof{table}{
        Plasmid segregation loss was estimated over 24 hours without treatment and with selective treatment as a control. 
        Frequencies of plasmid retention were compared between $t_0$ -- $t_1$ using the Mann-Whitney~U test. 
        Confidence intervals and the mean frequencies were estimated by bootstrapping the binary data (plasmid retained or lost) pooled across replicates. 
        No significant plasmid loss was observed in either the main data or the control.
    }
    \begin{tabular}{lrlrlrrlr}
\toprule
plasmids & $f(t_0)$ & $CI(t_0)$ & $f(t_1)$ & $CI(t_1)$ & $p$-value & $f(t_1)$ control & $CI(t_1)$ control & $p$-value control \\
\midrule
$p_A$ & 1.00 & (1.0, 1.0) & 1.00 & (1.0, 1.0) & 1.00 & 1.00 & (1.0, 1.0) & 1.00 \\
$p_B$ & 1.00 & (1.0, 1.0) & 0.97 & (0.93, 1.0) & 0.20 & 1.00 & (1.0, 1.0) & 1.00 \\
$p_A$\&$p_B$ & 0.99 & (0.97, 1.0) & 1.00 & (1.0, 1.0) & 0.50 & 1.00 & (1.0, 1.0) & 0.50 \\
\bottomrule
\end{tabular}

    \label{tab:loss_data}
\end{table}

\begin{table}[H]
    \centering
    \caption{95 \% confidence intervals for the final bacterial density measured by colony plating and the maximal growth rates measured by evaluating OD-growth curves.}
    \label{tab:growth_and_density}
    \begin{tabular}{lllll}
\toprule
strain & antibiotic & cmp $[\mu g / ml]$ &             cfu $[1/\mu l]$ & growthrate $[1/h]$ \\
\midrule
   double-resistant &         AB &                  5 & $(2.14 - 3.33) \times 10^5$ &    $(0.43 - 0.86)$ \\
   double-resistant &         AB &                 25 & $(2.68 - 3.99) \times 10^5$ &     $(0.4 - 0.85)$ \\
    A-resistant &          A &                  5 & $(1.58 - 2.62) \times 10^5$ &     $(0.51 - 0.7)$ \\
    A-resistant &          A &                 25 & $(2.05 - 3.21) \times 10^5$ &    $(0.46 - 0.71)$ \\
     sensitive &       None &                  5 & $(1.03 - 1.27) \times 10^6$ &    $(0.68 - 0.74)$ \\
     sensitive &       None &                 25 &  $(1.15 - 1.4) \times 10^6$ &    $(0.55 - 0.96)$ \\
    B-resistant &          B &                  5 & $(4.05 - 5.62) \times 10^5$ &    $(0.45 - 0.74)$ \\
    B-resistant &          B &                 25 & $(3.53 - 5.01) \times 10^5$ &     $(0.4 - 0.87)$ \\
\bottomrule
\end{tabular}

\end{table}

\begin{table}[H]
    \centering
    \begin{minipage}[t]{0.5\textwidth}
        \centering
        \captionof{table}{Definition of resistance profiles (rows) by growth patterns on differently treated \textit{agar plates} (columns). X indicates colony formation, whereas o indicates no growth.}
        \begin{tabular}{r|cccc}
            \toprule
                 & None & A & B & AB  \\ 
             \midrule
             $U$   & o & o & o & o \\
             $S$   & X &o&o&o\\
             $A_r$   & X & X &o&o\\
             $B_r$   & X &o& X &o\\
             $(A_r\&B_r)$   & X & X & X &o\\
             $AB_r$   & X & X & X & X \\
             \bottomrule
        \end{tabular}
        \label{tab:phenotyping}
    \end{minipage}
\end{table}

\begin{table}[H]
    \centering
    \begin{minipage}[t]{0.55\textwidth}
        \centering
        \captionof{table}{Association between bacterial phenotypes (rows) and resistance profiles \(\phi\) (columns). An 'X' denotes that a phenotype is obligatory for a given profile, while a '\(\surd\)' indicates that it is optional. 
        }
        \begin{tabular}{r|cccccc}
            \toprule
                 & $U$ & $S$ & $A_r$ & $B_r$ & $(A_r\&B_r)$ & $AB_r$ \\ 
             \midrule
             sensitive     &  & X &  $\surd$ &   $\surd$ &  $\surd$ &  $\surd$ \\
             A-resistant   & &  & X  &   & X  &  $\surd$ \\
             B-resistant   & & & & X  & X  &  $\surd$  \\
             double-resistant  & &  &  & & & X  \\
             \bottomrule
        \end{tabular}
        \label{tab:bacterial-phenotypes}
    \end{minipage}
\end{table}


\begin{table}[H]
        \centering
        \captionof{table}{Clearance probability of well phenotypes across the three experiments.} 
        \label{tab:clearance}
        \makeCSVtable{r|ccc}%
                { &   $S$ &  $A_r$ & $B_r$ }%
                {supplementary_1/tables/clearance_mean.csv}%
                {\treatmentwith  & \S   & \Ar   & \Br  }
\end{table}


\begin{table}[H]
  \centering
  \caption{\textbf{Mixing rules.}
    During the plate‐handling phase, we mix wells due to infections.
    The resulting phenotype of the two mixed wells can be calculated using this table.
    More than two phenotypes can be combined by applying associative logic.
  }
  \label{tab:mixing_rules}
  \resizebox{\linewidth}{!}{%
    \begin{tabular}{r|cccccc}
      \diagbox[width=20mm]{$\phi_1(\hat{T})$}{$\phi_2(\hat{T})$}
        & $U$ & $S$ & $A_r$ & $B_r$ & $(A_r\&B_r)$ & $AB_r$ \\\hline
      \csvreader[
        separator=semicolon,
        head to column names,
        late after line=\\,
        filter test=\ifnumgreater{\thecsvinputline}{1}
      ]{supplementary_1/tables/mixing_pheno.csv}{
        index=\idx,
        U=\u,    S=\s,
        Ar=\ar,  Br=\br,
        AuB=\aub,ABr=\abr
      }{%
        \idx & \u & \s & \ar & \br & \aub & \abr
      }%
    \end{tabular}%
  }
\end{table}



\begin{table}[H]
    \centering
    \begin{minipage}{.82\linewidth} % Adjust the width as needed
        \centering
        \captionof{table}{Mean parameter leading to $n$ single wins during the sensitivity analysis without preexisting double resistance. Strategies that did not yield at least one single win were excluded.}
        \begin{tabular}{lrrrrrrrr}
\toprule
 & turnover & infection & U & S & A\_r & B\_r & AB\_r & n \\
\midrule
Combination & 0.50 & 0.50 & 0.26 & 0.24 & 0.24 & 0.25 & 0.00 & 9311.00 \\
Mono A & 0.80 & 0.27 & 0.26 & 0.29 & 0.37 & 0.08 & 0.00 & 56.00 \\
Cycling & 0.10 & 0.39 & 0.33 & 0.42 & 0.14 & 0.10 & 0.00 & 4.00 \\
Mixing & 0.11 & 0.33 & 0.12 & 0.49 & 0.38 & 0.01 & 0.00 & 1.00 \\
\bottomrule
\end{tabular}

        \label{tab:best_stats_no_preex}
    \end{minipage}

    \vspace{.5cm}

    \begin{minipage}{.82\linewidth} % Adjust the width as needed
        \centering
        \captionof{table}{Mean parameter leading to $n$ single losses during the sensitivity analysis without preexisting double resistance. Strategies that did not yield at least one single loss were excluded.}
        \begin{tabular}{lrrrrrrrr}
\toprule
 & turnover & infection & U & S & A\_r & B\_r & AB\_r & n \\
\midrule
Mono B & 0.60 & 0.43 & 0.26 & 0.25 & 0.16 & 0.33 & 0.00 & 4132.00 \\
Mono A & 0.55 & 0.43 & 0.27 & 0.25 & 0.36 & 0.12 & 0.00 & 2648.00 \\
Cycling & 0.07 & 0.87 & 0.28 & 0.34 & 0.27 & 0.11 & 0.00 & 8.00 \\
Mixing & 0.07 & 0.90 & 0.69 & 0.25 & 0.01 & 0.05 & 0.00 & 1.00 \\
\bottomrule
\end{tabular}

        \label{tab:worst_stats_no_preex}
    \end{minipage}

    \vspace{.5cm}

    \begin{minipage}{.82\linewidth} % Adjust the width as needed
        \centering
        \captionof{table}{Mean parameter leading to $n$ single wins during the sensitivity analysis with preexisting double resistance. Strategies that did not yield at least one single win were excluded.}
        \begin{tabular}{lrrrrrrrr}
\toprule
 & turnover & infection & U & S & A\_r & B\_r & AB\_r & n \\
\midrule
Combination & 0.61 & 0.47 & 0.21 & 0.18 & 0.22 & 0.22 & 0.17 & 5487.00 \\
Mono A & 0.52 & 0.56 & 0.23 & 0.20 & 0.16 & 0.15 & 0.26 & 365.00 \\
Mixing & 0.13 & 0.30 & 0.23 & 0.21 & 0.18 & 0.15 & 0.22 & 40.00 \\
Cycling & 0.14 & 0.20 & 0.23 & 0.20 & 0.21 & 0.20 & 0.16 & 9.00 \\
\bottomrule
\end{tabular}

        \label{tab:best_stats_preexisting}
    \end{minipage}

    \vspace{.5cm}

    \begin{minipage}{.82\linewidth} % Adjust the width as needed
        \centering
        \captionof{table}{Mean parameter leading to $n$ single losses during the sensitivity analysis with preexisting double resistance. Strategies that did not yield at least one single loss were excluded.}
        \begin{tabular}{lrrrrrrrr}
\toprule
 & turnover & infection & U & S & A\_r & B\_r & AB\_r & n \\
\midrule
Mono B & 0.57 & 0.46 & 0.21 & 0.19 & 0.13 & 0.28 & 0.18 & 4250.00 \\
Mono A & 0.51 & 0.43 & 0.23 & 0.19 & 0.30 & 0.10 & 0.18 & 2359.00 \\
Cycling & 0.42 & 0.72 & 0.23 & 0.21 & 0.14 & 0.14 & 0.28 & 6.00 \\
Mixing & 0.34 & 0.71 & 0.23 & 0.20 & 0.18 & 0.09 & 0.30 & 2.00 \\
\bottomrule
\end{tabular}

        \label{tab:worst_stats_preexisting}
    \end{minipage}

\end{table}




\begin{table}[H]
    \centering
    \captionof{table}{Wins and losses during the sensitivity analysis. With preexisting double resistance. 606 parameter sets yielded an insignificant result.}
    \resizebox{\linewidth}{!}{
    \begin{tabular}{lrrrrrrrr}
\toprule
 & single winner [\%] & single loser [\%] & loser [\%] & winner [\%] & single winner & single loser & loser & winner \\
strategy &  &  &  &  &  &  &  &  \\
\midrule
Combination & 54.87 & 0.00 & 0.95 & 86.76 & 5487 & 0 & 95 & 8676 \\
Cycling & 0.09 & 0.06 & 12.10 & 23.32 & 9 & 6 & 1210 & 2332 \\
Mixing & 0.40 & 0.02 & 14.78 & 20.54 & 40 & 2 & 1478 & 2054 \\
Mono A & 3.65 & 23.59 & 48.57 & 17.49 & 365 & 2359 & 4857 & 1749 \\
Mono B & 0.00 & 42.50 & 62.29 & 8.11 & 0 & 4250 & 6229 & 811 \\
\bottomrule
\end{tabular}

    }
    \label{tab:win_loss_statistic_preex}

    \vspace{1cm}
    \captionof{table}{Wins and losses during the sensitivity analysis without preexisting double resistance. 100 parameter sets yielded an insignificant result.}
    \resizebox{\linewidth}{!}{
    \begin{tabular}{lrrrrrrrr}
\toprule
 & single winner [\%] & single loser [\%] & loser [\%] & winner [\%] & single winner & single loser & loser & winner \\
strategy &  &  &  &  &  &  &  &  \\
\midrule
Combination & 93.11 & 0.00 & 0.00 & 98.35 & 9311 & 0 & 0 & 9835 \\
Cycling & 0.04 & 0.08 & 18.89 & 3.49 & 4 & 8 & 1889 & 349 \\
Mixing & 0.01 & 0.01 & 22.05 & 1.32 & 1 & 1 & 2205 & 132 \\
Mono A & 0.56 & 26.48 & 57.03 & 1.65 & 56 & 2648 & 5703 & 165 \\
Mono B & 0.00 & 41.32 & 63.44 & 1.11 & 0 & 4132 & 6344 & 111 \\
\bottomrule
\end{tabular}

    }
    \label{tab:win_loss_statistic_no_preex}

    \vspace{1cm}
    \captionof{table}{Wins and losses during the sensitivity analysis, using filtered transition probabilities and preexisting double resistance.  659 parameter sets yielded an insignificant result.}
    \resizebox{\linewidth}{!}{
    \begin{tabular}{lrrrrrrrr}
\toprule
 & single winner [\%] & single loser [\%] & loser [\%] & winner [\%] & single winner & single loser & loser & winner \\
strategy &  &  &  &  &  &  &  &  \\
\midrule
Combination & 48.31 & 0.00 & 1.72 & 81.27 & 4831 & 0 & 172 & 8127 \\
Cycling & 0.07 & 0.10 & 11.57 & 24.94 & 7 & 10 & 1157 & 2494 \\
Mixing & 0.59 & 0.03 & 14.34 & 23.40 & 59 & 3 & 1434 & 2340 \\
Mono A & 6.85 & 23.91 & 47.57 & 22.15 & 685 & 2391 & 4757 & 2215 \\
Mono B & 0.00 & 42.97 & 61.45 & 9.57 & 0 & 4297 & 6145 & 957 \\
\bottomrule
\end{tabular}

    }
    \label{tab:win_loss_statistic_clean_preexisting}

    \vspace{1cm}
    \captionof{table}{Wins and losses during the sensitivity analysis with filtered transition matrices and no preexisting double resistance. 8 parameter sets yielded an insignificant result.}
    \resizebox{\linewidth}{!}{
    \begin{tabular}{lrrrrrrrr}
\toprule
 & single winner [\%] & single loser [\%] & loser [\%] & winner [\%] & single winner & single loser & loser & winner \\
strategy &  &  &  &  &  &  &  &  \\
\midrule
Combination & 87.04 & 0.00 & 0.00 & 95.98 & 8704 & 0 & 0 & 9598 \\
Cycling & 0.08 & 0.06 & 14.80 & 6.16 & 8 & 6 & 1480 & 616 \\
Mixing & 0.00 & 0.04 & 19.66 & 2.00 & 0 & 4 & 1966 & 200 \\
Mono A & 3.82 & 27.53 & 55.71 & 5.62 & 382 & 2753 & 5571 & 562 \\
Mono B & 0.00 & 43.67 & 65.10 & 1.41 & 0 & 4367 & 6510 & 141 \\
\bottomrule
\end{tabular}

    }
    \label{tab:win_loss_statistic_clean_no_preex}
\end{table}



\begin{table}[H]
  \begin{minipage}{.49\linewidth}
    \centering
    \caption{$M^{\mathrm{none}}$. Unfiltered transition matrix for untreated wells.}
    \label{tab:m_none}
    \transitionmatrixbody{M_none}
  \end{minipage}
  \hfill
  \begin{minipage}{.49\linewidth}
    \centering
    \caption{$M^{\mathrm{A}}$. Unfiltered transition matrix for wells treated with antibiotic A.}
    \label{tab:m_a}
    \transitionmatrixbody{M_A}
  \end{minipage}
\end{table}

\begin{table}[H]
  \begin{minipage}{.49\linewidth}
    \centering
    \caption{$M^{\mathrm{B}}$. Unfiltered transition matrix for wells treated with antibiotic B.}
    \label{tab:m_b}
    \transitionmatrixbody{M_B}
  \end{minipage}
  \hfill
  \begin{minipage}{.49\linewidth}
    \centering
    \caption{$M^{\mathrm{AB}}$. Unfiltered transition matrix for wells treated with antibiotic AB.}
    \label{tab:m_ab}
    \transitionmatrixbody{M_AB}
  \end{minipage}
\end{table}

\begin{table}[H]
  \begin{minipage}{.49\linewidth}
    \centering
    \caption{$M_1^{\mathrm{none}}$. Unfiltered transition matrix for the first time point in untreated wells.}
    \label{tab:m1_none}
    \transitionmatrixbody{M1_none}
  \end{minipage}
  \hfill
  \begin{minipage}{.49\linewidth}
    \centering
    \caption{$M_1^{\mathrm{A}}$. Unfiltered transition matrix for the first time point in wells treated with antibiotic A.}
    \label{tab:m1_a}
    \transitionmatrixbody{M1_A}
  \end{minipage}
\end{table}

\begin{table}[H]
  \begin{minipage}{.49\linewidth}
    \centering
    \caption{$M_1^{\mathrm{B}}$. Unfiltered transition matrix for the first time point in wells treated with antibiotic B.}
    \label{tab:m1_b}
    \transitionmatrixbody{M1_B}
  \end{minipage}
  \hfill
  \begin{minipage}{.49\linewidth}
    \centering
    \caption{$M_1^{\mathrm{AB}}$. Unfiltered transition matrix for the first time point in wells treated with antibiotic AB.}
    \label{tab:m1_ab}
    \transitionmatrixbody{M1_AB}
  \end{minipage}
\end{table}

\begin{table}[H]
  \begin{minipage}{.49\linewidth}
    \centering
    \caption{$M^{\mathrm{none}}$. Filtered transition matrix for untreated wells.}
    \label{tab:m_none_clean}
    \transitionmatrixbody{M_none_clean}
  \end{minipage}
  \hfill
  \begin{minipage}{.49\linewidth}
    \centering
    \caption{$M^{\mathrm{A}}$. Filtered transition matrix for wells treated with antibiotic A.}
    \label{tab:m_a_clean}
    \transitionmatrixbody{M_A_clean}
  \end{minipage}
\end{table}

\begin{table}[H]
  \begin{minipage}{.49\linewidth}
    \centering
    \caption{$M^{\mathrm{B}}$. Filtered transition matrix for wells treated with antibiotic B.}
    \label{tab:m_b_clean}
    \transitionmatrixbody{M_B_clean}
  \end{minipage}
  \hfill
  \begin{minipage}{.49\linewidth}
    \centering
    \caption{$M^{\mathrm{AB}}$. Filtered transition matrix for wells treated with antibiotic AB.}
    \label{tab:m_ab_clean}
    \transitionmatrixbody{M_AB_clean}
  \end{minipage}
\end{table}

\begin{table}[H]
  \begin{minipage}{.49\linewidth}
    \centering
    \caption{$M_1^{\mathrm{none}}$. Filtered transition matrix for the first time point in untreated wells.}
    \label{tab:m1_none_clean}
    \transitionmatrixbody{M1_none_clean}
  \end{minipage}
  \hfill
  \begin{minipage}{.49\linewidth}
    \centering
    \caption{$M_1^{\mathrm{A}}$. Filtered transition matrix for the first time point in wells treated with antibiotic A.}
    \label{tab:m1_a_clean}
    \transitionmatrixbody{M1_A_clean}
  \end{minipage}
\end{table}

\begin{table}[H]
  \begin{minipage}{.49\linewidth}
    \centering
    \caption{$M_1^{\mathrm{B}}$. Filtered transition matrix for the first time point in wells treated with antibiotic B.}
    \label{tab:m1_b_clean}
    \transitionmatrixbody{M1_B_clean}
  \end{minipage}
  \hfill
  \begin{minipage}{.49\linewidth}
    \centering
    \caption{$M_1^{\mathrm{AB}}$. Filtered transition matrix for the first time point in wells treated with antibiotic AB.}
    \label{tab:m1_ab_clean}
    \transitionmatrixbody{M1_AB_clean}
  \end{minipage}
\end{table}


%%%%%%%%%%%%%%%%%%%%%%%% 20220417
\createANOVATable{20210417U_anova}{treatment strategies}{uninfecteds}{\textit{Prevention}~scenario}

\createTukeyTable{20210417U_tukey}{treatment strategies}{uninfecteds}{\textit{Prevention}~scenario}

\createANOVATable{20210417single_anova}{treatment strategies}{single resistance}{\textit{Prevention}~scenario}

\createTukeyTable{20210417single_tukey}{treatment strategies}{single resistance}{\textit{Prevention}~scenario}

\createANOVATable{20210417double_anova}{treatment strategies}{double resistance}{\textit{Prevention}~scenario}

\createANOVATable{20210417double_tukey}{treatment strategies}{double resistance}{\textit{Prevention}~scenario}


%%%%%%%%%%%%%%%%%%%%%%%% 20220417
\createANOVATable{20220127U_anova}{treatment strategies}{uninfecteds}{\textit{Containment}~scenario}

\createTukeyTable{20220127U_tukey}{treatment strategies}{uninfecteds}{\textit{Containment}~scenario}

\createANOVATable{20220127single_anova}{treatment strategies}{single resistance}{\textit{Containment}~scenario}

\createTukeyTable{20220127single_tukey}{treatment strategies}{single resistance}{\textit{Containment}~scenario}

\createANOVATable{20220127double_anova}{treatment strategies}{double resistance}{\textit{Containment}~scenario}
%% Not significant


%%%%%%%%%%%%%%%%%%%%%%%% 20220417
\createANOVATable{20220412U_anova}{treatment strategies}{uninfecteds}{\textit{Maximum-emergence}~scenario}

\createTukeyTable{20220412U_tukey}{treatment strategies}{uninfecteds}{\textit{Maximum-emergence}~scenario}

\createANOVATable{20220412single_anova}{treatment strategies}{single resistance}{\textit{Maximum-emergence}~scenario}

\createTukeyTable{20220412single_tukey}{treatment strategies}{single resistance}{\textit{Maximum-emergence}~scenario}

\createANOVATable{20220412double_anova}{treatment strategies}{double resistance}{\textit{Maximum-emergence}~scenario}

\createANOVATable{20220412double_tukey}{treatment strategies}{double resistance}{\textit{Maximum-emergence}~scenario}


%% Emergence 
\createANOVATable{emergence_anova}{treatment strategies}{newly emerging double resistance}{\textit{Maximum-emergence}~scenario}

\createTukeyTable{emergence_tukey}{treatment strategies}{newly emerging double resistance}{\textit{Maximum-emergence}~scenario}


\createANOVATable{superinfection_anova}{treatment strategies}{superinfections}{\textit{Maximum-emergence}~scenario}

\createTukeyTable{superinfection_tukey}{treatment strategies}{superinfections}{\textit{Maximum-emergence}~scenario}


\createANOVATable{treatment_emergence_anova}{drug $\vartheta$}{emergence per superinfection}{\textit{Maximum-emergence}~scenario}

\createTukeyTable{treatment_emergence_tukey}{drug $\vartheta$}{emergence per superinfection}{\textit{Maximum-emergence}~scenario}