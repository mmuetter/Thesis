% !TEX root = ../main.tex

\documentclass[../main.tex]{subfiles}

\begin{document}

\chapter{Concluding Remarks}
Antimicrobial resistance (AMR) will remain a significant health threat for the foreseeable future.
Avoiding a post-antibiotic era, in which common infections again become difficult or impossible to treat, therefore requires sustained changes in how antibiotics are produced and deployed.
In this thesis, we focused on how antibiotics can be deployed while limiting selection for resistance.

Given that theoretical and experimental work suggests that multidrug strategies, and in particular combination therapy, can limit the evolution of chromosomal resistance \cite{Angst2021, Bonhoeffer, tepekule, ueker}, in Chapter~2 we asked how these insights extend to clinically important plasmid-borne resistance and to the pharmacodynamics of antibiotic combinations at inhibitory concentrations.
We explored the impact of six treatment arms (combination, cycling, mixing, two monotherapies, and an untreated control) on the epidemiological dynamics of plasmid-mediated resistance in an \textit{in vitro} hospital-ward model, with wells representing patients.
The two plasmids were horizontally transferable and were isolated from clinical strains~\cite{SUttner2024}.
One plasmid conferred resistance to ceftazidime (CAZ), while the other conferred resistance to tetracycline (TET).
Interestingly, we observed a suppressive antagonistic interaction between CAZ and TET, meaning that CAZ alone achieved a stronger within-well clearance effect than the CAZ+TET combination.
Despite these unfavourable conditions, combination therapy was consistently among the best strategies to maximise the number of uninfected patients, and it was the best strategy to prevent the emergence of double resistance.
The superiority of combination therapy in preventing double resistance followed from two strategic advantages.
First, it reduced the probability of \emph{superinfection}, i.e.\ a host becoming infected by both single-resistant strains.
Second, it reduced the probability that double resistance would emerge within a superinfected host via plasmid conjugation.

However, an \textit{in vitro} ward is still a model, and important constraints that shape clinical decision-making---including toxicity, host heterogeneity, and patient-specific pharmacokinetics---are not represented.
Ultimately, the efficacy of strategy-level interventions must be demonstrated in clinical studies.
At present, randomised controlled trials are typically not designed or powered to detect within-patient resistance evolution~\cite{Siedentop2024}, indicating that more targeted and better-powered trials will be required to evaluate how treatment strategies shape resistance evolution in patients.

Although combination therapy performed well in our experiments, treatment success largely depends on the interaction between the drugs applied, which strongly influences resistance evolution~\cite{Gjini2021}.
To explore these drug interactions at scale, we evaluated in Chapter~3 whether bioluminescence can be used to measure population decline at clinically relevant, inhibitory concentrations.
In Chapter~4, we then used this method to test whether interaction patterns measured at sub-inhibitory concentrations predict interactions at inhibitory concentrations.
Specifically, we investigated 15 pairwise drug combinations, each measured on \(12\times 12\) checkerboards spanning sub-inhibitory to inhibitory regimes.
We found that interaction classifications are frequently not transferable between concentration regimes.
In addition we found that interaction types can vary with mixing ratio and depend on the chosen reference model.

These results make generalising statements such as ``this drug pair synergises'' hard to justify.
Furthermore, it is questionable whether interaction labels (synergy, antagonism, or independence) are of practical use.
Synergy implies that combining drugs can increase the killing rate, whereas suppressive antagonism can, in some cases, deselect resistance~\cite{Chait}.
These labels also contain no explicit information about the absolute clearance rate.
As both properties are desireable, it remains debatable whether synergism or suppressive antagonism is more desirable.
Furthermore, these labels  contain no explicit information about the absolute clearance rate.
For example, drugs A and B might antagonise while drugs C and D synergise, but this alone does not indicate whether clearance under A+B is stronger or weaker than under C+D.
Much of this confusion arises because the original question---which drug pairs maximise treatment success while minimising resistance---has been replaced by a heuristic question: which drug pairs synergise.
A more direct route may be to advance  PK/PD models that predict treatment success and resistance risk under realistic dosing.
This thesis can help inform these models by providing empirical data on the pharmacodynamics of drug combinations and by offering methodological insights into how such data can be acquired.

Finally, curbing AMR will require a bundle of complementary measures beyond clinical strategy design.
Approaches that have been proven very effective at reducing AMR frequencies include reducing antibiotic use~\cite{Agerso2012, Vellefaux2012}, strengthening infection control and coordinated outbreak responses~\cite{Edgeworth2020, Schwaber2011}, and expanding vaccination to reduce infections and, in turn, antibiotic exposure~\cite{Kyaw2006}.

In conclusion, continued increases in resistance would carry a substantial societal cost.
However, I believe that a bundle of countermeasures that have already proven effective can curb the rise of AMR.
In this thesis, I showed that combination therapy can be one part of this broader approach, and I hope the work provides insights into how combination pharmacodynamics can be quantified to optimise this strategy.

\end{document}