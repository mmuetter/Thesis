\pagebreak

%%%%%%%%%%%%%%%%%%%%%%
%%%%%%%%%%%%%%%%%%%%%
%%%%%%%%%%%%%%%%%%%%%%

\subsection{Time weighted net growth rates}\label{ssec:netgrowth}
We define $\psi$ the linearly time weighted average of the temporal net growth rate $\hat{\psi}$:
\begin{equation}
  \psi =  \int_{0}^{T} \hat{\psi}(t)\,w(t)\,dt,
  \qquad
  w(t)=\frac{2}{T^{2}}(T-t),
  \label{eq:psi_weighted}
\end{equation}
where  the kernel \(w(t)\) decreases linearly from \(T\) to \(0\) and is normalised such \(\int_0^T w(t)\,dt=1\).
To derive \(\psi\) from the luminescence trajectories \(I(t) = I_0 e^{\int_{0}^{t}\hat{\psi}(u)\,du}\), we first normalise
\begin{equation}
  y(t)=\frac{I(t)}{I(0)}.
  \label{eq:norm}
\end{equation}
We then define the log-normalised trajectory
\begin{equation}
  Y(t)=\ln y(t)=\ln I(t)-\ln I(0)=\int_{0}^{t}\hat{\psi}(u)\,du,
  \label{eq:Ydef}
\end{equation}
and obtain \(\psi\) directly from \(Y(t)\) by integrating over time.
Applying Fubini's theorem yields
\begin{align}
  \int_{0}^{T} Y(t)\, dt
  &=
  \int_{0}^{T} \left( \int_{0}^{t} \hat{\psi}(u)\, du \right) dt
  =
  \int_{0}^{T} (T - u)\, \hat{\psi}(u)\, du,
  \label{eq:intY}
\end{align}
so that
\begin{equation}
  \psi=\frac{2}{T^{2}}\int_{0}^{T}Y(t)\,dt.
  \label{eq:psi_eff}
\end{equation}
The area between two curves (control and treatment) is
\begin{equation}
  \int_{0}^{T}\!\big(Y_{\mathrm{ctrl}}(t)-Y_{\mathrm{treat}}(t)\big)\,dt
  =
  \int_{0}^{T}\!(T-t)\,\big(\hat{\psi}_{\mathrm{ctrl}}(t)-\hat{\psi}_{\mathrm{treat}}(t)\big)\,dt,
  \label{eq:intYdiff}
\end{equation}
and we define the treatment effect as the difference between the corresponding time-weighted net growth rates,
\begin{equation}
  \tau
  =
  \psi_{\mathrm{ctrl}}-\psi_{\mathrm{treat}}
  =
  \frac{2}{T^{2}}\int_{0}^{T}\!\big(Y_{\mathrm{ctrl}}(t)-Y_{\mathrm{treat}}(t)\big)\,dt.
  \label{eq:tau_eff}
\end{equation}

\begin{mdframed}[linewidth=0.8pt,backgroundcolor=gray!5,roundcorner=3pt]
  \textbf{Intuition for \(\psi\) and \(\tau\).}\\

  \emph{Non–monophasic case.}
  If rates vary over time, \autoref{eq:intYdiff} shows that \(\tau\) represents a time–weighted average of instantaneous rate differences,
  \[
    \tau = 2/T^2 \int_{0}^{T}\!(T-t)\,\big(\hat{\psi}_{\mathrm{ctrl}}(t)-\hat{\psi}_{\mathrm{treat}}(t)\big)\,dt,
  \]
  where the weighting factor \((T-t)\) decreases linearly from \(T\) to \(0\).
  Early rate differences thus contribute more strongly than later ones.
  This weighting is desirable because an early population reduction has a bigger therapeutical impact than a delayed effect, even if both reach the same endpoint.

  \medskip

  \emph{Constant–rate case.}
  If the temporal net growth rate  is constant $\hat{\psi}(t) = \hat{\psi}$, \autoref{eq:psi_eff} simplifies to:
  \[\psi = \hat{\psi}.\]
  Accordingly; if the temporal net growth rate of control and treatment condition are constant; the treatment effect (\autoref{eq:intYdiff}) simplifies to:
  \[\tau = \psi_\mathrm{ctrl} - \psi_\mathrm{treat}.\]
  Hence, when trajectories are log–linear, \(\tau\) directly measures the difference in net rates.
\end{mdframed}

%%%%%%%%%%%%%%%%%%%%%%
%%%%%%%%%%%%%%%%%%%%%
%%%%%%%%%%%%%%%%%%%%%%
\pagebreak
\subsection{Quantifying Light Noise.}\label{ssec:lightnoise}
We quantified luminescence leakage by a calibration plate in which six source wells (E5, E12, E20, L5, L12, L20) contained \SI{50}{\micro\liter} of overnight culture and all other wells were empty.
A single luminescence read yielded raw intensities $\hat I$.
For each empty well $i$ we calculated the ratio $I_{\mathrm{empty}}/I_{\mathrm{culture}}$ relative to the nearest source well $j$ at the distance of the grid $d_{ij}$ (with diagonal neighbours at distance $\sqrt{2}$).
For each discrete distance, we summarized ratios by their median to obtain a kernel $r(d)$ (\autoref{fig:light_distribution}).
We estimate the light contribution from well $j$ into well $i$ as $\hat I_{ij}=r(d_{ij})\,\hat I_j$.
For each experimental plate and time point, we corrected each well by subtracting received light noise,
\begin{equation}
  I_i=\hat I_i-\sum_j r(d_{ij})\,\hat I_j,
  \label{eq:noise_corr}
\end{equation}
where the sum is restricted to wells within $d\le 3$.

\begin{comment}
\subsection{Estimating the effective growth rate per well.}
For each well $w$, we normalize the noise-corrected light intensities $I_w(t)$ by the pre-treatment light intensity $I_w(0)$:
\[
y_w(t_i) = \frac{I_w(t_i)}{I_w(0)},
\]
and then log transform:
\[
Y_w(t_i) = \ln y_w(t_i).
\]

The effective net growth rate (\autoref{eq:psi_eff}) is approximated numerically using the trapezoidal rule,
\[
\psi_w
\;\approx\;
\frac{2}{T^{2}}
\sum_{k=0}^{m-1}
\frac{x_{k+1}-x_k}{2}\,
\bigl(Y_w(x_k)+Y_w(x_{k+1})\bigr),
\]
as implemented via \texttt{numpy.trapz}.
\end{comment}

%%%
%%% PD
%%%

\subsection{Drug conditions and response functions}

\paragraph{Single-drug pharmacodynamic curves.}
For each drug \(i\) we describe the relationship between concentration \(c\) and net growth rate by the fitted pharmacodynamic curve
\begin{equation}
  f_i(c)
  =
  \psi_{0}
  -
  \frac{(\psi_{0}-\psi_{\min,i})(c/z_{\mathrm{MIC},i})^{\kappa_i}}
  {(c/z_{\mathrm{MIC},i})^{\kappa_i}-(\psi_{\min,i}/\psi_{0})},
  \qquad
  [0,\infty)\rightarrow[\psi_{\min,i},\psi_{0}].
  \label{eq:single_drug_pd_si}
\end{equation}
On this domain, \(f_i\) is monotone and therefore invertible, with inverse \(f_i^{-1}:[\psi_{\min,i},\psi_{0}]\to[0,\infty)\).

\paragraph{Closed-form inversion.}
Using \autoref{eq:single_drug_pd_si}, the inverse \(f_i^{-1}\) can be written in closed form as
\begin{equation}
  f_i^{-1}(\psi)
  =
  z_{\mathrm{MIC},i}
  \left(
    \frac{\bigl(\psi_0-\psi\bigr)\,\psi_{\min,i}/\psi_0}
    {\bigl(\psi_0-\psi\bigr)-\bigl(\psi_0-\psi_{\min,i}\bigr)}
  \right)^{1/\kappa_i},
  \qquad
  \psi\in[\psi_{\min,i},\psi_0].
  \label{eq:psi_inverse_closed_form}
\end{equation}

\paragraph{Cartesian conditions.}
A \emph{condition} is defined by a set of drug concentrations \(\{c_i\}\), with drug \(i \in \{\mathrm{AMO},\mathrm{CHL},\mathrm{COL},\mathrm{PEN},\mathrm{POL},\mathrm{TET}\}\).
The control condition corresponds to \(\{c_i = 0\ \forall i\}\) and is denoted by \(\emptyset\).

Throughout this work, quantities characterizing a treatment are written as functions of the underlying condition \(\{c_i\}\).
For example, \(\psi(\{c_i\})\) denotes the effective growth rate associated with a condition.
For brevity, we write \(\psi(c_A)\) when only one drug is present, \(\tau(c_A,c_B)\) when two drugs are present, and \(\psi_0\) for the untreated control.

\paragraph{Polar pharmacodynamic curves.}
For two-drug combinations, conditions are parameterized in polar coordinates \((z,\phi)\), and the corresponding effective growth rate is written as
\begin{equation}
  \psi(z,\phi) =
  \psi_{0} -
  \frac{\bigl(\psi_{0}-\psi_{\min}(\phi)\bigr)\,\bigl(z/z_{\mathrm{MIC}}(\phi)\bigr)^{\kappa(\phi)}}
  {\bigl(z/z_{\mathrm{MIC}}(\phi)\bigr)^{\kappa(\phi)}-\psi_{\min}(\phi)/\psi_{0}}.
  \label{eq:polar_pd}
\end{equation}

%%%%%%%%%%%%%%%%%%%%%%
%%%%%%%%%%%%%%%%%%%%%
%%%%%%%%%%%%%%%%%%%%%%
\subsection{Bliss independence}\label{ssec:bliss}
Bliss independence is defined by assuming that the probabilities of a cell being killed by drugs \(A\) and \(B\) within a fixed observation window \(T>0\) are statistically independent \cite{Bliss1939}.
Under this assumption,
\begin{equation}
  p_{AB}^\mathrm{Bliss}=p_A+p_B-p_A\,p_B,
  \label{eq:bliss_prob}
\end{equation}
where \(p_A\) and \(p_B\) denote the single–drug kill probabilities within \(T\).

\paragraph{Time--varying hazards.}
Let \(\hat{\tau}_i(t)\ge 0\) denote a time--dependent kill hazard under treatment \(i\) on \(t\in[0,T]\).
The survival fraction at time \(T\) is \(S_i(T)=\exp\!\big(-\int_0^T\hat{\tau}_i(u)\,du\big)\).
Bliss independence implies multiplicative survival \(S_{AB}(T)=S_A(T)\,S_B(T)\), hence additive cumulative hazards:
\begin{equation}
  \int_0^T \hat{\tau}_{AB}^\mathrm{Bliss}(u)\,du=\int_0^T \hat{\tau}_A(u)\,du+\int_0^T \hat{\tau}_B(u)\,du.
  \label{eq:hazard_additivity}
\end{equation}
To connect this survival formulation to the growth-based log-trajectories, we interpret the instantaneous treatment effect as the reduction in temporal net growth relative to the untreated control, i.e.\ \(\hat{\tau}_i(t)=\hat{\psi}_{\mathrm{ctrl}}(t)-\hat{\psi}_i(t)\).
With \(Y_i(t)=\int_{0}^{t}\hat{\psi}_i(u)\,du\) from \autoref{eq:Ydef}, it follows that
\[
  Y_i(t)=Y_{\mathrm{ctrl}}(t)-\int_0^t \hat{\tau}_i(u)\,du.
\]
Using additive cumulative hazards under Bliss then yields
\begin{equation}
  Y_{AB}^\mathrm{Bliss}(t)=Y_{A}(t)+Y_{B}(t)-Y_{\mathrm{ctrl}}(t).
  \label{eq:Y_additivity}
\end{equation}

Integrating \autoref{eq:Y_additivity} over time yields the Bliss prediction for the effective treatment effect:
\begin{equation}
  \tau^{\mathrm{Bliss}}(c_A,c_B)
  =
  \tau(c_A)
  +
  \tau(c_B).
  \label{eq:tau_eff_bliss}
\end{equation}

\paragraph{Bliss-based interaction index \(\mu\).}
Based on the Bliss expectation for a combination at concentrations \((c_A,c_B)\), we define the interaction index
\begin{equation}
  \mu(c_A,c_B)
  =
  \frac{\tau(c_A)+\tau(c_B)-\tau(c_A,c_B)}
  {\mathrm{median}\!\bigl(\tau(c_A)\bigr)+\mathrm{median}\!\bigl(\tau(c_B)\bigr)}.
  \label{eq:mu}
\end{equation}
To avoid division by zero, normalization is performed using the medians of the monotherapy effects.
The index is only evaluated where a condition \((c_A,c_B)\) is classified as eligible if the combined effect \(\tau(c_A,c_B)\) is significant and at least one of the two monotherapy treatment effects, \(\tau(c_A)\) or \(\tau(c_B)\), is significant.

\begin{mdframed}[linewidth=0.8pt,backgroundcolor=gray!5,roundcorner=3pt]
  \textbf{Interpreting \(\mu\).}\\[3pt]
  Positive values of \(\mu\) indicate synergy (\(\tau(c_A,c_B)>\tau(c_A)+\tau(c_B)\)),
  negative values indicate antagonism, and \(\mu=0\) corresponds to neutral interactions.
\end{mdframed}

%%%%%%%%%%%%%%%%%%%%%%
%%%%%%%%%%%%%%%%%%%%%
%%%%%%%%%%%%%%%%%%%%%%
\subsection{Loewe additivity}\label{ssec:loewe}
Loewe additivity is a form of dose additivity and formalizes self--additivity, i.e.\ that a drug combined with itself should behave like a higher dose of the same drug.
This concept is independent of the particular choice of effect measure, provided the effect can be represented on a shared, monotone dose--response scale.
In our case, this effect scale is the effective net growth rate \(\psi\) (see \autoref{eq:psi_eff}).

For a combination with observed effect \(\psi(c_A,c_B)\), Loewe additivity requires that the fractions of the corresponding monotherapy-equivalent concentrations sum to one,
\begin{equation}
  \frac{c_A}{f_A^{-1}\!\big(\psi(c_A,c_B)\big)}
  +
  \frac{c_B}{f_B^{-1}\!\big(\psi(c_A,c_B)\big)}
  =
  1,
  \label{eq:loewe_core_si}
\end{equation}
where \(f_A\) and \(f_B\) are the single--drug pharmacodynamic curves mapping concentrations to \(\psi\), and \(f_A^{-1}\) and \(f_B^{-1}\) map an effect level \(\psi\) to the corresponding monotherapy-equivalent concentration.
As pharmacodynamic function \(f_i\), we use the fitted single-drug pharmacodynamic curve (\autoref{eq:single_drug_pd_si}) for the well-based interaction analysis, whereas for the surface-based analysis we infer \(f_i\) from the monotherapy edge profiles of the chequerboard surface splines.
The Loewe prediction \(\psi_{\mathrm{Loewe}}(c_A,c_B)\) is obtained by numerically solving \autoref{eq:loewe_core_si} for \(\psi\).
\paragraph{Loewe interaction index (\(\nu\)).}
We quantify deviations from Loewe additivity by the centered dose-equivalence index
\begin{equation}
  \nu(c_A,c_B)
  =
  \frac{c_A}{f_A^{-1}\!\big(\psi(c_A,c_B)\big)}
  +
  \frac{c_B}{f_B^{-1}\!\big(\psi(c_A,c_B)\big)}
  -1.
  \label{eq:fici}
\end{equation}
The index is only evaluated where a condition \((c_A,c_B)\) is classified as eligible if the combined effect \(\tau(c_A,c_B)\) is significant and at least one of the two monotherapy treatment effects, \(\tau(c_A)\) or \(\tau(c_B)\), is significant.

Because \(f_A^{-1}\) and \(f_B^{-1}\) are only defined on \(\mathrm{Im}(f_A)=[\psi_{\min,A},\psi_{0}]\) and \(\mathrm{Im}(f_B)=[\psi_{\min,B},\psi_{0}]\), \(\nu\) is only defined if
\begin{equation}
  \psi(c_A,c_B)\in \mathrm{Im}(f_A)\cap \mathrm{Im}(f_B)=[\max(\psi_{\min,A},\psi_{\min,B}),\,\psi_{0}].
  \label{eq:loewe_domain}
\end{equation}

\begin{mdframed}[linewidth=0.8pt,backgroundcolor=gray!5,roundcorner=3pt]
  \textbf{Interpreting \(\nu\).}\\[3pt]
  By construction \(\nu=0\) corresponds to Loewe additivity, \(\nu<0\) indicates synergy, and \(\nu>0\) indicates antagonism.\\
  Because Loewe additivity requires monotherapy-equivalent concentrations for both drugs at the observed effect level, \(\nu\) can only be evaluated when \(\psi(c_A,c_B)\) lies in the overlap of the two monotherapy effect ranges, i.e.\ \(\psi\ge \max(\psi_{\min,A},\psi_{\min,B})\).
\end{mdframed}

\clearpage
\subsection{Peptide--antibiotic interaction model}\label{ssec:toy_model}
We consider two drugs, \(A\) and \(B\), acting on a population with size trajectory \(x(t)\).
Drug~\(A\) represents a peptide-like effect that causes an instantaneous multiplicative reduction in population size by a factor \(\alpha\in(0,1]\), whereas drug~\(B\) represents a conventional antibiotic that changes the subsequent net growth rate but does not cause an initial drop.
We assume that the combination of two drugs induces a sharp drop by the factor $\alpha$ followed by net growth, determined by drug B:
\begin{align}
  x_{\mathrm{ctrl}}(t) &= x_0\,\exp\!\big(\hat{\psi}_{\mathrm{ctrl}}\,t\big), \notag\\
  x_{B}(t) &= x_0\,\exp\!\big(\hat{\psi}_{B}\,t\big), \notag\\
  x_{A}(t) &= \alpha\,x_{\mathrm{ctrl}}(t), \notag\\
  x_{AB}(t) &= \alpha\,x_{B}(t), \notag
\end{align}
and:
\begin{align}
  Y_{\mathrm{ctrl}}(t) &= \ln x_0+\hat{\psi}_{\mathrm{ctrl}}\,t, \notag\\
  Y_{B}(t) &= \ln x_0+\hat{\psi}_{B}\,t, \notag\\
  Y_{A}(t) &= Y_{\mathrm{ctrl}}(t)+\ln \alpha, \notag\\
  Y_{AB}(t) &= Y_{B}(t)+\ln \alpha. \notag
\end{align}

Using the definition of effective-rate \(\tau_i=\frac{2}{T^2}\int_0^T\!\big(Y_{\mathrm{ctrl}}(t)-Y_i(t)\big)\,dt\), we obtain the following.
\[
  \tau_A=\frac{2}{T^2}\int_0^T\!(-\ln\alpha)\,dt,
  \qquad
  \tau_B=\frac{2}{T^2}\int_0^T\!\bigl(\hat{\psi}_{\mathrm{ctrl}}-\hat{\psi}_B\bigr)t\,dt,
\]
and
\begin{equation}
  \tau_{AB}
  =
  \frac{2}{T^2}\int_0^T\!\Bigl(\bigl(\hat{\psi}_{\mathrm{ctrl}}-\hat{\psi}_B\bigr)t-\ln\alpha\Bigr)\,dt.
  \label{eq:combined_pep}
\end{equation}
Separating the two terms in \autoref{eq:combined_pep} yields
\begin{equation}
  \tau_{AB}=\tau_A+\tau_B,
  \label{eq:toy_model}
\end{equation}
which matches the Bliss expectation for additive treatment effects.

\subsection{Condition-wise interaction inference on the checkerboard}\label{ssec:bootstrap_interactions}
Each condition \(\{c_i\}\) is realized by a pool of replicate wells \(\mathcal{W}(c_A, c_B)\), from which effective growth rates are inferred as described above.
Treatment effects are estimated by bootstrap resampling from the pools of replicate wells.
We draw \(B=200\) pairs consisting of one control well and one condition well, sampled with replacement from \(\mathcal{W}(\emptyset)\) and \(\mathcal{W}(c_A,c_B)\), respectively.
For each draw \(b\), the treatment effect is defined as
\[
  \tau_b(c_A,c_B)
  =
  \psi_b(\emptyset)-\psi_b(c_A,c_B),
\]
yielding a bootstrap distribution of treatment effects.

For combination conditions, the corresponding mono-treatment growth rates \(\psi_b(c_A,0)\) and \(\psi_b(0,c_B)\) are sampled analogously and included in each bootstrap draw, as they are required for interaction analysis.

\begin{comment}
\paragraph{Significance of treatment effects.}
A treatment effect was classified as significant if the corresponding 95-percentile interval excluded zero, and as not significant otherwise.
All subsequent interaction metrics derived from $\tau$, including the Bliss interaction index $\mu_b$ and the Loewe-based $\nu_b$, inherit this bootstrap-based uncertainty quantification.
\end{comment}

\paragraph{Regime-wise aggregation across checkerboard conditions.}
To summarise interactions for a given drug pair within a regime (sub-inhibitory or inhibitory), we combine the condition-wise bootstrap interaction estimates from all eligible checkerboard cells.
In each aggregation draw, we sample 200 eligible conditions with replacement, using sampling probabilities proportional to
\begin{equation}
  w(\phi)=\sin(2\phi).
  \label{eq:weights}
\end{equation}
This excludes monotherapy edges and gives the highest weight to near-equal mixing.
For every sampled condition, we draw one interaction score from its condition-wise bootstrap distribution.
Repeating this procedure yields an aggregated distribution, from which we report the median and central \(95\%\) interval for the regime-level interaction estimate.

\begin{comment}
\paragraph{Grouping inhibitory and sub-inhibitory conditions.}
For a given condition, we counted the number of wells with $\psi>0$ and $\psi<0$ and applied a binomial sign test against the null hypothesis of equal probability.
Conditions were classified as inhibitory if the binomial test supported $\psi<0$, as sub-inhibitory if it supported $\psi>0$, and as not significantly different from zero otherwise ($\alpha = .1)$.
This condition-wise classification is used both to annotate the $\psi$ heatmaps and to define the sub-inhibitory, inhibitory, and not-different pools used in downstream interaction summaries.
\end{comment}

%%%%%%%%%%%%%%%%%%%%%%%%%%%%%%%%
%%%%%%%%%%%%%%%%
%%%%%%%%%%%%%%%%%%%%%%%%%%%%%%%%
\clearpage
\subsection{Continuous interaction surfaces and geometric exploration}\label{ssec:surface}

\paragraph{Density-based restriction of the surface domain.}
For some drugs, the empirical data density decreases at high concentrations as trajectories drop below the detection limit and are excluded.
We therefore restrict the chequerboard to the high-density subdomain.
For each drug pair, we compute, for every grid cell \((i,j)\), the valid fraction \(f(i,j)=n_{\mathrm{valid}}(i,j)/4\) (with \(n_{\mathrm{valid}}\) counting non-\(\mathrm{NaN}\) values of \(\psi\)).
We then fit a monotone-decreasing helper surface \(f_{\mathrm{dens}}(i,j)\) by alternating one-dimensional isotonic regression sweeps (\texttt{sklearn.isotonic.IsotonicRegression}) along both index axes, followed by bivariate spline interpolation on the \((i,j)\) grid (\texttt{scipy.interpolate.RectBivariateSpline}).
Cells with \(f_{\mathrm{dens}}(i,j)<f_{\min}\) are excluded, and all subsequent bootstrap resampling and surface-spline fitting are performed only on the remaining cells.

\paragraph{Bootstrap datasets and coordinate transform.}
For each pair of drugs \((A,B)\) we first collect all wells belonging to that pair together with their estimated effective net growth rates \(\psi\).
Each well is indexed by a normalised concentration pair  \((z_A, z_B)\).
To stabilise interpolation near zero, we transform the axes to \(\mathrm{Z}_A=\log_2(z_A+1)\) and \(\mathrm{Z}_B=\log_2(z_B+1)\).
For each bootstrap draw \(b\) we generate a resampled surface dataset by sampling, with replacement, one well for every occupied grid point, and retaining its triplet \((\mathrm{Z}_A,\mathrm{Z}_B,\psi)\).

\paragraph{Monotone surface construction.}
For each bootstrap dataset we enforce a monotone-decreasing \(\psi\)-surface by alternating one-dimensional isotonic regressions along rows and columns.
This is important to ensure that the backtransformation $z_\phi(\psi)$ is unique.
Specifically, for each column and then for each row we apply an isotonic regression with a nonincreasing constraint using \texttt{sklearn.isotonic.IsotonicRegression} \cite{Pedregosa2011}.
On each monotone grid we fit a bivariate spline \(s_{AB}^{(b)}(\mathrm{Z}_A,\mathrm{Z}_B)\) using \texttt{scipy.interpolate.RectBivariateSpline} with a linear basis functions in both directions.
For each drug combination we calculate a consensus surface $\psi^\mathrm{cons}$ based on the median of the single splines.

\paragraph{Bliss and Loewe interaction indices on the surface.}
Given the consensus surface \(\psi^{\mathrm{cons}}(z_A,z_B)\), we evaluate Bliss and Loewe interaction measures on the same effect scales defined above.
In particular, we use the surface-derived treatment effects \(\tau(\cdot)\) (see \autoref{eq:tau_eff}) to compute the Bliss expectation
\[
  \tau_{\mathrm{Bliss}}(z_A,z_B)=\tau(z_A)+\tau(z_B),
\]
where \(\tau(z_A)\) and \(\tau(z_B)\) are obtained from the consensus edge profiles \(\psi^{\mathrm{cons}}(z_A,0)\) and \(\psi^{\mathrm{cons}}(0,z_B)\), with \(\psi^\mathrm{cons}(\emptyset)=\psi^{\mathrm{cons}}(0,0)\).
We then define a normalized Bliss interaction index on the surface by
\[
  \mu_{\mathrm{surf}}(z_A,z_B)
  =
  \frac{\tau_{\mathrm{Bliss}}(z_A,z_B)-\tau(z_A,z_B)}{\tau_{\mathrm{Bliss}}(z_A,z_B)},
\]
using the same sign convention as \autoref{eq:mu}.

For Loewe additivity, we evaluate \(\nu(z_A,z_B)\) in the same way as \autoref{eq:fici}, using the consensus monotherapy edges to obtain the monotherapy-equivalent doses \(z_A^\ast\) and \(z_B^\ast\) at the observed effect \(\psi^{\mathrm{cons}}(z_A,z_B)\):
\[
  \nu(z_A,z_B)
  =
  \frac{z_A}{z_A^\ast}
  +
  \frac{z_B}{z_B^\ast} -1,
  \qquad
  \psi^{\mathrm{cons}}(z_A^\ast,0)=\psi^{\mathrm{cons}}(0,z_B^\ast)=\psi^{\mathrm{cons}}(z_A,z_B).
\]

%%%%%%%%%%%%%%%%%%%%%%%%%%%%%
%%%%%%%%%%%%%%%% vi
%%%%%%%%%%%%%%%%%%%%%%%%%%%%%
\subsection{Single-drug inoculum effect analysis}\label{ssec:inoculum}
During data analysis, we noted that inocula varied more than expected, which we traced to variability induced by the pintool.
Across all wells, the pre-treatment luminescence standard diviation was ($\sigma(I_0)=6.89\times 10^{4}$\,RLU),  between experiments ($\sigma_{\mathrm{between}}=4.33\times 10^{4}$\,RLU) and within experiments ($\sigma_{\mathrm{within}}=5.47\times 10^{4}$\,RLU).
To investigate whether the observed net growth rates $\psi$ systematically depended on inoculum size, we used the pre-treatment luminescence signal ($I_0$) as a proxy for initial cell density.

For all monotherapy drug concentration conditions with at least \(n \ge 5\) replicates, we fitted an ordinary least-squares regression of \(\psi\) on \(I_0\) to obtain a concentration-specific slope \(\beta_{\mathrm{inoculum}}\) together with a confidence interval, the value \(p\) and the determination coefficient \(R^2\).
These slopes and their uncertainty summaries were visualised as concentration--slope profiles with \(95\%\) intervals in \autoref{fig:inoculum}.
Across AMO, CHL, PEN, and TET, the estimated slopes were negligible across concentrations, whereas COL and POL showed the strongest dependence on $I_0$ at intermediate concentrations (around \SI{1}{\micro\gram\per\milli\liter}).

To assess whether inoculum size varies systematically along the one-dimensional concentration index, we pooled all single-drug wells and mapped each well to its concentration index.
We then performed a permutation test on the ordinary least-squares slope of $\log_{10} I_0$ versus index by permuting $\log_{10} I_0$ across indices to obtain a null distribution.
We found no evidence for a systematic trend of inoculum size with index in our data ($p=0.092$), suggesting that inoculum variation contributes noise but does not introduce a strong directional bias along the concentration series.
Consistent with this, the inoculum effect primarily adds scatter at intermediate concentrations of COL and POL, contributing to the increased spread in the corresponding single-drug PD curves.
