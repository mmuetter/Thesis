\section{SI figures}

% --- Panel helpers with explicit cm sizing (pass full dimensions like 5.9cm) ---
\newcommand{\TimecoursePanel}[3]{% #1 drug, #2 letter, #3 width (dimension)
  \begin{overpic}[width=#3]{chapter_4sup/figures/timecourse_#1.pdf}%
    \put(2,65){\large\textbf{#2}}%
  \end{overpic}
}
\newcommand{\PDCurvePanel}[3]{% #1 drug, #2 letter, #3 width (dimension)
  \begin{overpic}[width=#3]{chapter_4sup/figures/pdcurve_#1.pdf}%
    \put(2,60){\large\textbf{#2}}%
  \end{overpic}
}

% --- Layout constants (dimensions; no spaces) ---
\newcommand{\SingleGridWidth}{18cm}
\newcommand{\SingleColSep}{0.25cm}
\newcommand{\SingleRowSep}{0.6em}
\newcommand{\SinglePanelWidth}{7.5cm} % (= (12cm - 0.25cm)/2)

% --- Figure 1: all timecourses (2 columns × 3 rows) ---
\begin{figure}[h]
  \centering
  \begin{minipage}{\SingleGridWidth}
    \centering
  \TimecoursePanel{amoxicillin}{a)}{\SinglePanelWidth}
\TimecoursePanel{chloramphenicol}{b)}{\SinglePanelWidth}
\TimecoursePanel{colistin}{c)}{\SinglePanelWidth}
\TimecoursePanel{penicillin}{d)}{\SinglePanelWidth}
\TimecoursePanel{polymyxinB}{e)}{\SinglePanelWidth}
\TimecoursePanel{tetracycline}{f)}{\SinglePanelWidth}
\end{minipage}
\captionsetup{width=\SingleGridWidth}
\caption{Single-drug timecourse data for the six antibiotics used in this study.
Panels~(a--f) show the single-drug trajectories across \num{12} concentrations for amoxicillin, chloramphenicol, colistin, penicillin, polymyxin~B, and tetracycline, respectively.
For each concentration we aggregate \num{20} replicate curves, obtained from five independent combination experiments with four technical replicates each.
Lines represent the concentration-wise median over replicates at each measurement cycle, and shaded bands indicate the central 50\,\% percentile interval (25th--75th percentiles).
Time on the horizontal axis corresponds to the per-cycle mean acquisition time across wells to account for \SI{200}{\milli\second} per-well readout staggering.}
\label{fig:single-timecourses}
\end{figure}
\clearpage

% --- Figure 2: all PD curves (2 columns × 3 rows) ---
\begin{figure}[p]
\centering
\begin{minipage}{\SingleGridWidth}
\centering
\PDCurvePanel{amoxicillin}{a)}{\SinglePanelWidth}\hspace{\SingleColSep}%
\PDCurvePanel{chloramphenicol}{b)}{\SinglePanelWidth}
\PDCurvePanel{colistin}{c)}{\SinglePanelWidth}\hspace{\SingleColSep}%
\PDCurvePanel{penicillin}{d)}{\SinglePanelWidth}
\PDCurvePanel{polymyxinB}{e)}{\SinglePanelWidth}\hspace{\SingleColSep}%
\PDCurvePanel{tetracycline}{f)}{\SinglePanelWidth}
\end{minipage}
\captionsetup{width=\SingleGridWidth}
\caption{Single-drug pharmacodynamic (PD) curves corresponding to the timecourse data in \autoref{fig:single-timecourses}.
Panels~(a--f) show the net rate of population change as a function of drug concentration for amoxicillin, chloramphenicol, colistin, penicillin, polymyxin~B, and tetracycline, respectively.
Curves are fitted to the same replicate-aggregated datasets as in \autoref{fig:single-timecourses}, and intervals indicate uncertainty in the fitted PD relationship.}
\label{fig:single-pdcurves}
\end{figure}
\clearpage

%%%%%%%%%%%%%%%%%%%%%%%%%%%%%%%%%%%%%%%%%%%%%%%%%%%%%%%%%%%%%%%%%%%%%%%
%%%%%%%%%%%%%%%%%%%%%%%% INTERACTION FIGURES %%%%%%%%%%%%%%%%%%%%%%%%%%
%%%%%%%%%%%%%%%%%%%%%%%%%%%%%%%%%%%%%%%%%%%%%%%%%%%%%%%%%%%%%%%%%%%%%%%

\newcommand{\InteractionGrid}[4]{%
\begin{figure}[p]
\centering
% #1 = base filename prefix
% #2 = caption text
% #3 = legend filename (empty = no legend)
% #4 = legend width (e.g. 40mm, 0.6\totalW)

\newcommand{\totalW}{140mm}%
\newcommand{\colsep}{1.5mm}%
\newcommand{\PanelW}{(\totalW-2\colsep)/3}%

% Row 1
\begin{overpic}[width=\PanelW]{chapter_4sup/figures/#1_amoxicillin_chloramphenicol.pdf}\put(3,92){\large\textbf{a)}}
\end{overpic}\hspace{\colsep}%
\begin{overpic}[width=\PanelW]{chapter_4sup/figures/#1_amoxicillin_colistin.pdf}\put(3,92){\large\textbf{b)}}
\end{overpic}\hspace{\colsep}%
\begin{overpic}[width=\PanelW]{chapter_4sup/figures/#1_amoxicillin_penicillin.pdf}\put(3,92){\large\textbf{c)}}
\end{overpic}\\

% Row 2
\begin{overpic}[width=\PanelW]{chapter_4sup/figures/#1_amoxicillin_polymyxinB.pdf}\put(3,92){\large\textbf{d)}}
\end{overpic}\hspace{\colsep}%
\begin{overpic}[width=\PanelW]{chapter_4sup/figures/#1_amoxicillin_tetracycline.pdf}\put(3,92){\large\textbf{e)}}
\end{overpic}\hspace{\colsep}%
\begin{overpic}[width=\PanelW]{chapter_4sup/figures/#1_chloramphenicol_colistin.pdf}\put(3,92){\large\textbf{f)}}
\end{overpic}\\

% Row 3
\begin{overpic}[width=\PanelW]{chapter_4sup/figures/#1_chloramphenicol_penicillin.pdf}\put(3,92){\large\textbf{g)}}
\end{overpic}\hspace{\colsep}%
\begin{overpic}[width=\PanelW]{chapter_4sup/figures/#1_chloramphenicol_polymyxinB.pdf}\put(3,92){\large\textbf{h)}}
\end{overpic}\hspace{\colsep}%
\begin{overpic}[width=\PanelW]{chapter_4sup/figures/#1_chloramphenicol_tetracycline.pdf}\put(3,92){\large\textbf{i)}}
\end{overpic}\\

% Row 4
\begin{overpic}[width=\PanelW]{chapter_4sup/figures/#1_colistin_penicillin.pdf}\put(3,92){\large\textbf{j)}}
\end{overpic}\hspace{\colsep}%
\begin{overpic}[width=\PanelW]{chapter_4sup/figures/#1_colistin_polymyxinB.pdf}\put(3,92){\large\textbf{k)}}
\end{overpic}\hspace{\colsep}%
\begin{overpic}[width=\PanelW]{chapter_4sup/figures/#1_colistin_tetracycline.pdf}\put(3,92){\large\textbf{l)}}
\end{overpic}\\

% Row 5
\begin{overpic}[width=\PanelW]{chapter_4sup/figures/#1_penicillin_polymyxinB.pdf}\put(3,92){\large\textbf{m)}}
\end{overpic}\hspace{\colsep}%
\begin{overpic}[width=\PanelW]{chapter_4sup/figures/#1_penicillin_tetracycline.pdf}\put(3,92){\large\textbf{n)}}
\end{overpic}\hspace{\colsep}%
\begin{overpic}[width=\PanelW]{chapter_4sup/figures/#1_polymyxinB_tetracycline.pdf}\put(3,92){\large\textbf{o)}}
\end{overpic}

% Optional legend
\if\relax\detokenize{#3}\relax
\else
\vspace{-2mm}
\begin{center}
\includegraphics[width=60mm]{chapter_4sup/figures/#3}
\end{center}
\fi

\captionsetup{width=\totalW}%
\caption{#2.}%
\label{fig:#1}%
\end{figure}%
\clearpage%
}

\InteractionGrid{psi}{Distribution of the effective net growth rate $\psi$ across all drug combinations measured in checkerboard assays.
Each panel corresponds to one drug pair, and each cell represents a single concentration condition.
Colors indicate the median estimate of $\psi$ across bootstrap resamples.
Symbols indicate whether $\psi$ differs significantly from zero: stars denote significantly positive growth, diamonds significantly negative growth.
These effective growth rates are inferred from time-resolved luminescence trajectories as described in the Methods.}{psi_colorbar_global}{40mm}

\InteractionGrid{mu_interaction_heatmap}{Bliss interaction classification based on bootstrap inference of treatment effects.
For each concentration pair, treatment effects were resampled to obtain a distribution of the Bliss interaction index $\mu$.
Cells are colored according to statistical significance: teal indicates synergistic interactions (95\% confidence interval entirely below zero), orange indicates antagonistic interactions (95\% confidence interval entirely above zero), and white indicates no significant deviation from additivity.
All estimates are based on effective growth rates $\psi$ inferred from checkerboard experiments.}{interaction_legend_bliss}{80mm}

\InteractionGrid{fici_interaction_heatmap}{Loewe interaction classification based on bootstrap inference of treatment effects.
For each concentration pair, bootstrap samples were used to compute a distribution of the fractional inhibitory concentration index (FICI).
Teal indicates significant synergy (95\% confidence interval entirely below 1), orange indicates significant antagonism (95\% confidence interval entirely above 1), and white indicates no significant deviation from Loewe additivity.
All estimates are derived from effective growth rates $\psi$ obtained from checkerboard assays.}{interaction_legend_loewe}{80mm}

\begin{figure}
\centering

\begin{overpic}[width=89mm]{chapter_4sup/figures/loewe_vs_bliss_sup.pdf}
\put(2,96){\large\textbf{a)}}
\end{overpic}

\vspace{4mm}

\begin{overpic}[width=89mm]{chapter_4sup/figures/loewe_vs_bliss_sub.pdf}
\put(2,96){\large\textbf{b)}}
\end{overpic}

\caption{
Comparison of Loewe and Bliss interaction scores in two growth regimes.
(a) Inhibitory (super-MIC) concentrations.
(b) Sub-inhibitory (sub-MIC) concentrations.
Each point represents a drug combination with error bars indicating bootstrap confidence intervals.
Colors encode agreement or disagreement between Loewe and Bliss classifications.
}
\label{fig:si-loewe-vs-bliss-2d}
\end{figure}

%%%%%%%%%%%%%%%%%%%%%%%%%%%%%%%%%%%%%%%%%%%%%%%%%%%%%%%%%%%%%%%%%%%%%%%
%%%%%%%%%%%%%%%%%%%%%%%%%% SURFACE FIGURES  %%%%%%%%%%%%%%%%%%%%%%%%%%%
%%%%%%%%%%%%%%%%%%%%%%%%%%%%%%%%%%%%%%%%%%%%%%%%%%%%%%%%%%%%%%%%%%%%%%%
% FILE TO EDIT: your macros file (e.g., supplementary_1/commands.tex)
\newcommand{\overpicpanel}[6]{%
\begin{overpic}[#2]{#1}%
\pgfmathsetmacro{\panelULx}{#3}%
\pgfmathsetmacro{\panelULy}{#4}%
\pgfmathsetmacro{\panelLRx}{#5}%
\pgfmathsetmacro{\panelLRy}{#6}%
\pgfmathsetmacro{\panelNX}{3}%
\pgfmathsetmacro{\panelNY}{5}%
\pgfmathsetmacro{\panelDX}{(\panelLRx-\panelULx)/(\panelNX-1)}%
\pgfmathsetmacro{\panelDY}{(\panelLRy-\panelULy)/(\panelNY-1)}%
\setcounter{panelk}{0}%
\foreach \rr in {0,...,4}{%
\foreach \cc in {0,...,2}{%
\pgfmathsetmacro{\panelX}{\panelULx+\cc*\panelDX}%
\pgfmathsetmacro{\panelY}{\panelULy+\rr*\panelDY}%
\stepcounter{panelk}%
\StrChar{abcdefghijklmnopqrstuvwxy}{\thepanelk}[\panelLetter]%
\put(\panelX,\panelY){\bfseries \small \panelLetter)}%
}%
}%
\end{overpic}%
}

\begin{figure}
\centering
\overpicpanel{chapter_4sup/figures/surface_psi.pdf}{width=130mm}{1}{98}{39}{22}
\caption{Effective net growth rate surfaces $\psi$ for all drug combinations.
Each panel shows the median effective growth rate estimated from time-resolved luminescence trajectories across the concentration grid.
Colors indicate the magnitude of $\psi$, with negative values corresponding to net killing and positive values to net growth.
All panels share a common color scale to allow direct comparison across drug pairs.}
\label{fig:surface_psi}
\end{figure}

\begin{figure}
\centering
\overpicpanel{chapter_4sup/figures/surface_pd.pdf}{width=130mm}{5}{98}{41}{23}
\caption{
Polar pharmacodynamic curves for all drug combinations.
For each drug pair, we show one-dimensional cuts through the \(\psi\)-surface splines at \(\phi=45^\circ\) (equal mixing in units of \(\mathrm{zMIC}\)), plotting \(\psi\) as a function of the combined dose \(z\).
The blue dotted line marks \(z=1/\sqrt{2}\), corresponding to both single-drug doses being at \(0.5\,\mathrm{zMIC}\), and the blue dashed line marks \(z=\sqrt{2}\), corresponding to both being at \(1\,\mathrm{zMIC}\).
Solid curves show the median surface-derived \(\psi\), with corresponding Bliss- and Loewe-based predictions evaluated along the same trajectory.
}
\label{fig:surface_pd}
\end{figure}

\begin{figure}
\centering
\overpicpanel{chapter_4sup/figures/surface_isoboles.pdf}{width=130mm}{1}{98}{40}{21}
\caption{Isoboles of constant effective growth rate $\psi$ derived from the interpolated response surfaces.
Each panel shows curves of equal $\psi$ in concentration space, illustrating deviations from linear additivity.
Bending toward the origin indicates synergistic interactions, whereas outward bending indicates antagonism.}
\label{fig:surface_isoboles}
\end{figure}

\begin{figure}
\centering
\overpicpanel{chapter_4sup/figures/surface_angular_interaction.pdf}{width=130mm}{5}{98}{40}{23}
\caption{
Angular interaction profiles for all drug combinations.
For each drug pair, we extract isoboles (paths of constant \(\psi\)) from the consensus \(\psi\)-surface and evaluate Bliss- and Loewe-based predictions for \(\psi\) along these paths $(z, \phi)$, plotted as a function of the mixing angle \(\phi\).
Each combination is shown for one sub-inhibitory isobole, the MIC isobole, and one inhibitory isobole.
}
\label{fig:surface_angular_interaction}
\end{figure}

\begin{comment}
\begin{figure}
    \centering
    \overpicpanel{chapter_4sup/figures/topo_fici_grid.pdf}{width=130mm}{1}{98}{39}{22}
    \caption{Loewe interaction classification across all drug combinations.
        Each panel shows the interaction class inferred from the fractional inhibitory concentration index (FICI).
        Yellow indicates significant synergy (95\% confidence interval entirely below 1), orange indicates significant antagonism (95\% confidence interval entirely above 1), and white indicates no significant deviation from additivity.}
    \label{fig:topo_fici_grid}
\end{figure}

\begin{figure}
    \centering
    \overpicpanel{chapter_4sup/figures/topo_mu_grid.pdf}{width=130mm}{1}{98}{39}{22}
    \caption{Bliss interaction classification across all drug combinations.
        Each panel shows the interaction class inferred from the Bliss interaction index $\mu$.
        Yellow indicates significant synergy (95\% confidence interval entirely below 0), orange indicates significant antagonism (95\% confidence interval entirely above 0), and white indicates no significant deviation from Bliss independence.}
    \label{fig:topo_mu_grid}
\end{figure}
\end{comment}

%%%%%%%%%%%%%%%%%%%%%%%%%%%%%%%%%%%%%%%%%%%%%%%%%%%%%%%%%%%%%%%%%%%%%%%
%%%%%%%%%%%%%%%%%%%%%%%%%% PEPTIDE FIGURES  %%%%%%%%%%%%%%%%%%%%%%%%%%%
%%%%%%%%%%%%%%%%%%%%%%%%%%%%%%%%%%%%%%%%%%%%%%%%%%%%%%%%%%%%%%%%%%%%%%%
\begin{comment}
\begin{figure}
    \centering
    \includegraphics[width=0.5\linewidth]{chapter_4sup/figures/colistin_distribution.pdf}
    \caption{Caption}
    \label{fig:placeholder}
\end{figure}

\begin{figure}
    \centering
    \includegraphics[width=0.5\linewidth]{chapter_4sup/figures/colistin_k_fit.pdf}
    \caption{Caption}
    \label{fig:placeholder}
\end{figure}
\end{comment}
\begin{comment}
\begin{figure}[htbp]
        \centering
        \setlength{\unitlength}{1mm}
        \begin{overpic}[height=5cm]{chapter_4sup/figures/colistin_amoxicillin_interaction.pdf}
          \put(1,58){\textbf{(a)}}
          \put(48,60){COL, \SI{1}{\micro\gram\per\milli\liter}}
          \put(-3,24){\rotatebox{90}{AMX, \SI{8}{\micro\gram\per\milli\liter}}}
        \end{overpic}%
        \hspace{4mm}%
        \begin{overpic}[height=5cm]{chapter_4sup/figures/polymyxinB_amoxicillin_interaction.pdf}
          \put(-4,64){\textbf{(b)}}
          \put(40,68){POL, \SI{1}{\micro\gram\per\milli\liter}}
        \end{overpic}
        \\[-6mm]
        \begin{overpic}[height=5cm]{chapter_4sup/figures/colistin_chloramphenicol_interaction.pdf}
          \put(1,58){\textbf{(c)}}
          \put(-3,24){\rotatebox{90}{CHL, \SI{8}{\micro\gram\per\milli\liter}}}
        \end{overpic}%
        \hspace{4mm}%
        \begin{overpic}[height=5cm]{chapter_4sup/figures/polymyxinB_chloramphenicol_interaction.pdf}
          \put(-4,64){\textbf{(d)}}
        \end{overpic}
        \\[-6mm]
        \begin{overpic}[height=5cm]{chapter_4sup/figures/colistin_penicillin_interaction.pdf}
          \put(1,58){\textbf{(e)}}
          \put(-3,24){\rotatebox{90}{PEN, \SI{64}{\micro\gram\per\milli\liter}}}
        \end{overpic}%
        \hspace{4mm}%
        \begin{overpic}[height=5cm]{chapter_4sup/figures/polymyxinB_penicillin_interaction.pdf}
          \put(-4,64){\textbf{(f)}}
        \end{overpic}
        \\[-6mm]
        \begin{overpic}[height=5cm]{chapter_4sup/figures/colistin_tetracycline_interaction.pdf}
          \put(1,58){\textbf{(g)}}
          \put(-3,24){\rotatebox{90}{TET, \SI{2}{\micro\gram\per\milli\liter}}}
        \end{overpic}%
        \hspace{4mm}%
        \begin{overpic}[height=5cm]{chapter_4sup/figures/polymyxinB_tetracycline_interaction.pdf}
          \put(-4,64){\textbf{(h)}}
        \end{overpic}
        \\
        \includegraphics[width=18.3cm]{chapter_4sup/figures/peptide_combi_legend.pdf}
        \caption{
        Time–kill trajectories for combinations of colistin at \SI{1}{\micro\gram\per\milli\liter} (a, c, e, g) or polymyxin B at \SI{1}{\micro\gram\per\milli\liter} (b, d, f, h) with amoxicillin at \SI{16}{\micro\gram\per\milli\liter} (a, b), chloramphenicol at \SI{8}{\micro\gram\per\milli\liter} (c, d), penicillin at \SI{256}{\micro\gram\per\milli\liter} (e, f), or tetracycline at \SI{8}{\micro\gram\per\milli\liter} (g, h).
        We plot the kill curves of the peptides alone using solid lines, the antibiotics alone using dashed lines, and the combinations using dotted lines.
        }
        \label{fig:si_peptide-interactions}
\end{figure}
\end{comment}

%\begin{figure}
%\centering
%\includegraphics[width=\textwidth]{chapter_4sup/figures/blank_dist.pdf}
%\caption{First figure}
%\end{figure}

\begin{figure}
\centering
\includegraphics[width=12cm]{chapter_4sup/figures/dist_ratios.pdf}
\caption{Light noise distribution (shows the fraction of the original light intensity arriving in neighbouring wells; dependent on the distance to the source well; in well lengths $d_W$.}
\label{fig:light_distribution}
\end{figure}

\begin{figure}
\centering
\includegraphics[width=12cm]{chapter_4sup/figures/inoculum_concentration_slopes.pdf}
\caption{Slope of the inoculum effect ($\mathrm{d}\psi/\mathrm{d}I_0$) as a function of drug concentration.
For each drug and concentration, slopes ($\mathrm{d}\psi/\mathrm{d}I_0$) are obtained by regressing the observed net growth rates across inocula.
Crosses denote concentrations with non-significant inoculum effects, whereas stars indicate statistically significant effects.
}
\label{fig:inoculum}
\end{figure}
