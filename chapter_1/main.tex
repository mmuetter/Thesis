% !TEX root = ../main.tex

\documentclass[../main.tex]{subfiles}

\begin{document}
\chapter{introduction}

\section{Antibiotics and Evolution}
Antibiotics are among the most important achievements of modern medicine and save countless lives every day.
This success is increasingly threatened by the global rise of antimicrobial resistance (AMR), which contributed to an estimated 4.71 million deaths in 2021, with 1.14 million deaths directly attributable to resistant infections~\cite{Naghavi2024}.
The rise of AMR is a consequence of bacterial resistance evolution.
Resistance genes arise de novo through mutations, either on the chromosome or on mobile genetic elements.
Once established, these genes can spread vertically—i.e., passed from a mother cell to its offspring during cell division—or horizontally, whereby previously susceptible cells acquire resistance by taking up foreign DNA.
A key vehicle of horizontal gene transfer are plasmids, small circular DNA elements that replicate independently of the chromosome.
Some plasmids are conjugative, meaning they can transfer between bacterial cells via a pilus, a physical tunnel formed during conjugation.
This process enables gene exchange across lineages and even species boundaries~\cite{Frost2005, Smillie2010}.

If the emerged resistance genes can persist in the population, depends on if the resistance is selcted for.
Selection is shaped not only by the administered drugs and their concentration, but also by the pharmacodynamics, which define the relationship between drug concentration and the effect on bacterial population dynamics.
While pharmacokinetics (PK) describes the temporal dynamics of drug concentrations in the host, pharmacodynamics (PD) links these concentrations to their effects on bacterial populations \cite{Regoes2004}.
PD curves, often derived from controlled time-kill or luminescence assays, quantify this concentration–effect relationship and serve as a cornerstone for PK–PD modelling frameworks in infectious disease dynamics.
To better understand the mechanistic underpinnings of these treatment outcomes, we next investigate how antibiotics interact at the population level using pharmacodynamic assays.

\section{Treatment Strategies and Combination Therapy}
Given the growing threat of antibiotic resistance, the ideal scenario is to tailor antimicrobial treatment based on the resistance phenotype of the infecting pathogen. However, in clinical practice this is often not feasible for several reasons: (i) in emergency settings, sometimes immediate treatment is required, leaving no time for diagnostics; (ii) resistance may exist below the detection threshold, rendering diagnostics inconclusive; (iii) in clinical practice, combination regimens can be curtailed or switched because of additive toxicity or other side‑effects, which are mostly not considered in the modeling literature \cite{...}, (iv) in prophylactic contexts, such as surgery or treatment of immunocompromised patients, antibiotics are prescribed preemptively without targeting a pathogen-specific.

In such cases of limited information, clinicians rely on empirical treatment strategies that aim to minimize the risk of selecting for resistance.

Commonly discussed approaches include combination therapy (simultaneous use of multiple antibiotics), mixing (random patient allocation to different antibiotics), and cycling (periodic rotation of antibiotics over time).

Among these, combination therapy --- first introduced in agriculture to prevent resistance in plant pathogens~\cite{Kable1980, Delp1980, Skylakakis1981} --- is often considered the most promising strategy in theoretical models~\cite{Bonhoeffer1997, Tepekule2017,Uecker2021}.
It also plays a central role in clinical protocols for rapidly evolving pathogens such as \textit{HIV}, \textit{Mycobacterium tuberculosis}, and \textit{Plasmodium falciparum}~\cite{Goldberg2012}.

Nonetheless, the clinical evidence remains inconclusive.
A recent meta-analysis found no consistent advantage of combination therapy in preventing resistance across bacterial infections~\cite{siedentop_metaanalysis_2024}.
This discrepancy may arise from several factors: (i) clinical studies are typically not designed to specifically assess resistance outcomes and may thus be underpowered for this purpose~\cite{siedentop_metaanalysis_2024}; (ii) theoretical models often oversimplify treatment dynamics and neglect key biological complexities~\cite{Uecker2021}; (iii) clinical trials may exclude critical pathogens for ethical reasons, such as when omitting combination therapy would pose undue risk; (iv) the effect of drug interactions, which can shape both treatment efficacy and resistance selection, which might increase the variance of the outcomes.

To bridge the gap between theoretical models and clinical studies, Angst et al.~\cite{Angst2021} developed an \textit{in vitro} experimental system that mimics the epidemiological dynamics of hospital wards under different treatment strategies.
They compared the effects of combination therapy, mixing, cycling, monotherapy, and no treatment (control) on the evolution of chromosomal resistance to streptomycin and nalidixic acid in \textit{Escherichia coli}.
Their results showed that combination therapy was the most effective in preventing resistance evolution, while mixing and cycling were less effective, but still outperformed monotherapy in most cases.

In Chapter~2, we refined this experimental framework to study the dynamics of plasmid-mediated resistance evolution under similar clinical conditions.
Specifically, we used two conjugative, compatibility-tested plasmids—originally isolated from hospital patients in a previous study~\cite{Sutter2016} --- that confer resistance to ceftazidime and tetracycline, respectively.
Our results confirmed that combination therapy remains the most effective strategy even for plasmid-borne resistance, with mixing and cycling again performing better than monotherapy in most scenarios.

\section{Population Dynamics under Combination Therapy}
To better understand the potential of combination therapy, a deeper insight into drug interactions is needed.
Drug interaction studies evaluate how the combined effect of two antibiotics compares to a predicted neutral effect derived from their individual actions.
Two classical null models are typically used for this purpose: the Loewe additivity model~\cite{Loewe1926} and the Bliss independence model~\cite{Bliss1939}.
Numerous studies have investigated drug interactions below the minimum inhibitory concentration (MIC)~\cite{Yeh2006}, (\cite{...}) providing valuable insights into sublethal interactions.
However, it remains unclear whether these findings extend to the super-MIC range, where antibiotics are expected to kill bacteria.
This uncertainty arises from the scarcity of studies exploring interactions above the MIC.

The reason for this is methodological: to quantify treatment effects, we estimate the exponential change in time-series population curves.
Below the MIC, bacterial growth is typically quantified using optical density (OD), which allows for high-throughput measurement.
However, OD does not distinguish between live and dead cells, making it unreliable for estimating net growth rates above the MIC.
At super-MIC concentrations, colony-forming unit (CFU) assays remain the gold standard, but they are labor-intensive, costly, and can become unreliable at high drug levels.

To address this limitation, we evaluated in Chapter3 whether luminescence assays\cite{Kishony2003}, commonly used to measure growth below the MIC, remain reliable in the super-MIC range.
The key assumptions for this method are: (i) light intensity is proportional to the number of living bacteria, and (ii) this proportionality remains stable over time.
While (i) holds as long as bacterial density remains low enough to prevent overshadowing and the measurement setup is consistent, (ii) depends on the stability of the specific luminosity---the average light emitted per cell.
To improve stability, we integrated the \textit{lux} operon from the pCS-$\lambda$ plasmid into the bacterial chromosome, thereby minimizing copy number effects that could alter cell-specific luminosity.
We found that for antibiotics that do not induce strong filamentation, specific luminosity remains approximately constant, enabling a reliable approximation of net growth rates in the super-MIC range.

Using this method, Chapter4 explores the pharmacodynamics of drug interactions among six antibiotics: amoxicillin, colistin, chloramphenicol, fosfomycin, polymyxinB, and tetracycline.
In total, we measured interactions for 2160 drug-combination-concentration pairs.
We extended the standard pharmacodynamic formalism to describe combination treatments using polar coordinates and compared the resulting curves to predictions from the Loewe additivity and Bliss independence models.

\end{document}