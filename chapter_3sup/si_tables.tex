\begin{table}[h]
  \centering
  \captionof{table}{
    Drugs used in this study, their MICs, working concentrations, and stock solvents.
    In the MIC column, we report the highest concentration of the dilution series (numerator) and the maximum inhibiting dilution (denominator).
    Kanamycin (\SI{50}{\micro\gram\per\milli\liter}) was used as the selection marker for the \textit{lux} operon.
  }
  
\setlength{\arrayrulewidth}{1pt}
\arrayrulecolor{white}
\begin{tabular}{p{0.14\linewidth} p{0.14\linewidth} p{0.1\linewidth} p{0.11\linewidth} p{0.11\linewidth}
  p{0.22\linewidth}}
  \hline

  \sffamily
  \cellcolor{tablecolor!90}
  \strut \textcolor{white}{\textbf{drug}} &

  \sffamily
  \cellcolor{tablecolor!90}
  \strut \textcolor{white}{\textbf{MIC [\si{\micro\gram\per\milli\litre}]}} &

  \sffamily
  \cellcolor{tablecolor!90}
  \strut \textcolor{white}{\textbf{cwork [\si{\micro\gram\per\milli\litre}]}} &

  \sffamily
  \cellcolor{tablecolor!90}
  \strut \textcolor{white}{\textbf{cwork [MIC]}} &

  \sffamily
  \cellcolor{tablecolor!90}
  \strut \textcolor{white}{\textbf{solvent}} &

  \sffamily
  \cellcolor{tablecolor!90}
  \strut \textcolor{white}{\textbf{supplier}} \\ \hline

  \sffamily
  \cellcolor{tablecolor!10}
  \strut Amoxicillin &

  \sffamily
  \cellcolor{tablecolor!10}
  \strut $\frac{10}{4} = 2.50$ &

  \sffamily
  \cellcolor{tablecolor!10}
  \strut 25 &

  \sffamily
  \cellcolor{tablecolor!10}
  \strut 10 &

  \sffamily
  \cellcolor{tablecolor!10}
  \strut DMSO &

  \sffamily
  \cellcolor{tablecolor!10}
  \strut Sigma, A8523 \\ \hline

  \sffamily
  \cellcolor{tablecolor!20}
  \strut Ampicillin &

  \sffamily
  \cellcolor{tablecolor!20}
  \strut $\frac{100}{128} = 0.78$ &

  \sffamily
  \cellcolor{tablecolor!20}
  \strut 10, 2 &

  \sffamily
  \cellcolor{tablecolor!20}
  \strut 12.8, 2.56 &

  \sffamily
  \cellcolor{tablecolor!20}
  \strut  water &

  \sffamily
  \cellcolor{tablecolor!20}
  \strut Sigma, A9518 \\ \hline

  \sffamily
  \cellcolor{tablecolor!10}
  \strut Cefepime &

  \sffamily
  \cellcolor{tablecolor!10}
  \strut $\frac{4}{256} = 0.02$ &

  \sffamily
  \cellcolor{tablecolor!10}
  \strut 0.15 &

  \sffamily
  \cellcolor{tablecolor!10}
  \strut 9.6 &

  \sffamily
  \cellcolor{tablecolor!10}
  \strut  DMSO &

  \sffamily
  \cellcolor{tablecolor!10}
  \strut ThermoFisher, J66237 \\ \hline

  \sffamily
  \cellcolor{tablecolor!20}
  \strut Ceftazidime &

  \sffamily
  \cellcolor{tablecolor!20}
  \strut $\frac{4}{64} = 0.06$ &

  \sffamily
  \cellcolor{tablecolor!20}
  \strut 0.62 &

  \sffamily
  \cellcolor{tablecolor!20}
  \strut 10 &

  \sffamily
  \cellcolor{tablecolor!20}
  \strut  DMSO &

  \sffamily
  \cellcolor{tablecolor!20}
  \strut Sigma, PHR1847 \\ \hline

  \sffamily
  \cellcolor{tablecolor!10}
  \strut Cefuroxime &

  \sffamily
  \cellcolor{tablecolor!10}
  \strut $\frac{4}{2} = 2.00$ &

  \sffamily
  \cellcolor{tablecolor!10}
  \strut 16 &

  \sffamily
  \cellcolor{tablecolor!10}
  \strut 8 &

  \sffamily
  \cellcolor{tablecolor!10}
  \strut  water &

  \sffamily
  \cellcolor{tablecolor!10}
  \strut Sigma, C4417 \\ \hline

  \sffamily
  \cellcolor{tablecolor!20}
  \strut Chloramphenicol &

  \sffamily
  \cellcolor{tablecolor!20}
  \strut $\frac{128}{64} = 2.00$ &

  \sffamily
  \cellcolor{tablecolor!20}
  \strut 20 &

  \sffamily
  \cellcolor{tablecolor!20}
  \strut 10 &

  \sffamily
  \cellcolor{tablecolor!20}
  \strut  DMSO &

  \sffamily
  \cellcolor{tablecolor!20}
  \strut Sigma, C0378 \\ \hline

  \sffamily
  \cellcolor{tablecolor!10}
  \strut Ciprofloxacin &

  \sffamily
  \cellcolor{tablecolor!10}
  \strut $\frac{1}{128} = 0.01$ &

  \sffamily
  \cellcolor{tablecolor!10}
  \strut 0.08 &

  \sffamily
  \cellcolor{tablecolor!10}
  \strut 10 &

  \sffamily
  \cellcolor{tablecolor!10}
  \strut water &

  \sffamily
  \cellcolor{tablecolor!10}
  \strut Sigma, 17850 \\ \hline

  \sffamily
  \cellcolor{tablecolor!20}
  \strut Colistin &

  \sffamily
  \cellcolor{tablecolor!20}
  \strut $\frac{50}{64} = 0.78$ &

  \sffamily
  \cellcolor{tablecolor!20}
  \strut 1.6 &

  \sffamily
  \cellcolor{tablecolor!20}
  \strut 2.05 &

  \sffamily
  \cellcolor{tablecolor!20}
  \strut  water &

  \sffamily
  \cellcolor{tablecolor!20}
  \strut Sigma, C4461 \\ \hline

  \sffamily
  \cellcolor{tablecolor!10}
  \strut Doripenem &

  \sffamily
  \cellcolor{tablecolor!10}
  \strut $\frac{4}{256} = 0.02$ &

  \sffamily
  \cellcolor{tablecolor!10}
  \strut 0.16 &

  \sffamily
  \cellcolor{tablecolor!10}
  \strut 10.24 &

  \sffamily
  \cellcolor{tablecolor!10}
  \strut  water &

  \sffamily
  \cellcolor{tablecolor!10}
  \strut VWR,  ACRO463870010 \\ \hline

  \sffamily
  \cellcolor{tablecolor!20}
  \strut Fosfomycin &

  \sffamily
  \cellcolor{tablecolor!20}
  \strut $\frac{4}{4} = 1.00$ &

  \sffamily
  \cellcolor{tablecolor!20}
  \strut 8 &

  \sffamily
  \cellcolor{tablecolor!20}
  \strut 8 &

  \sffamily
  \cellcolor{tablecolor!20}
  \strut  water &

  \sffamily
  \cellcolor{tablecolor!20}
  \strut VWR, APOSBIM0107 \\ \hline

  \sffamily
  \cellcolor{tablecolor!10}
  \strut Imipenem &

  \sffamily
  \cellcolor{tablecolor!10}
  \strut $\frac{5}{64} = 0.08$ &

  \sffamily
  \cellcolor{tablecolor!10}
  \strut 1 &

  \sffamily
  \cellcolor{tablecolor!10}
  \strut 12.8 &

  \sffamily
  \cellcolor{tablecolor!10}
  \strut  water &

  \sffamily
  \cellcolor{tablecolor!10}
  \strut Sigma, PHR1796 \\ \hline

  \sffamily
  \cellcolor{tablecolor!20}
  \strut Mecilinam &

  \sffamily
  \cellcolor{tablecolor!20}
  \strut $\frac{10}{128} = 0.08$ &

  \sffamily
  \cellcolor{tablecolor!20}
  \strut 0.78 &

  \sffamily
  \cellcolor{tablecolor!20}
  \strut 10 &

  \sffamily
  \cellcolor{tablecolor!20}
  \strut  DMSO &

  \sffamily
  \cellcolor{tablecolor!20}
  \strut Sigma, 33447 \\ \hline

  \sffamily
  \cellcolor{tablecolor!10}
  \strut Meropenem &

  \sffamily
  \cellcolor{tablecolor!10}
  \strut $\frac{5}{512} = 0.01$ &

  \sffamily
  \cellcolor{tablecolor!10}
  \strut 0.1 &

  \sffamily
  \cellcolor{tablecolor!10}
  \strut 10.24 &

  \sffamily
  \cellcolor{tablecolor!10}
  \strut  water &

  \sffamily
  \cellcolor{tablecolor!10}
  \strut Sigma, PHR1772 \\ \hline

  \sffamily
  \cellcolor{tablecolor!20}
  \strut Penicillin &

  \sffamily
  \cellcolor{tablecolor!20}
  \strut $\frac{500}{32} = 15.62$ &

  \sffamily
  \cellcolor{tablecolor!20}
  \strut 100 &

  \sffamily
  \cellcolor{tablecolor!20}
  \strut 6.4 &

  \sffamily
  \cellcolor{tablecolor!20}
  \strut  water &

  \sffamily
  \cellcolor{tablecolor!20}
  \strut Roth, HP48.2 \\ \hline

  \sffamily
  \cellcolor{tablecolor!10}
  \strut Pexiganan &

  \sffamily
  \cellcolor{tablecolor!10}
  \strut $\frac{64}{32} = 2.00$ &

  \sffamily
  \cellcolor{tablecolor!10}
  \strut 8, 16 &

  \sffamily
  \cellcolor{tablecolor!10}
  \strut 4, 8 &

  \sffamily
  \cellcolor{tablecolor!10}
  \strut  water &

  \sffamily
  \cellcolor{tablecolor!10}
  \strut Sigma, SML3787 \\ \hline

  \sffamily
  \cellcolor{tablecolor!20}
  \strut Piperacillin &

  \sffamily
  \cellcolor{tablecolor!20}
  \strut $\frac{10}{16} = 0.62$ &

  \sffamily
  \cellcolor{tablecolor!20}
  \strut 6.25 &

  \sffamily
  \cellcolor{tablecolor!20}
  \strut 10 &

  \sffamily
  \cellcolor{tablecolor!20}
  \strut  DMSO &

  \sffamily
  \cellcolor{tablecolor!20}
  \strut Sigma, J66419 \\ \hline

  \sffamily
  \cellcolor{tablecolor!10}
  \strut Polymyxin B &

  \sffamily
  \cellcolor{tablecolor!10}
  \strut $\frac{50}{32} = 1.56$ &

  \sffamily
  \cellcolor{tablecolor!10}
  \strut 2.5 &

  \sffamily
  \cellcolor{tablecolor!10}
  \strut 1.6 &

  \sffamily
  \cellcolor{tablecolor!10}
  \strut  water &

  \sffamily
  \cellcolor{tablecolor!10}
  \strut Roth, 0235.1 \\ \hline

  \sffamily
  \cellcolor{tablecolor!20}
  \strut Rifampicin &

  \sffamily
  \cellcolor{tablecolor!20}
  \strut $\frac{40}{16} = 2.50$ &

  \sffamily
  \cellcolor{tablecolor!20}
  \strut 25 &

  \sffamily
  \cellcolor{tablecolor!20}
  \strut 10 &

  \sffamily
  \cellcolor{tablecolor!20}
  \strut  DMSO &

  \sffamily
  \cellcolor{tablecolor!20}
  \strut Sigma, R3501 \\ \hline

  \sffamily
  \cellcolor{tablecolor!10}
  \strut Tetracycline &

  \sffamily
  \cellcolor{tablecolor!10}
  \strut $\frac{10}{32} = 0.31$ &

  \sffamily
  \cellcolor{tablecolor!10}
  \strut 3.12 &

  \sffamily
  \cellcolor{tablecolor!10}
  \strut 10 &

  \sffamily
  \cellcolor{tablecolor!10}
  \strut  DMSO &

  \sffamily
  \cellcolor{tablecolor!10}
  \strut Sigma, T3383 \\ \hline

  \sffamily
  \cellcolor{tablecolor!20}
  \strut Trimethoprim &

  \sffamily
  \cellcolor{tablecolor!20}
  \strut $\frac{10}{128} = 0.08$ &

  \sffamily
  \cellcolor{tablecolor!20}
  \strut 0.78 &

  \sffamily
  \cellcolor{tablecolor!20}
  \strut 10 &

  \sffamily
  \cellcolor{tablecolor!20}
  \strut  DMSO &

  \sffamily
  \cellcolor{tablecolor!20}
  \strut Sigma, T7883 \\ \hline

\end{tabular}
\setlength{\arrayrulewidth}{0.4pt} % Resetting the width to default for future tables
\arrayrulecolor{black} % Resetting the color to default for future tables
  \label{tab:drugs}
\end{table}

\begin{table}
  \captionof{table}{
    Point estimates and 95\% percentile intervals of $\psi_\mathrm{CFU}$, $\psi_I$, $\psi_I^*$, and $\psi_J$ for different treatments.
    Sig$_X$ indicates whether the rate of change of signal $X \in \{I, I^*, J\}$ differs significantly (*) from the distribution of $\psi_\mathrm{CFU}$, or not (n.s.), based on the significance criterion defined in the Methods section.
    Estimates are based on data from the CFU-luminescence assays (see Methods).
  }
  \setlength{\tabcolsep}{3pt}
  \begin{tabular}{lccccccc}
\toprule
 & $\psi_\mathrm{CFU}$ [\si{\per\hour}] & $\psi_I$ [\si{\per\hour}] & sig$_I$ & $\psi_I^*$ [\si{\per\hour}] & sig$_I$* & $\psi_J$ [\si{\per\hour}] & sig$_J$ \\
 &  &  &  &  &  &  &  \\
\midrule
\textbf{ampicillin $10\,\si{\micro\gram\per\milli\litre}$} & -3.23 [-4.11, -2.54] & -2.41 [-2.91, -2.11] & * &  &  & -2.63 [-3.31, -2.29] & n.s. \\
\textbf{ampicillin $2\,\si{\micro\gram\per\milli\litre}$} & -0.81 [-1.19, -0.37] & -0.30 [-0.41, -0.22] & * &  &  &  &  \\
\textbf{amoxicillin $25\,\si{\micro\gram\per\milli\litre}$} & -2.15 [-2.42, -1.88] & -2.04 [-2.21, -1.86] & n.s. &  &  &  &  \\
\textbf{cefepime $0.15\,\si{\micro\gram\per\milli\litre}$} & -1.17 [-1.50, -0.85] & -0.47 [-0.74, -0.24] & * & -0.88 [-1.13, -0.65] & * &  &  \\
\textbf{ceftazidime $0.62\,\si{\micro\gram\per\milli\litre}$} & -0.88 [-1.06, -0.71] & -0.09 [-0.24, 0.04] & * &  &  & -0.38 [-0.52, -0.15] & * \\
\textbf{cefuroxime $16\,\si{\micro\gram\per\milli\litre}$} & -1.59 [-2.41, -0.99] & -1.41 [-1.64, -1.25] & n.s. & -1.62 [-1.92, -1.34] & n.s. &  &  \\
\textbf{chloramphenicol $20\,\si{\micro\gram\per\milli\litre}$} & 0.05 [-0.11, 0.20] & 0.04 [-0.03, 0.10] & n.s. &  &  &  &  \\
\textbf{ciprofloxacin $0.078\,\si{\micro\gram\per\milli\litre}$} & -1.96 [-2.29, -1.64] & 0.83 [0.44, 1.17] & * &  &  & 0.48 [0.09, 0.90] & * \\
\textbf{colistin $1.6\,\si{\micro\gram\per\milli\litre}$} & -1.00 [-1.45, -0.52] & -0.93 [-1.26, -0.66] & n.s. &  &  &  &  \\
\textbf{doripenem $0.16\,\si{\micro\gram\per\milli\litre}$} & -0.54 [-0.83, -0.25] & -0.24 [-0.37, -0.15] & * & -0.37 [-0.53, -0.26] & * &  &  \\
\textbf{fosfomycin $8\,\si{\micro\gram\per\milli\litre}$} & -1.22 [-1.65, -0.83] & -1.12 [-1.37, -0.93] & n.s. &  &  &  &  \\
\textbf{imipenem $1\,\si{\micro\gram\per\milli\litre}$} & -1.42 [-2.06, -0.72] & 0.00 [-0.16, 0.14] & * & -0.32 [-0.45, -0.22] & * &  &  \\
\textbf{mecillinam $0.78\,\si{\micro\gram\per\milli\litre}$} & -0.32 [-0.57, -0.10] & 0.02 [-0.17, 0.17] & * & -0.34 [-0.47, -0.20] & n.s. &  &  \\
\textbf{meropenem $0.1\,\si{\micro\gram\per\milli\litre}$} & -1.85 [-2.36, -1.32] & -0.16 [-0.47, 0.10] & * &  &  & -0.55 [-0.80, -0.21] & * \\
\textbf{penicillin $100\,\si{\micro\gram\per\milli\litre}$} & -2.17 [-2.78, -1.66] & -2.20 [-2.53, -1.94] & n.s. &  &  &  &  \\
\textbf{piperacillin $6.25\,\si{\micro\gram\per\milli\litre}$} & -0.26 [-0.51, 0.02] & 0.09 [-0.01, 0.19] & * &  &  &  &  \\
\textbf{polymyxinB $2.5\,\si{\micro\gram\per\milli\litre}$} & -1.45 [-1.73, -1.22] & -1.41 [-1.87, -1.07] & n.s. &  &  &  &  \\
\textbf{rifampicin $25\,\si{\micro\gram\per\milli\litre}$} & -0.23 [-0.40, 0.04] & -0.15 [-0.23, -0.07] & n.s. &  &  &  &  \\
\textbf{tetracycline $3.125\,\si{\micro\gram\per\milli\litre}$} & -0.06 [-0.12, 0.00] & -0.01 [-0.07, 0.05] & n.s. &  &  &  &  \\
\textbf{trimethoprim $0.78\,\si{\micro\gram\per\milli\litre}$} & -0.61 [-0.76, -0.47] & 0.48 [0.38, 0.59] & * &  &  & 0.34 [0.18, 0.51] & * \\
\textbf{pexiganan $8\,\si{\micro\gram\per\milli\litre}$} & -44.59 [-66.19, -20.08] & -45.95 [-48.21, -43.34] & n.s. &  &  &  &  \\
\textbf{pexiganan $16\,\si{\micro\gram\per\milli\litre}$} & -61.63 [-93.40, 1.26] & -60.27 [-63.06, -56.42] & n.s. &  &  &  &  \\
\bottomrule
\end{tabular}

  \label{tab:comp_stats}
\end{table}

\begin{table}
  \centering
  \captionof{table}{
    Bootstrapped 95\% confidence intervals and point estimates for the length, width, and volume of cells after 2\,h of treatment, estimated from microscopy images.
  Significance was assessed by comparing the confidence intervals of cell volumes for each antibiotic treatment to the untreated control (\texttt{control\_2h}), as described in the Methods section of the main paper.}

  \setlength{\tabcolsep}{3pt}
  \renewcommand{\arraystretch}{1.75}
  \begin{tabular}{lcccccccccr}
\toprule
 & L\_lower & L\_upper & L\_mean & W\_lower & W\_upper & W\_mean & V\_lower & V\_upper & V\_mean & V\_sig \\
\midrule
control\_2h & 2.28 & 4.92 & 3.41 & 0.77 & 1.4 & 1.09 & 2 & 6.51 & 3.89 & ref. \\
amoxicillin & 2.64 & 10.77 & 5.09 & 0.47 & 1.37 & 1.17 & 1.93 & 9.67 & 6.21 & n.s. \\
ampicillin & 3.87 & 27.14 & 12.36 & 0.93 & 1.67 & 1.27 & 5.7 & 30.39 & 16.67 & significant \\
ceftazidime & 32.05 & 68.69 & 54.54 & 0.67 & 1.22 & 0.95 & 11.52 & 80.86 & 40.42 & significant \\
ciprofloxacin & 10.88 & 36.82 & 21.77 & 0.7 & 1.27 & 1.03 & 6.35 & 45.7 & 19.55 & significant \\
colistin & 2.19 & 4.53 & 3.14 & 0.6 & 1.59 & 1.07 & 1 & 8.32 & 3.71 & n.s. \\
fosfomycin & 1.77 & 5.32 & 3.18 & 0.69 & 1.15 & 0.93 & 1.09 & 4.75 & 2.65 & n.s. \\
meropenem & 3.44 & 10.61 & 5.9 & 1.13 & 3.71 & 2.26 & 5.33 & 71.01 & 31.11 & significant \\
rifampicin & 2.41 & 8.38 & 4.77 & 0.59 & 1.4 & 0.98 & 1.15 & 9.44 & 4.32 & n.s. \\
tetracycline & 2.43 & 7.7 & 4.71 & 0.65 & 1.71 & 1.08 & 1.29 & 13.4 & 5.25 & n.s. \\
trimethoprim & 4.67 & 30.74 & 12.19 & 0.65 & 1.29 & 0.93 & 2.64 & 19.79 & 9.05 & significant \\
\bottomrule
\end{tabular}

  \label{tab:morph_post}
\end{table}
\vspace{3cm}

\begin{table}
  \centering
  \captionof{table}{
    Estimated group means and $95\%$ confidence intervals from an ordinary least squares (OLS) model fitted to log-transformed CFU data, collected from the first sampled time point after diluting pexiganan-treated strains in supplemented PBS.
    Grouping is based on the supplement (\ce{CaCl2} or \ce{MgCl2}) and concentration (\SIrange{0}{100}{\milli M}).
    Confidence intervals were computed using heteroscedasticity-consistent standard errors (HC3).
  The compact letter display (cld) indicates groups that are not significantly different by sharing a common letter, based on mutual inclusion of their $95\%$ confidence intervals.}
  \begin{tabular}{lllll} \toprule
group & mean & ci\_lower & ci\_upper & cld \\
\hline
\ce{CaCl2} (0 mM) & -0.00 & -0.00 & 0.00 & a \\
\ce{CaCl2} (1 mM) & 2.77 & -0.56 & 6.09 & b \\
\ce{CaCl2} (10 mM) & 5.06 & 4.92 & 5.20 & c \\
\ce{CaCl2} (100 mM) & 5.85 & 5.38 & 6.31 & d \\
\ce{MgCl2} (0 mM) & -0.00 & -0.00 & 0.00 & a \\
\ce{MgCl2} (1 mM) & 2.77 & -0.56 & 6.09 & b \\
\ce{MgCl2} (10 mM) & 5.31 & 5.20 & 5.42 & e \\
\ce{MgCl2} (100 mM) & 6.36 & 6.27 & 6.46 & f \\
\bottomrule \end{tabular}
  \label{tab:pbs_ols}
\end{table}

\begin{table}
  \centering
  \captionof{table}{Comparison of kill rates [$h^{-1}$] between cultures treated with pexiganan (\SI{16}{\micro\gram\per\milli\liter}) for 5 minutes and cultures exposed to the supernatant collected after the kill assay.
  The two rates differ significantly; the confidence interval of the rate of change of CFU in the supernatant includes zero.}
  \begin{tabular}{lrrr}
\toprule
 & mean & lower & upper \\
experiment &  &  &  \\
\midrule
pexiganan & -46.980603 & -63.153906 & -35.797836 \\
supernatant & 1.281551 & -2.187859 & 5.111196 \\
\bottomrule
\end{tabular}

  \label{tab:supernatant}
\end{table}

\begin{table}
  \centering
  \captionof{table}{Primer sequences used for $\lambda$-red mediated integration of the \emph{luxCDABE} operon into \textit{E. coli}.
  Lowercase letters indicate homology regions binding to the \emph{lux} operon on the plasmid; uppercase letters indicate chromosomal homology regions at the integration site.}

  \label{tab:primers}
  \begin{tabular}{ll}
    \toprule
    Primer name & Sequence (5' $\rightarrow$ 3')\\
    \midrule
    forward primer & CGGTACGGCTGACCATCGGGTGCCAGTGCGGGAGTTTCGT\textbf{acccagtaaggcagcggtatc}\\
    reverse primer & AGTCAGCGATATCCATTTTCGCGAATCCGGAGTGTAAGAA\textbf{taggtctagggcggcgga}\\
    \bottomrule
  \end{tabular}
  \vspace{2cm}
\end{table}