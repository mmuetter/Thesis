\paragraph{Strains.}
We generated a bioluminescent strain by integrating a modified \textit{P. luminescens luxCDABE} operon, driven by the constitutive $\lambda$-Pr promoter, together with a kanamycin resistance cassette (as a marker) into the chromosome of \textit{Escherichia coli} MG1655.
This integration replaced the galK gene and was achieved using $\lambda$-Red recombination (\cite{Datsenko2000}), following a protocol by Hughes \cite[19--26]{Hughes2015}.
The integrated elements were derived from the pCS-$\lambda$ plasmid (\cite{Kishony2003, Bjarnason2003}).
Primers are listed in \sitab{6}.
For all time-course experiments, we prepared three replicate exponential cultures by diluting overnight cultures (grown for approximately 18 hours) 1:100 and growing them to exponential phase for 1--1.5 hours.

\paragraph{Media.}
We used LB (Sigma L3022) as a liquid medium and, as a solid medium for CFU plating, LB with \SI{1.5}{\percent} agar.
Cultures were treated by diluting the tenfold working concentration of one of 20 antimicrobials 1:10.
All MICs, determined by broth microdilution (\cite{EUCAST2025}), and the concentrations used are listed in \sitab{1}.
Working concentrations were centered around the MIC, with some variation due to rounding convenience and variability in repeated MIC tests.
For colistin and polymyxin B, we used lower concentrations, as higher concentrations in our setup consistently yielded too few colonies for meaningful analysis.

Phosphate-buffered saline (PBS, Sigma 79383) was used as the diluent for CFU assays.
If cultures were treated with pexiganan, \SI{100}{\milli M} \ce{MgCl2} was added.
For microscopy, we added \SI{1}{\micro\gram\per\milli\liter} propidium iodide (PI) to the liquid medium and used PBS/PI agar plates (containing PBS with \SI{1.5}{\percent} agar and \SI{1}{\micro\gram\per\milli\liter} PI) as solid medium.

\paragraph{Automated CFU plating.}
We automated the high-throughput colony-count method described by \cite{Jett1997} using an Evo 200 liquid-handling platform (Tecan) integrated with a Liconic STX100 incubator.
The platform handles liquids and automatically moves, images, and incubates plates.
We produced six colony streaks by spotting six \SI{10}{\micro\liter} drops of diluted bacterial culture onto a one‑well agar plate.
Plates were automatically tilted for \SI{7}{\second} on a custom‑built tilter integrated into the handling platform, to spread the drops and distribute the bacteria.
After incubation, plate images were captured using the Pickolo camera (SciRobotics).

The platform is controlled by custom-generated worklists executed in the native software ``Evoware''.
These worklists were generated using the Python package \texttt{pypetting} (version 1.0.1).
We analyzed the captured images of the agar plates using a custom colony-recognition script that automatically identifies colonies and allowed the manual addition of unidentified colonies and the removal of mismatched ones.
All Python classes for generating the worklists and analyzing colonies are available at Zenodo (DOI: \href{https://doi.org/10.5281/zenodo.15261184}{10.5281/zenodo.15261184}).

\paragraph{Luminescence measurements.}
To record the luminescent light intensity, we used an Infinite F200 spectrophotometer plate reader (Tecan), which is also integrated into the liquid-handling platform, with an exposure time of one second.
We set \(20\;\mathrm{rlu}\) as the lower detection limit and excluded all data points below.

\paragraph{Luminescence-CFU assay setup.}
To measure the CFU and light intensity at seven time points, we treated the exponential cultures and then distributed them onto seven (one for each time point) white 384-well plates (Greiner, 781073), with each culture well containing \SI{54}{\micro\liter} medium and \SI{6}{\micro\liter} 10x stock solution.
We adjusted the duration of the experiments between two and five hours, depending on the anticipated kill rate.
For each time point, an assay plate was transferred from the incubator to the plate reader for luminescence measurement.
Subsequently, a dilution series was conducted directly in the white plate and plated using the automated plating method, after which the plate was discarded.

\paragraph{Rapid luminescence-CFU assay setup.}
This experiment is a variation of the \textit{Luminescence-CFU assay setup}, adjusted to measure rapid kill curves for the antimicrobial peptide pexiganan.
In this setup, we captured four time points within \SI{5}{\minute}.
Cultures were treated in a 96-deep-well plate (Greiner, 780285) by adding \SI{100}{\micro\liter} of the 10x stock solution to \SI{900}{\micro\liter} exponential phase culture.
\SI{60}{\micro\liter} of the treated culture was then transferred to a 384-well white plate (Greiner, 781073) and placed in the plate reader for continuous luminescence recording.
For the four CFU time points, samples were taken directly from the deep-well plate, automatically diluted in PBS supplemented with 100mM \ce{MgCl2} in a 96-well plate (Greiner, 655101) to halt the antimicrobial activity and then plated.

\paragraph{Morphology experiments.}
To assess treatment-induced morphological changes, we imaged treated (for 2 hours) and untreated bacteria by spotting \SI{2}{\micro\liter} droplets onto PBS/PI-agar plates.
The spots were cut out and flipped onto Ibidi $\mu$-dishes (Ibidi, 80136) for imaging.
We used an Eclipse Ti2 microscope (Nikon) with a 100x objective connected to a DS-Qi2 Nikon Scientific CMOS (sCMOS) camera to image the bacterial cells.
The microscope setup included an additional 1.5x zoom, which was used only for some images due to unintentional variation.
We estimated the width and length of the bacterial cells using a custom Python script, as described in \siappendix{4}.

\paragraph{Fitting rates of change $\psi$.}
To compare the rates of change of two signals, we first excluded all data below the detection limits (empty plates or light intensity below \SI{20}{rlu}).
We then truncated both signals at the latest time point where both remained above the detection limit, ensuring the same time frame was used for comparison.
Next, we bootstrapped 200 datasets with replacement per signal, while ensuring that each dataset contained more than one time point
For each dataset, we applied a simple regression to fit an exponential function to all time points of each time-kill curve resulting in distributions with 200 rate estimates each.

\paragraph{Significance.}
We classify two distributions of rates as not significantly different (n.s.) if the mean of each distribution falls within the 95\% confidence interval of the other.
Otherwise, we classify them as significantly different (*).
